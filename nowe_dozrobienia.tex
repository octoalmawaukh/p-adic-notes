\textbf{Notacja}.
Oznaczamy ciała przez $\cialo$, pierścienie: $\pierscien$, grupy: $\grupa$, przestrzenie liniowe: $\liniowa$, kule: $\kula$.
Indeks sumowania to zazwyczaj $k$ lub $i$, pod literą $j$ zawsze kryje się jednostka urojona, zaś $m$ i $n$ to najczęściej jakieś liczby całkowite.
Elementy ciała to $x, y, z$, promienie kul: $r, R$, bliżej nieokreślone stałe: $C$.
Kule domknięte odróżniamy od otwartych nawiasami ($\kula[\cdot, \cdot]$ kontra $\kula(\cdot, \cdot)$), jako że domknięcie kuli otwartej zazwyczaj nie jest kulą domkniętą o tym samym promieniu.
Wykładnikami są greckie litery: $\alpha, \beta$, niekoniecznie naturalnymi, ale $\lambda$ to tylko mnożnik.
