\section{Regionalnie czy wszechstronnie?}
Poznany dopiero co lemat Hensela pozwala zazwyczaj łatwo sprawdzić, czy wielomian o całkowitych współczynnikach ma zera w $\Z_p$, jako że wystarczy ich poszukać w $\mathbb F_p$.
Podobnie jest na prostej rzeczywistej: jeśli w dwóch punktach ciągła funkcja przyjmuje wartości różnych znaków, to między nimi musi przeciąć oś argumentów.

Powiedzmy, że interesują nas jednak pierwiastki żyjące w $\Q$.
Pierwiastki z $\Q$ przenoszą się w oczywisty sposób do każdego z ciał $\Q_p$ dla $p \le infty$.
Trzeba o tym myśleć tak: ciała $p$-adyczne są odpowiednikami ciał rozwinięć Laurenta i dają ,,lokalną'' informację ,,blisko'' $p$.
Fakt, że pierwiastki przenoszą się z $\Q$ do $\Q_p$ oznacza bowiem, że ,,globalny'' pierwiastek jest też ,,lokalnym'' dla każdego $p$, czyli ,,wszędzie''.
Ale czy to rozumowanie daje się odwrócić?

\begin{fakt}
	Liczba $x \in \Q$ jest kwadratem, wtedy i tylko wtedy gdy jest nim w każdym $\Q_p$.
\end{fakt}

Już Helmut mógł to zrobić, jednak to Hasse jako pierwszy stwierdził jawnie, że złożenie w całość lokalnych informacji powinno dawać globalną wiedzę.
Zbyt niejasne, żeby móc nazwać twierdzeniem:

\begin{fakt}[reguła lokalno-globalna]
	Istnienie rozwiązań równania diofantycznego w $\Q$ można wykryć badając lokalne równania, w $\Q_p$.
\end{fakt}

Reguła brzmi naprawdę pięknie, jednak w obecnej, bardzo naiwnej postaci, jest po prostu fałszywa, jak pokazują następujące przykłady:

\begin{przyklad}
	$(x^2 -2)(x^2-17)(x^2-34) = 0$.
\end{przyklad}

\begin{przyklad}
	$x^4 - 2 y^2 = 17$.
\end{przyklad}

\begin{twierdzenie}[Hasse, Minkowski]
	Forma kwadratowa nad ciałem liczbowym $\cialo$ (na przykład $\Q$) reprezentuje nietrywialnie zero w $\cialo$, wtedy i tylko wtedy gdy reprezentuje je w każdym uzupełnieniu $\cialo$.
	Taką forme nazywamy izotropową.
\end{twierdzenie}

\begin{proof}
	Zbyt trudny (przez wyrwy w wiedzy o kwadratowych formach), nawet dla samego $\cialo = \Q$.
	Można go jednak znaleźć w pierwszej połowie książki Serre'a (\cite{serre73}), jako kulminację pierwszej jej połowy.
\end{proof}

Zwracamy uwagę na to, że to rozwiązuje właściwie kompletnie nasz problem.
Stosowna wersja lematu Hensela pokazuje, jak w skończonym czasie rozstrzygnąć, co dzieje się w $\Q_p$.
Liczb pierwszych jest nieskończenie wiele, ale okazuje się, iż można napisać procedurę komputerową, która sobie z tym radzi.

\begin{historia}[Hasse Helmut]\end{historia}

\begin{historia}[Minkowski Hermann]\end{historia}

Ograniczymy się do rozwiązania tylko jednego równania: $ax^2 + by^2 + cz^2 = 0$ dla wymiernych $a, b,c $.
Poczynimi kilka założeń: $abc \neq 0$ jest bezkwadratowa oraz $a,b,c \in \Z$, gdyż możemy.
Wynika stąd, że $a,b,c$ są parami względnie pierwsze i różnych znaków (patrz $p= \infty$!).

\begin{fakt}
	Jeśli liczba pierwsza $p > 2$ nie dzieli $abc$, to istnieją liczby $x_0, y_0, z_0 \in \Z$, nie wszystkie podzielne prez $p$, że $ax_0^2 + by_0^2 + cz_0^2 = 0$.
\end{fakt}

\begin{proof} %Specjalny przypadek twierdzenia Chevalleya-Warninga.
	To specjalny przypadek tw. Chevalleya i Warninga, ale podamy bardziej bezpośredni dowód.
	Istnieje $p^3$ trójek $(x,y,z)$ w $\mathbb F_p^3$, ale ile z nich pasuje do równania?

	Użyjemy nikczemnej sztuczki: $(ax^2+by^2+cx^2)^{p-1}$ jest równe $1$, wtedy i tylko wtedy gdy trójka nie jest rozwiązaniem ($0$ w przeciwnym przypadku), wynika to z małego twierdzenia Fermata.
	Liczba nierozwiązań to $\sum (ax^2+by^2+cz^2)^{p-1}$ (modulo $p$).

	Rozwijamy potęgi i dostajemy sumy postaci $\sum \lambda x^{2i}y^{2k}z^{2l}$ z $2i+2k+2l = 2(p-1)$ i $\lambda \in \Z$.
	Każda z nich jest zerem modulo $p$: 
	przynajmniej jedna z $2i, 2k, 2l$ jest mniejsza od $p-1$ (powiedzmy, $2i$).
	Wtedy nasza suma to $\sum_{y,z} (\lambda y^{2k} z^{2l} \sum_x x^{2i})$.
	Przywołany lemat z poniżej kończy dowód, gdyż $p$ dzieli liczbę rozwiązań, a znamy już jedno (trywialne).
\end{proof}

\begin{lemat}
	Jeśli $0 \le n \le p-1$, to $p$ dzieli $\sum_{i=0}^{p-1} i^n$.
\end{lemat}

\begin{proof}

	Wybierzmy takie $y$, że $y^n \neq 1$ mod $p$.
	Wtedy
	\[
		0 \equiv \sum_{i=0}^{p-1} i^n - \sum_{i=0}^{p-1} (yi)^n = (1-y^n) \sum_{i = 0}^{p-1} i^n \qedhere
	\]
\end{proof}

Niech $(x_0, y_0, z_0)$ będzie jakimś rozwiązaniem modulo $p$, takim że $p$ nie dzieli $x_0$ (jeśli jest inaczej, permutujemy nazwy).
Stosujemy lemat Hensela wobec $ax^2 + by_0^2 + cz_0^2 = 0$ i znajdujemy nietrywialne rozwiązanie $(x, y_0, z_0)$.

To jeszcze nie koniec.
Załóżmy teraz, że $p = 2$, ale $a,b,c$ są nieparzyste.
Jeśli rozwiązanie $(x, y, z)$ istnieje, to bez straty ogólności nie wszystkie niewiadome leżą w $2\Z_2$ -- jeśli tak nie jest, mnożymy przez odpowiednią potęgę dwójki.

Po redukcji mod $2\Z_2$ widzimy, że $y,z$ są jednościami $2$-adycznymi, $x$ dzieli się przez $2$.
Kwadrat $2$-adycznej jedności leży w $1+4\Z_2$, zaś kwadrat czegoś z $2\Z_2$ leży w $4\Z_2$. Redukując modulo $4\Z_2$ dostajemy więc: $b+ c \equiv 0$ mod $4$.
Okazuje się, że warunek ten jest nie tylko konieczny, ale też wystarczający.

\begin{lemat}
	Równanie $ax^2 + by^2 + cZ^2 = 0$ ma nietrywialne rozwiązanie w $\Q_2$, gdy $2 \nmid abc$ i $4$ dzieli sumę dwóch z $a, b, c$.
\end{lemat}

\begin{proof}
	Szukamy początkowego rozwiązania $(x_0, y_0, z_0)$, dla którego $8 \mid ax_0^2 + by_0^2 + cz_0^2$.
	Jeśli $8 \mid a+b$, to kładziemy $x_0 = 1$, $y_0 = 1$, $z_0= 0$.
	Jeśli nie, to $z_0 = 2$, $x_0 = y_0 = 1$.

	Stosujemy lemat Hensela.
\end{proof}

\begin{lemat}
	Równanie $ax^2 + by^2 + cz^2 = 0$ ma nietrywialne rozwiązanie w $\Q_2$, gdy jedna z $a, b, c$ jest parzysta, zaś suma dwóch lub trzech z nich dzieli się przez 8.
\end{lemat}

\begin{proof}
	Załóżmy, że $2$ dzieli tylko $a$ oraz że $ax^2 + by^2 + cz^2 = 0$.
	Możemy przyjąć, że któraś z $x, y,z$ jest $2$-adyczną jednością, zaś wszystkie leżą w $\Z_2$.
	Kwadrat $2$-adycznej jedności leży w $1 + 8 \Z_2$, zatem $0 = ax^2 + by^2 + cz^2 \equiv b +c$ (mod $8$), jeśli $x \in 2\Z_2$ (wtedy $y,z$ muszą być $2$-adycznymi jednościami).

	Jeśli $x$ jest $2$-adyczną jednością, to $y, z$ i tak też muszą nimi być, co prowadzi nas do $a+b+c \equiv 0$ mod $8$.
	
	Twierdzenie odwrotne jest prawdziwe na mocy uogólnionego lematu Hensela.
\end{proof}

\begin{lemat}
	Jeżeli \prawo{Gouvea\\Prob. 131} $p \neq 2$ dzieli $a$, to równanie ma nietrywialne rozwiązanie dokładnie wtedy, gdy $-b/c$ to kwadratowa reszta mod $p$.
\end{lemat}

\begin{proof}
	Ponownie, lemat Hensela.
\end{proof}

\begin{fakt}
	Niech liczby $a,b,c \in \Z$ będą parami względnie pierwsze, bezkwadratowe.
	Równanie $ax^2+by^2+cz^2 = 0$ posiada w $\Q$ nietrywialne rozwiązania, wtedy i tylko wtedy gdy:
	\begin{enumx}
		\item ($a,b,c$ nie są tego samego znaku)
		\item każdy nieparzysty dzielnik pierwszy liczby $a$ posiada $r \in \Z$, że $p \mid b+r^2c$, podobnie dla $b$ i $c$
		\item jeśli $2 \nmid abc$, to $4$ dzieli sumę pewnych dwóch z $a$, $b$, $c$.
		\item jeśli $2 \mid a$, to $8$ dzieli $b+c$ lub $a+b+c$ (podobnie $b$ i $c$).
	\end{enumx}

	Pierwszy warunek wynika z pozostałych.
\end{fakt}

Bezpośredni dowód można znaleźć w rozdziałach 3 -- 5 książki [Cas91].
Strategią jest użycie trzech warunków, a także ,,geometrii liczb'' Minkowskiego do pokazania, że możliwe jest znalezienie rozwiązania $(x, y, z)$ spełniającego nierówność
\[
	|a| x^2 + |b| y^2 + |c| z^2 < 4 |abc|.
\]
