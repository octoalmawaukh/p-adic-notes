\section{Cewka}
Ciała \prawo{Rbrt\\1.App} $\R$ i $\Q_p$ połączyć można w interesującą grupę topologiczną, solenoid.
Przedstawimy jej konstrukcję i własności.

%Kanoniczne homomorfizmy  (dla $n \ge 0$) tworzą układ rzutowy $(\R/p^n \Z, \varphi_n)_{n \ge 0}$ grup topologicznych.

\begin{definicja}
	{Solenoid $p$-adyczny} $\mathbb S_p$ to rzutowa granica układu grup $\R/p^n\Z$ z kanonicznymi morfizmami $\varphi_n \colon \R/p^{n+1} \Z \to \R/p^n\Z$, $x$ mod $p^{n+1} \Z \mapsto x$ mod $p^n \Z$
\end{definicja}

$\mathbb S_p$ jest zwartą grupą abelową z kanonicznymi rzutami $\psi_n \colon \mathbb S_p \to \R/p^n\Z$, morfizmami (grup), które są ciągłymi surjekcjami.
W szczególności ($\psi = \psi_0$) solenoid jest nakryciem okręgu.
Krótki ciąg dokładny $0 \to \Z_p \to \mathbb S_p \to \R/\Z \to 0$ przedstawia okrąg jako iloraz solenoidu, zaś solenoid jako nakrycie okręgu o włóknie $\Z_p$.

Powszechnie \prawo{Rbrt\\1.App.2} wiadomo, że każdej liczbie całkowitej $m \ge 1$ odpowiada jedyna cykliczna podgrupa rzędu $m$ w okręgu, czyli $m^{-1} \Z/\Z \subseteq \R/\Z$.
Prawdą jest też:

\begin{fakt}
	Jeśli $m \ge 1$ nie jest krotnością $p$, to dokładnie jedna cykliczna podgrupa $\mathbb S_p$ ma rząd $m$.
\end{fakt}

\begin{proof}
	Oznaczmy przez $C_m^n$ cykliczną podgrupę rzędu $m$ w okręgu $\R/p^n \Z$. 
	
	Funkcje przejścia $\varphi_n$ mają jądra rzędu $p$ względnie pierwszego z $m$, indukują więc izomorfizmy $C_m^{n+1} \cong C_m^n$.
	Rzutową granicą tego stałego ciągu jest cykliczna podgrupa $C_m \subseteq \mathbb S_p$.
	Aby udowodnić jedyność, rozpatrzmy dowolny homomorfizm $\sigma \colon \Z/m\Z \to \mathbb S_p$.
	Złożenie $\psi_n \circ\sigma$ ma obraz w jedynej cyklicznej podgrupie $C_m^n$ okręgu $\R/p^n\Z$.
	Zatem $\sigma$ ma obraz w $C_m$, a to kończy dowód.
\end{proof}

Można ją zrzutować na okrąg: $\psi(C_m) = m^{-1} \Z/\Z$.
Skoro $C_m \times \Z_p$ jest izomorficzny z $\psi^{-1} (m^{-1}\Z/\Z)$, $C_m$ jest maksymalną skończoną podgrupą przeciwobrazu ($\psi$).

\begin{fakt}
	Solenoid $p$-adyczny $\mathbb S_p$ nie ma $p$-torsji.
\end{fakt}

\begin{proof}
	Ustalmy morfizm $\sigma \colon \Z / p\Z \to \mathbb S_p$.
	Wtedy poniższe złożenia są trywialne.
	\[
		\varphi_n \circ \psi_{n+1} \circ \sigma \colon \Z/p\Z \to \mathbb S_p\to \R/p^{n+1}\Z \to \R/p^n\Z
	\]
	Obraz złożenia $\psi_{n+1} \circ \sigma$ leży w jedynej cyklicznej podgrupie okręgu $\R/p^{n+1} \Z$ rzędu $p$.
	Podgrupa ta jest jądrem morfizmu $\varphi_n$ oraz $\psi_n \circ \sigma = \varphi_n(\psi_{n+1} \circ \sigma)$.
	Zatem nie istnieje element rzędu $p^k$, $k \ge 1$, w $\mathbb S_p$.
\end{proof}

\begin{fakt}
	Solenoid \prawo{Rbrt\\1.App.3} $p$-adyczny zawiera podgrupę izomorficzną z $\R$ (gęstą), podobnie dla $\Q_p$.
\end{fakt}

\begin{proof}
	Projekcje $f_n \colon \R \to \R/p^n\Z$ są zgodne z funkcjami przejścia układu rzutowego, definiującym solenoid ($f_n = \varphi_n \circ f_{n+1}$).
	Istnieje więc jedyna faktoryzacja $f \colon \R \to \mathbb S_p$.
	Jeśli $x \neq 0 \in \R$ to $p^n > x$, pociąga $f_n(x) \neq 0 \in \R/p^n\Z$ oraz $f(x) \neq 0 \in \mathbb S_p$.
	To pokazuje, że homomorfizm $f$ jest injekcją (poza tym, $\bigcap_n \ker f_n = \bigcap_n p^n\Z = \{0\}$).

	Obraz $f$ jest gęsty jak obrazy surjekcji $f_n$.
	Rozpatrzmy podgrupy $\psi^{-1}(p^{-k} \Z/\Z) \le \mathbb S_p$, $H_k$.
	Podgrupa $H_0 = \Z_p$ ma indeks $p^k$ w $H_k = \lim_n p^{-k}\Z / p^n\Z \cong p^{-k}\Z_p$.
	Zatem
	\[
		\Q_p \cong \psi^{-1}(\Z[1/p] / \Z) = \bigcup_k \psi^{-1} (p^{-k}\Z/\Z) = \bigcup_k H_k \subsetneq S_p,
	\]
	Gęstość tej podgrupy wynika z gęstości $\psi_n (\Q_p) = \Z[1/p] / p^n \Z \subset \R/p^n\Z$.
\end{proof}

Dzięki temu mamy prosty wniosek:

\begin{fakt}
	Solenoid jest continuum (zwarty i spójny).
\end{fakt}

\begin{proof}
	Wiemy z topologii, że gdy $A \subseteq X$ jest spójny, to każdy $B \subseteq X$, że $A \subseteq B \subseteq \operatorname{cl} A$, też.
	Weźmy za $A$ podprzestrzeń (spójną) $f(\R) \subseteq \mathbb S_p$, która jest gęsta w solenoidzie.
\end{proof}

Na \prawo{Rbrt\\1.App.4} zakończenie przedstawimy solenoid jako bardzo skręconą linę.
Ciąg morfizmów ciągłych $f_n \colon \R \times \Q_p \to \R / p^n \Z$, $f_n (t, x) = t + \sum_{i < n} a_i p^i \pmod{p^n\Z }$
nie psuje układu rzutowego, więc faktoryzujemy go do $f (t, x) = t + x \colon \R \times \Q_p \to \mathbb S_p$.

\begin{lemat}
	Jądrem $f$ jest dyskretna podgrupa $\Gamma = \{(a, -a) : a \in \Z[1/p]\} \le \R \times \Q_p$.
\end{lemat}

\begin{proof}
	Jeśli $f(t, x) = 0$, to $f_n(t, x) = (\psi_n \circ f)(t, x) = 0$, zatem $t + \sum_{i < n} a_o p^i \in p^n\Z$ dla $n \ge 1$ i $t = -x$.
	Zawieranie $\Gamma \subseteq \ker f$ jest oczywiste.

	Dla pokazania dyskretności wystarczy wskazać otoczenie zera w produkcie $\R \times \Q_p$, które kroiłoby trywialnie $\Gamma$.
	Jeśli para $(-a, a)$ leży w otwartym zbiorze $(-1, 1) \times \Z_p$ i $\Gamma$, to $a \in \Z[1/p]$ jest postaci $\sum_{i \ge 0} a_i p^i$.
	Ale $\Z[1/p] \cap \Z_p$ to $\Z$, więc $a \in \Z \cap (-1, 1) = \{0\}$ i dowód jest zakończony.
\end{proof}

\begin{fakt}
	$f' \colon \R \times \Q_p / \Gamma \to \mathbb S_p$ jest izomorfizmem (algebraicznie i topologicznie)/.
\end{fakt}

\begin{proof}
	Funkcje $f_n$ są ,,na'', zaś ich granica $f$ ma gęsty obraz.
	Dodatkowo $f(t, x) = f(s, y)$, gdzie $s = t + \langle x \rangle \in \R$ i $y = [x] \in \Z_p$.
	Pójdźmy o krok dalej, $f(s, y) = f(s - [s], y + [s])$.
	Wiemy, że $s - [s] \in [0, 1)$, zatem obraz $f$, $f[[0,1] \times \Z_p]$, jest zwarty, domknięty.

	Funkcja $f$ jest surjekcją, $f'$ zaś bijekcją (ciągłą).
	Jest izomorfizmem na mocy zwartości obrazu i dziedziny.
\end{proof}

Solenoid to też iloraz topologicznej przestrzeni $[0, 1] \times \Z_p$ przez $(1, x) \simeq (0, x+1)$.
Wyobrażamy sobie walec $[0,1] \times \Z_p$, o skręcających przy zszywaniu końcach.

\begin{fakt}
	Domknięte \prawo{Rbrt\\1.App.5} podgrupy $\mathbb S_p$ to $C_m$, $C_m \times p^k\Z_p$ i $\mathbb S_p$, gdzie $p \nmid m$ i $k \in \Z$.
\end{fakt}

\begin{fakt}
	Solenoid jest nierozkładalnym continuum.
\end{fakt}
