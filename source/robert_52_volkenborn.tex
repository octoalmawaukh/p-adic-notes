\section{Całka Volkenborna}
Zdefiniujemy \prawo{Rbrt\\5.5.1} całkę z funkcji $f \colon \Z_p \to \cialo$, gdzie $\cialo$ to zupełne rozszerzenie $\Q_p$.
Niestety, fakt \ref{glorreichetraume} pokazuje, że nie można zdefiniować formy liniowej $\varphi$ na przestrzeni $\mathcal C(\Z_p)$, ciekawej i  niezmienniczej na przesunięcia. % podlinkuj: IV.3.5.3
%Przypomnijmy: dla formy $\varphi \colon \mathcal C(\Z_p) \to K$ spełniającej $\varphi(f(x+1)) = \varphi(f(x))$  mamy $\varphi = 0$.

Można tego dokonać dla $\mathcal F^{lc}(\Z_p)$.
Przyjmijmy, że $\varphi(1) = 1$.
Odporność na przesuwanie wymusza tę samą wartość $p^{-n}$ dla indykatorów warstw $p^n \Z_p$.
Supremum tych funkcji wynosi $1$, ale $|\varphi(f)|$ jest dowolnie duże, gdyż równe $p^n$, co odbiera formie $\varphi$ ciągłość.

Zdefiniujmy ,,objętość'' kuli $\kula[j, |p^n|]$ w $\Z_p$ jako $p^{-n} \in \Q_p$.
Przedstawiona za moment konstrukcja nie jest niezmiennicza na przesunięcia i wymaga istnienia co najmniej $(\suma f)'(0)$, choć nie ma nic złego w ograniczaniu się do $f \in \mathcal S^1(\Z_p)$.
Zacznijmy od wyrażenia
\[
	S(f, n) := \frac{1}{p^n}\sum_{j=0}^{p^n-1} f(j) = \sum_{j=0}^{p^n-1} f(j) m(j+p^n\Z_p).
\]
Reprezentuje ono sumę Riemanna dla $f$.

Nieoznaczona suma $F$ dla funkcji $f$ była zdefiniowana tak, by $\nabla F = f$ ($F(0) = 0$).
Zachodzi wtedy
\[
	S(f, n) = \frac{F(p^n) - F(0)}{p^n}.
\]
Widać więc, że jeśli $F$ różniczkuje się w początku, to granica z $n \to \infty$ istnieje.
Kiedy $f$ ma współczynniki Mahlera $c_n$, współczynniki $\suma f$ są ,,przesunięte'', zaś różniczkowalność w początku ma miejsce dokładnie gdy $|c_{n-1}/n| \to 0$ (ale tak jest dla $f \in S^1(\Z_p)$). % twierdzenie w Robercie: chyba5.1.5.1

\begin{definicja}
{Całką Volkenborna} dla funkcji $f \in S^1(\Z_p)$ jest:
\[
	\int_{\Z_p} f(x) \,\textrm{d}x = \lim_{n\to \infty} \frac{1}{p^n} \sum_{j=0}^{p^n-1} f(j) = (\suma f)'(0)
\]
\end{definicja}

\begin{przyklad}
	Jeśli $b_k$ są liczbami Bernoulliego, to % (ich wykładniczą funkcją tworzącą jest $t / (\exp t - 1)$), to
	\[
		\intzp x^k \,\textrm{d}x = b_k.
	\]
\end{przyklad}

%Całka ze stałej $f = c$ to $c$.
%A oto podstawowa własność całki:

\begin{fakt}
	Jeśli $\|f_n - f\|_* $ dąży do $0$, to $\int_{\Z_p} f_n$ zbiega do $\int_{\Z_p} f$.
	Dla $f \in S^1(\Z_p)$ prawdziwe jest poniższe oszacowanie.
	\[
		I = \left | \int_{\Z_p} f(x)\,\textrm{d}x \right| \le p \| f\|_1.
	\]
\end{fakt}

\begin{proof}
	$I = |(\suma f)'(0)| \le \|\suma f\|_* \le p \|f\|_*$ (patrz \ref{heartsoul}).
\end{proof}

Przypomnijmy, że $\nabla f$ to dyskretny gradient funkcji $f$, to jest: $\nabla f(x) = f(x+1) - f(x)$.

\begin{fakt} \label{festniemalsauf}
	Dla $f \in S^1(\Z_p)$ mamy $I := \int_{\Z_p} \nabla f(x) \,\textrm{d}x = f'(0)$.
\end{fakt}

\begin{proof}
	$I = (\suma \nabla f)'(0) = (f - f(0))'(0) = f'(0)$.
\end{proof}

%Nieoznaczoną sumą funkcji dwumianowej $C^\cdot_n$ jest następna, $C_{n+1}^\cdot$.
%Ta obserwacja umożliwia policzenie całki Volkenborna funkcji $f \in S^1$ o znanym rozwinięciu w szereg Mahlera.
Znamy nieoznaczone sumy dla funkcji dwumianowych, a zatem scałkowanie szeregu Mahlera nie sprawi trudności.

\begin{fakt}
	Niech \prawo{Rbrt\\5.5.2} $\sum_{k\ge0} c_k {x \choose k}$ będzie szeregiem Mahlera dla $f$, funkcji klasy $S^1$.
	Wtedy
	\[
		\int_{\Z_p} f(x) \,\textrm{d}x = \sum_{k = 0}^\infty \frac{(-1)^kc_k}{k+1}.
	\]
\end{fakt}

\begin{proof}
	Skoro $f_n(x) = \sum_{k\le n} c_k {x \choose k}$ dążą do $f(x)$ w $\|\cdot\|_*$, to wolno nam całkować wyraz po wyrazie, by otrzymać kolejno: $\sum_{k \ge 0} c_k \int (x \mbox{ nad } k) \,\textrm{d}x$, $\sum_{k \ge 0} c_k (x \mbox{ nad } k+1)'(0)$, a potem $\sum_{k \ge 0} c_k \lim_{x \to 0} (x-1 \mbox{ nad }k) / (k+1)$ i prawą stronę.
\end{proof}

\begin{przyklad}
	Ustalmy $t \in \C_p$, że $0 < |t| < 1$.
	Wtedy
	\[
		\int_{\Z_p} (1+t)^x \,\textrm{d}x =
		\int_{\Z_p} \sum_{k=0}^\infty t^k {x \choose k} \,\textrm{d}x = 
		\sum_{k=0}^\infty \frac{(-t)^k}{k+1},% = \frac{\log (1+t)}{t}.
	\]
	a to jest po prostu $\frac 1 t \log (1+t)$.
\end{przyklad}

% 5.3 - Robert
Przyda \prawo{Rbrt\\5.5.3} nam się więcej wzorów z tą całką.
Przypomnijmy, że $\mathfrak D$ komutuje z translacją, zatem także z $\nabla = \tau - \operatorname{id}$.

\begin{fakt}
	Niechaj $P_0 \colon f \mapsto f(0) \cdot 1$ rzutuje $S^1(\Z_p)$ na stałe.
	Wtedy $\mathfrak D\suma $ komutuje z translacjami $\tau_x$, a dodatkowo $\suma  \tau = \tau \suma  - P_0$, $\suma \mathfrak D = \mathfrak D\suma  - P_0\mathfrak D\suma $.
\end{fakt}

\begin{proof}
	Z definicji, dla $n \ge 1$ mamy
	\begin{align*}
		\dots & = \suma (\tau f)(n) = \sum_{j=0}^{n-1} \tau f(j) = \sum_{j=0}^{n-1} f(j+1) \\
		& = \sum_{j = 1}^{n} f(j) = \suma f(n+1) - f(0) = \tau \suma f(n) - f(0),
	\end{align*}
	co dowodzi pierwszego stwierdzenia (całkowite $n \ge 1$ w $\Z_p$ leżą gęsto, ciągłość funkcji).
	Z drugiej strony, różniczkowanie $\suma \tau f=  \tau \suma f - f(0)$ uzasadnia $\mathfrak D\suma \tau f = \mathfrak D \tau \suma f = \tau \mathfrak D \suma f$.
	Do tego, ponieważ $\nabla \suma f = f$, ale $S \nabla f = f - f(0)$, to $\nabla \suma = \textrm{id}$ oraz $\suma \nabla = \textrm{id} - P_0$.
	Wnioskujemy, że $\suma \mathfrak D$ to $\suma \mathfrak D \nabla \suma$, czyli $\suma \nabla \mathfrak D \suma = \mathfrak D \suma - P_0 \mathfrak D\suma$.
\end{proof}

\begin{fakt} % strona 267
	Niech $f \in \mathcal S^1(\Z_p)$. Wtedy
	\begin{align*}
		(\suma f)'(x) & = \intzp \tau_x f(t)\,\textrm{d}t \\
		\suma(f')(x)  & = \intzp f(x+t) - f(t) \,\textrm{d}t
	\end{align*}
\end{fakt}

\begin{proof}
	Całka z $f$ to pochodna $(\suma f)'(0)$, więc
	\begin{align*}
		\ldots & = \intzp f(t+1) \,\textrm{d}t = \intzp \tau f(t) \,\textrm{d}t = (\suma \tau f)'(0) \\
		& = \mathfrak D \suma \tau f(0) = \tau \mathfrak D \suma f(0) = \mathfrak D \suma f(1) = (Sf)'(1).
	\end{align*}
	Ogólny wzór dla $x = n$ otrzymujemy przez iterację, natomiast dla dowolnego $x \in \Z_p$ z ciągłości i gęstości.

	Skorzystamy teraz z \ref{festniemalsauf}: $\intzp \findif f(x + t) \,\textrm{d}t = f'(x)$.
	Stąd wynika, że $\suma (f')(n)$ jest sumą teleskopową:
	\[
		\suma(f')(n) = \intzp f(t+n) - f(t) \,\textrm{d}t
	\]
	Ponownie odwołujemy się do ciągłości i gęstości, by powyższy wzór był prawdziwy nie tylko dla $n \in \mathbb N$, ale też $n \in \Z_p$.
	%Teraz kładziemy $f' = g$ i $f = G \in \mathcal S^1(\Z_p)$ (dowolna).
\end{proof}

%Oczywistym jest, że dwie różne pierwotne funkcji $g$ mogą różnić się o dowolną funkcję $h$, że $h' = 0$.

\begin{fakt}
	Jeżeli $f \in \mathcal S^2(\Z_p)$, to $F \in \mathcal S^1(\Z_p)$ i 
	\[
		F'(x) := \left[\intzp f(x+t) \,\textrm{d}t\right]' = \intzp f'(x+t)\,\textrm{d}t.
	\]
\end{fakt}

\begin{proof}
	Skoro $f \in \mathcal S^2$, to $f' \in \mathcal S^1$, więc prawa strona równania z faktu definiuje funkcję $G(x)$ klasy $\mathcal S^1$ równą $(\suma f')'(x)$, czyli $(\mathfrak D \suma \mathfrak D f)(x)$, więc $G = \mathfrak D \suma \mathfrak D f$.
	Zauważmy przy tym, że $\suma \mathfrak D$ to $\mathfrak D \suma - P_0 \mathfrak D \suma$, zatem	$G = \mathfrak D \mathfrak D \suma f = (\suma f)'' = F'$. %, bo $F = (\suma f)'$.
\end{proof}

\begin{fakt}
	Jeśli $\sigma$ jest inwolucją $x \mapsto -1-x$ dla $\Z_p$, to 
	\[
		\intzp (f \circ \sigma) \,\textrm{d}x = \intzp f \,\textrm{d}x.
	\]
\end{fakt}

Dzięki temu możemy podać wzór na całkę Volkenborna z dowolnej funkcji nieparzystej: to po prostu $-{f'(0)}/ 2$.