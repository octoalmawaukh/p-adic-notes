\section{Pierścień $\Z_p$}
\begin{fakt}
	Pierścień $\Z_p$ nie ma dzielników zera.
\end{fakt}

\begin{proof}
	Ustalmy liczby $a, b \in \Z_p$ różne od zera.
	Wtedy $p$ nie dzieli iloczynu $a_vb_w$, gdzie $v = v_p(a)$, $w = v_p(b)$, ale to jest właśnie pierwszy niezerowy współczynnik w rozwinięciu $ab$, zatem $ab \neq 0$.
\end{proof}

Niech $\mathbb F_p = \Z / p\Z$. Odwzorowanie $\sum_i a_ip^i \mapsto a_0 \pmod p$ jest homomorfizmem pierścieni, \kolorowo{redukcją modulo $p$}.
Iloraz jest ciałem, więc jądro $p\Z_p$ jest ideałem maksymalnym w $\Z_p$.

\begin{fakt}
	Grupa $\Z_p^\times$ składa się z $p$-adycznych liczb całkowitych rzędu zero ($a_0 \neq 0$).
\end{fakt}

\begin{proof}
	Jeśli $p$-adyczna l. całkowita odwraca się, to jej redukcja w $\mathbb F_p$ również.
	Ustalmy więc $x \in \Z_p$ rzędu zero.

	Skoro redukcja $x$ w ciele $\mathbb F_p$ nie jest zerem, to odwraca się i możemy wskazać $0 < y_0 < p$, że $x_0 y_0 =  1+  kp$ dla pewnej liczby $k$.
	Przyjmijmy, że $x = x_0 + p \alpha$, wtedy $xy_0 = 1 + p \beta$ (gdzie $\beta \in \Z_p$), a taką liczbę łatwo odwrócić, co jakoś kończy rozumowanie.
	\[
		(1+p\beta)^{-1} = 1 - \beta p + \beta^2p^2 - \ldots \qedhere
	\]
\end{proof}

Pierścień $\Z_p$ ma jeden ideał maksymalny, $p \Z_p = \Z_p \setminus \Z_p^\times$.
Dokładniej: $\Z_p \setminus \{0\}$ to $\coprod_{k \ge 0} p^k \Z_p^\times$.
Ideały główne pierścienia $\Z_p$, $p^k \Z_p$, kroją się do $\{0\}$.

\begin{fakt}
	Pierścień $\Z_p$ jest dziedziną ideałów głównych (nie ma innych ideałów niż wyżej wymienione).
\end{fakt}

\begin{proof}
	Ustalmy  niezerowy ideał $I \le \Z_p$ z elementem $x \in I$ o minimalnym rzędzie, $k$.
	Wtedy $x = p^k u$, gdzie $u$ to jedność i $(p^k) \subseteq I$, gdyż $p^k = u^{-1} x$.
	Niechaj $y \in I$ będzie rzędu $l \ge k$.
	Wtedy druga inkluza wynika z $y = p^l u' = p^k p^{l-k} u'$.
\end{proof}