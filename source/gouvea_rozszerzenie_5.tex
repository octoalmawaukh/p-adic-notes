\section{Analiza}
Wiele \prawo{Gouvea\\5.5} z dotychczasowych osiągnięć przenosi się bez problemów na ogólny przypadek, bo nie korzystaliśmy z magicznych własności $\Q_p$.
Jedyne zmiany, o których trzeba pamiętać, mają związek z rozgałęzieniem: być może trzeba będzie użyć jednolitości $\pi$ zamiast $p$. Oto lista:

\begin{fakt}
	[\ref{reginald}, \ref{ingentis}, \ref{caedis}, \ref{decoris}, \ref{auctoris}, \ref{causeahriot}, \ref{stummeschreie}] Prawdziwy dla każdego ciała $\cialo$, a nie tylko $\Q_p$.
\end{fakt}

\begin{fakt}
	Twierdzenie Strassmana działa dla $\Q_p$, $\Z_p$ zastąpionego przez $\cialo$, $\mathcal O$.
	Wnioski z niego zachowują sens.
\end{fakt}

\begin{fakt}
	Logarytm $p$-adyczny jest homomorfizmem $1 + \pi \mathcal O \to \cialo$.
	Eksponens to homomorfizm $\kula(0, p^{1/(1-p)}) \to \cialo$.
	Obie funkcje są do siebie wzajemnie odwrotne.
\end{fakt}

\begin{fakt}
	Dla każdego $\alpha \in \Z_p$ i $|x| < 1$ szereg $(1+x)^\alpha$ zbiega.
\end{fakt}