\section{Granice rzutowe}
Homomorfizm $\varepsilon_n \colon \Z_p \to \Z/p^n\Z$ uogólnia redukcję.
Przeniesiemy zbieżność $\sum_{i < n} a_ip^i$ do liczby $p$-adycznej $\sum_{i \ge 0} a_i p^i$ na pierścienie $\Z/p^n\Z$ i $\Z_p$.
Nic trudnego.

Umożliwi nam to kanoniczny homomorfizm $\varphi_n \colon \Z/p^{n+1} \Z \to \Z / p^n \Z$ z przemiennym diagramem ,,$\varepsilon_n = \varphi_n \circ \varepsilon_{n+1}$''.

\begin{definicja}
	Układ rzutowy to ciąg funkcji $\varphi_n \colon E_{n+1} \to E_n$ (oraz samych zbiorów $E_n$). 
	Jego granicą jest zbiór $E$ z funkcjami $\psi_n \colon E \to E_n$, że $\psi_n = \varphi_n \circ \psi_{n+1}$, o ile jest uniwersalny: dla drugiej ,,granicy'' $E'$ z funkcjami $f_n$ mamy funkcję $f$, by $f_n = \psi_n \circ f \colon E' \to E \to E_n$ (,,faktoryzacja'').
	$E_0 \leftarrow E_1 \leftarrow \dots \leftarrow E_n \leftarrow \dots \leftarrow \varprojlim E_n = E$
\end{definicja}

Możemy iterować $f_n = \psi_n \circ f_{n+1}$, by uzyskać $f_n = \psi_n \circ f$:
\begin{align*}
	f_n & = \varphi_n \circ f_{n+1} = \varphi_n \circ \varphi_{n+1} \circ f_{n+2} \\
	& = (\varphi_n \circ \varphi_{n+1} \circ \ldots \circ \varphi_{n+k}) \circ f_{n+k+1} = \psi_n \circ f,
\end{align*}
$f$ zachowuje się jak granica $f_j$, zaś $\psi_n$ to granica złożeń funkcji przejścia.
Łatwo zauważyć, że granica rzutowa nie zależy od początkowych wyrazów (w skończonej ilości), więc można je pominąć.

\begin{fakt}
	Każdy układ rzutowy $(E_n, \varphi_n)_{n \ge 0}$ ma granicę.
\end{fakt}

Jeśli funkcje przejścia są ,,na'', to rzuty $\psi_n$ też i granica jest niepusta.
Granice rzutowe istnieją dla wielu obiektów, nie tylko grup, ale także przestrzeni topologicznych.

\begin{fakt}
	Klasa niepustych, zwartych przestrzeni topologicznych jest zamknięta na branie granic odwrotnych.
\end{fakt}

\begin{fakt}
	Jeśli $A$ jest podzbiorem granicy $E$ dla $E_n$, to $\operatorname{cl} A = \bigcap_{n=0}^\infty \psi_n^{-1} (\operatorname{cl} \psi_n (A))$.
\end{fakt}

\begin{fakt}
	Jeśli $\grupa$ jest granicą rzutową grup $\grupa_n$ z homomorfizmami $\psi_n \colon \grupa \to \grupa_n$, to przekrój ich jąder jest trywialny.
	Istnieje kanoniczny izomorfizm $\grupa \cong \varprojlim (\grupa / \ker \psi_n)$.
\end{fakt}

Pierścień szeregów formalnych $\Z_p$ można opisać w języku granic rzutowych.

\begin{fakt}
	Odwzorowanie, które wiąże liczbę $p$-adyczną z ciągiem jej częściowych sum modulo ${p^n}$, $\Z_p \to \varprojlim_n \Z/p^n\Z$, jest izomorfizmem topologicznych pierścieni.
\end{fakt}

\begin{proof}
	Z definicji, $\varphi_n (\sum_{i \le n} a_ip^i \mbox{ mod } {p^{n+1}}) = \sum_{i<n} a_ip^i \mbox{ mod } {p^n}$, więc konsekwentne ciągi w $\prod \Z/p^n\Z$ są dokładnie ciągami sum częściowych szeregu $\sum_{i \ge 0} a_i p^i$ ($0 \le a_i\le p-1$), czyli liczbami $p$-adycznymi.
	Relacje $x_n = \sum_{i \le n} a_ip^i$, $a_0 = x_1$, gdzie $a_n = (x_{n+1} - x_n)p^{-n}$ pokazują nam, że faktoryzacja $\Z_p \to \lim_{\leftarrow} \Z/p^n\Z$ jest bijekcją, czyli izomorfizmem, a jako ciągła bijekcja między p. zwartymi, jednocześnie homeomorfizmem.
\end{proof}

Homomorfizmy $\Z \to \Z/p^{n+1} \Z \to \Z/p^n\Z$ są źródłem granicznego homomorfizmu $\Z \to \lim_{\leftarrow} \Z/p^n\Z$, który można utożsamić z kanonicznym włożeniem $\Z \to \Z_p$.
Funkcja $\sum_{i < n} a_ip^i \mbox{ mod } {p^n} \mapsto \sum_{i<n} a_ip^i \mbox{ mod } {p^n \Z_p}$ definiuje izomorfizm $\Z/p^n\Z \to \Z_p/p^n\Z_p$, a w szczególności $\Z_p / p\Z_p \cong \Z/p\Z = \mathbb F_p$.
Ogólniej $\Z_p/p^n \Z_p \cong \Z/p^n\Z$.
Obcięcie redukcji $\Z_p \to \Z/p^n\Z$ do $\Z_{(p)} \subseteq \Q$ jest (z drugiej strony) ,,na'' i ma jądro $p^n\Z_{(p)}$.
Zyskaliśmy tak

\begin{fakt}
	Uzupełnienie $\Z_{(p)}$ to $\Z_p$, rzutowa granica $\varprojlim \Z_{(p)} / p^n\Z_{(p)}$.
\end{fakt}

Granica rzutowa to konstrukcja z teorii kategorii, z powodu jej dużej ogólności nie będzie nam więcej potrzebna.
Na zakończenie prostsza, choć bezużyteczna definicja.

\begin{fakt}
	Homomorfizm pierścieni $\Z[[x]] \to \Z_p$, ewaluacja w punkcie $x = p$, zadaje kanoniczny izomorfizm $\Z[[x]] / (x-p) \cong \Z_p$.
\end{fakt}
