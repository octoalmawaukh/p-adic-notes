\section{Lemat o podnoszeniu wykładnika}
Przytoczymy teraz mało znany, użyteczny lemat z pracy~\cite{parvardi11}.
\begin{lemat}
	Załóżmy, że liczby $x,y$ (całkowite), $n$ (naturalna) i $p$ (pierwsza) zostały dobrane tak, by $p \nmid nxy$ oraz $p \mid x \pm y$.
	Wtedy $v_p(x^n\pm y^n) = v_p(x \pm y)$.
\end{lemat}

Podamy teraz pierwsze dwie formy lematu, w zależności od znaku w ,,$p \mid x \pm y$''.

\begin{fakt}
	Załóżmy, że liczby $x,y$ (całkowite), $n$ (naturalna) i $p > 2$ (pierwsza) zostały dobrane tak, by $p \mid x \pm y$ i $p \nmid xy$.
	Wtedy $v_p(x^n \pm y^n) = v_p(x \pm y) + v_p(n)$.
\end{fakt}

Przypadek $p=2$ jak zwykle wymaga więcej uwagi.

\begin{fakt}
	Niech nieparzyste liczby całkowite $x,y$ dają tę samą resztę z dzielenia przez $4$.
	Wtedy \[v_2(x^n-y^n) = v_2(x-y) + v_2(n).\]
\end{fakt}

\begin{fakt}
	Niech dane będą nieparzyste liczby całkowite $x,y$ oraz parzysta $n > 0$.
	Wtedy 
	\[
		v_2(x^n-y^n) = v_2(x-y) + v_2(x+y) + v_2(n) - 1.
	\]
\end{fakt}
