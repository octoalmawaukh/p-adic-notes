Liczne podobieństwa między ciałami $\R$ oraz $\Q_p$ (lokalna zwartość, zupełność, algebraiczna niedomkniętość) sugerują, że tam, gdzie normalni ludzie korzystają z prostej rzeczywistej, my możemy wcisnąć $p$-adyczne obiekty.
Należy pamiętać o tym, że chociaż ciało $\R$ jest uporządkowane i spójne, to $\Q_p$ stanowią całkowite jego przeciwieństwo -- w połowie z nich istnieje jednostka urojona, wszystkie zaś są całkowicie niespójne.

Zaczniemy od najprostszych struktur znanych z analizy matematycznej: ciągów oraz funkcji, cały czas zwracając uwagę na to, ile twierdzeń jesteśmy w stanie uratować, a co jest skazane na zapomnienie.
Absencja pochodnych jest początkowo szokująca.
Zrozumienie, że naiwne twierdzenie o wartości średniej jest fałszywe, pozwoli przekonać się do szeregów potęgowych.
Zupełnie inną parę kaloszy stanowią całki.
Można wprawdzie się zająć nimi z podstawową wiedzą na temat liczb $p$-adycznych, jednak wstrzymamy się z tym.
Będziemy ich potrzebować w trzech miejscach: w rozdziale poświęconym mechanice kwantowej, podczas interpolacji funkcji $\zeta$ Riemanna oraz po poznaniu trudniejszej analizy.
Wygląda na to, że nie istnieje jednolita teoria obejmująca wszystkie trzy twory.

\section{Ciągi oraz szeregi}
Nawet, \prawo{Gouvea\\4.1} kiedy jasno tego nie zaznaczamy, pracujemy w $\Q_p$, właśnie tym ciele, gdzie marzenia stają się prawdziwe.

\begin{fakt} \label{reginald}
	Ciąg $(x_n)$ jest ciągiem Cauchy'ego, wtedy i tylko wtedy gdy jego pierwsza skończona różnica $x_{n+1} - x_n$ zbiega do zera.
\end{fakt}

\begin{proof}
	Jeśli $m = n+r > n$, to $|x_m - x_n| \le \max_{1 \le k \le r} |x_{n+k}-x_{n+k-1}|$, jako że norma jest niearchimedesowa.
\end{proof}

Zbieżność absolutna szeregu pociąga jego zbieżność, w ciele liczb $p$-adycznych zachodzi jednak jeszcze mocniejszy fakt.

Warto w tym miejscu nadmienić, iż wiele dowodów z analizy rzeczywistej przepisuje się bez zmieniania ani jednego znaku drukarskiego, toteż będziemy je (najczęściej) pomijać.

\begin{fakt} \label{ingentis} %zielony
	Zbieżność szeregu $\sum_n x_n$ o wyrazie ogólnym z $\Q_p$ jest równoważna zbieżności $x_n$ do $0$.
	Prawdziwe jest wtedy oszacowanie $|\sum_{n \ge 0} x_n| \le \max_n |x_n|$.
\end{fakt}

\begin{proof}
	Implikacja w prawo jest oczywista.
	
	Dla dowodu w lewo wynikania wystarczy zauważyć, że wyraz $x_n$ to różnica między dwoma sumami częściowymi i powołać się na poprzedni fakt.

	Nierówność wynika z
	\[
		\left|\sum_{n = 0}^{N-1} x_n + \sum_{n = N}^\infty x_n\right| \le \max_{n < N} |x_n| + \left|\sum_{n = N}^\infty x_n\right|,
	\] gdzie drugi składnik znika w nieskończoności.
\end{proof}

\begin{wniosek}
	Szereg z poprzedniego faktu zbiega bezwarunkowo, ale niekoniecznie bezwzględnie.
\end{wniosek}

\begin{proof}
	Nałożenie permutacji na wyrazy szeregu nie psuje ich zbieżności do zera.

	Nie każdy szereg zbiega jednak bezwzględnie, wystarczy dodać do siebie $p^k$ sztuk liczby $p^k$ dla $k \ge 0$.
	Nałożenie normy zmusza do wysumowania $1 + 1 + 1 + \ldots$, ale zwykłą sumą graniczną jest odwrotność $1 - p^2$, żyjąca w każdym $\Q_p$.
\end{proof}

Wynika stąd, iż sprawdzanie zbieżności ciągu jest naprawdę przyjemnym zajęciem i różne testy (Cauchy'ego, d'Alemberta, Raabego, całkowe i inne) nie mają racji bytu.
Aby zająć się podwójnymi sumami, potrzebujemy jednak czegoś więcej niż tylko zbieżność do zera.

\begin{definicja}
	Jeśli dla każdej dodatniej liczby $\varepsilon$ istnieje całkowita $N$ niezależna od $k$, że $i \ge N$ pociąga $|x_{ik}| < \varepsilon$, to $\lim_{i \to \infty} x_{ik} = 0$ jednostajnie względem $k$.
\end{definicja}

\begin{lemat} \label{veteris}
	Załóżmy, że $\lim_{k \to \infty} x_{ik} = 0$ (dla każdego $i$) oraz $\lim_{i \to \infty} x_{ik} = 0$ jednostajnie względem $k$.
	Wtedy każdemu $\varepsilon > 0$ odpowiada $N$, że $\max \{i, k\} \ge N$ pociąga $|x_{ik}| < \varepsilon$.
\end{lemat}

\begin{proof}
	Ustalmy $\varepsilon$.
	Drugi warunek zapewnia $N_0$ (zależne od $\varepsilon$), że $|x_{ik}| < \varepsilon$ dla $i \ge N_0$.
	
	Pierwszy zaś dla każdego $i$ daje $N_1$, dla którego $k \ge N_1$ pociąga $|x_{ik}| < \varepsilon$.
	Wystarczy przyjąć $N = \max\{N_0, N_1(0), N_1(1), \ldots, N_1(N_0-1)\}$.
\end{proof}

\begin{fakt} \label{caedis}
	Przy założeniach z lematu \ref{veteris} poniższe szeregi zbiegają do tej samej liczby: \[\sum_{i = 0}^\infty \sum_{k = 0}^\infty x_{ik} = \sum_{k = 0}^\infty \sum_{i = 0}^\infty x_{ik}.\]
\end{fakt}

\begin{proof}
	Lemat mówi, że każdemu $\varepsilon > 0$ odpowiada liczba $N$, dla której ,,$\max \{i, k\} \ge N$ pociąga $|x_{ik}| < \varepsilon$''.
	Skoro ciąg $x_{ik}$ zbiega do zera po ustaleniu jednego z indeksów, to oba szeregi wewnętrzne są zbieżne.
	Dla $i \ge N$ mamy $|\sum_{k \ge 0} x_{ik}| \le \max_k |x_{ik}| < \varepsilon$ na mocy faktu \ref{ingentis}, podobna nierówność prawdziwa jest dla $k \ge N$.
	
	Wnioskujemy stąd, że podwójne szeregi także zbiegają, bo
	\[
		\lim_{i \to \infty} \sum_{k \ge 0} x_{ik} = \lim_{k \to \infty} \sum_{i \ge 0} x_{ik} = 0.
	\]

	Pozostało nam uzasadnić, że sumy są sobie równe.
	Pozostańmy przy $N$, $\varepsilon$ wybranych wcześniej.
	Oznacza to, że $|x_{ik}| < \varepsilon$, gdy $i \ge N$ lub $k \ge N$.
	Zauważmy, że 
	\[
		\Bigl|\sum_{i, k \ge 0} x_{ik} - \sum_{i, k \le N} x_{ik} \Bigr| = 
		\Bigl|\sum_{i \le N} \sum_{k > N} x_{ik} + \sum_{i > N} \sum_{k \ge 0} x_{ik} \Bigr|.
	\]
	Jeśli więc $k \ge N+1$, to $|x_{ik}| < \varepsilon$ dla każdego $i$, zatem pierwszy składnik pod wartością bezwzględną można (ultrametrycznie) oszacować z góry przez $\varepsilon$; podobnie szacuje się drugi składnik. Oczywiście zamiana $i, k$ miejscami nic nie psuje, więc możemy je przestawić i wywnioskować stąd równość sum.
\end{proof}

\begin{fakt} \label{ludmori}
	Niech szeregi $\sum_i x_i$, $\sum_i y_i$ będą zbieżne.
	Wtedy
	\[
		\sum_{i = 0}^\infty x_i + y_i = \sum_{i = 0}^\infty x_i + \sum_{i = 0}^\infty y_i \,\bullet\,
		\sum_{i = 0}^\infty \sum_{k = 0}^i x_k y_{i-k} = \Bigr[\sum_{i = 0}^\infty x_i\Bigr] \cdot \Bigr[ \sum_{i = 0}^\infty y_i\Bigr].
	\]
\end{fakt}

Wyznaczymy teraz wartość konkretnych szeregów $p$-adycznych.
Fenomen związany z ich nieoczekiwanymi granicami wyjaśnić się może  po lekturze ostatniego ustępu w tym rozdziale, gdzie przytoczymy zaskakujący wynik Burgera i Struppecka.

\begin{fakt}
	Jeżeli \prawo{Koblitz\\Ex. 4.1.12} $k > 0$, to $\sum_{n \ge 0} n^k p^n$ jest wymierne w $\Q_p$.
\end{fakt}

\begin{proof}
	Wynika to z równości szeregów formalnych
	\[
		\sum_{n = 0}^\infty n^k x^n = \Bigl(x \cdot \frac {\textrm{d}}{\textrm{d}x} \Bigr)^k \frac{1}{1-x}.
	\]
	Szereg stojący po lewej stronie to specjalny przypadek funkcji $\zeta$ Hurwitza-Lercha, ale nam wystarczy wiedza o wielomianach Eulera. Okazuje się (skoro $|p| = 1 / p < 1$), że 
	\[
		\sum_{n = 0}^\infty n^k p^n = \sum_{n = 1}^k \left\{\begin{matrix} k \\ n \end{matrix}\right\} \cdot  \frac{p \cdot n!}{(p-1)^{n+1}},
	\]
	gdzie $\{\cdot, \cdot\}$ to (nieznakowana) druga licza Stirlinga.
\end{proof}

Na mocy równości $n \cdot n! = (n+1)! - n!$ suma \[\sum_{n \ge 0} n \cdot n!\]
jest teleskopowa w każdym z ciał $\Q_p$, łatwo tak pokazać, że jej wartość wynosi $-1$.
Nieco więcej wysiłku wymaga powtórzenie osiągnięć van Hamme'a, któremu Schikhof przypisuje jeszcze trzy równości.

\begin{fakt}
	Poniższe \prawo{Schikhof\\Ex. 23.J} szeregi zbiegają w każdym z ciał $\Q_p$, aczkolwiek ostatni wymaga, by $p$ było nieparzyste: $a_n = n^2(n+1)!$, $b_n = n^5(n+1)!$, $c_n = n^2 (n+1)! 4^{-n-1}$.
	 \[
	 	\sum_{n=0}^\infty a_n = 2 \,\bullet\,
	 	\sum_{n=0}^\infty b_n = 26 \,\bullet\,
	 	\sum_{n=0}^\infty c_n = -1
	 \]
\end{fakt}

\begin{proof}
	Ostatnia równość jest fałszywa (w książce Schikhofa), musiała więc zostać delikatnie poprawiona.
	Dla $p = 2$ szereg po lewej stronie nie jest nawet zbieżny.
	Podamy jedynie sumy częściowe, które uważny Czytelnik może zweryfikować:
	\begin{align*}
		a_1 + a_2 + \ldots + a_m &= (m+2)! (m-1) + 2\\
		b_1 + b_2 + \ldots + b_m &= (m+2)! (m^4 - m^3 - 3m^2 +12m-13) + 26\\
		c_1 + c_2 + \ldots + c_m &= (m + 2)! (m + 2) : 4^{m+1} - 1. \qedhere
	\end{align*}
\end{proof}

Zwiążemy \prawo{OEIS} teraz dwa pierwsze szeregi ze światem poza-$p$-adycznym.
Dla każdego $n$ istnieją (jedyne) liczby $a_n$, $b_n$ oraz wielomian $p_n(x)$, że (przy niefortunnej notacji!)
\[
	%\sum_{i = 1}^k (i^n - a_n) (i+1)! = (k+2)! \cdot p_n(k) + b_n.
	\sum_{i = 1}^k i^n (i+1)! = (k+2)! \cdot p_n(k) + b_n + \sum_{i = 1}^k a_n (i+1)!.
\]

Jeżeli $a_n = 0$, to lewa strona dąży do $b_n$ w $\Q_p$, ale niestety nie są znane żadne $n$ inne niż $2$ i $5$, które spełniają ten warunek.
Ciągi 074051 i 074052 w bazie danych OEIS zawierają więcej informacji.
Wykładnicza tworząca $a_n$ to $\exp(1-2x-e^{-x})$.

Schikhof stwierdza, że problem \prawo{Schikhof\\Ex. 7.D} wymierności liczby $x = \sum_n n!$ pozostaje otwarty w każdym ciele $\Q_p$.
Wiemy jednak, iż nie może być wymierna w każdym z nich: po pierwsze, nie zależałaby od $p$, po drugie, byłaby całkowita.

\begin{fakt}
	Mamy \prawo{Marty,\\Sumner} $x_k := \sum_{n \ge 1} n^k \cdot n! = v_k - u_k x$, $v_k, u_k \in \Z$.
\end{fakt}

\begin{lemat}
	$\sum_{n \ge 1} (n+k)! - n! = - \sum_{n\le k} n!$.
\end{lemat}

\begin{proof}
Rozwinięcie obu stron lematu daje $\sum_n n^2 \cdot n! = -x$ (dla $k = 2$), przypadek $k = 1$ rozważaliśmy wcześniej.
Teraz wystarczy zastosować indukcję.
\end{proof}

\begin{fakt}
	Zachodzi $u_k = \sum_{i=1}^{k+1} (-1)^{k+i} \cdot \{k+1, i\}$.
\end{fakt}

Wzór ten pozwala szybciej wyznaczać współczynniki $u_k$, wcześniej Dragovich sugerował rozwiązanie układu $k+1$ równań liniowych.

\begin{fakt}
	Jeśli $k \in 3\N + 1$, to $u_k \neq 0$, wtedy $x_k$ i $x$ są tak samo niewymierne.
\end{fakt}

Zachęcamy do sięgnięcia po długi na dziewięć stron artykuł (wspominamy go w ostatnim rozdziale) Murty'ego i Sumner, zachowaliśmy ich notację.
Tam znaleźć można kompletne dowody powyższych stwierdzeń, dalsze odnośniki oraz opis, co tu się właściwie dzieje.

Wrócimy teraz do rzeczy przyziemnych i $p$-adycznej analizy ,,numerycznej''.

\begin{fakt}
	Niech \prawo{Schikhof\\Ex. 3.K} $x \in \Z$ nie dzieli się przez $p$, zaś $x_0 \in \Z$ będzie takie, że $|1 - x_0x|_p < 1$.
	Formuła $1 - x_{n+1} x = (1 - x_n x)^2$, to znaczy $x_{n+1} = x_n (2 - x_nx)$ zadaje ciąg liczb $x_n$, które szybko zbiegają do odwrotności $x$: $v_p(x_n - 1 : x) \ge 2^n$.
\end{fakt}