\section{Antypochodna}
\begin{fakt}
	Dla każdej ciągłej funkcji $f \colon X \to \cialo$ ($X \subseteq \cialo$) i $\varepsilon > 0$ istnieje lokalnie stała $g \colon X \to \cialo$, że $|f(x) - g(x)| < \varepsilon$.
\end{fakt}

\begin{proof}
	Relacja $x \simeq y$, gdy $|f(x) - f(y)| < \varepsilon$, rozbija $X$ na otwarnięte klasy abstrakcji $U_i$.
	Wybierzmy z każdej po jednym elemencie $a_i$.
	Niech $g(x) = f(a_i)$, jeśli $x \in U_i$.
\end{proof}

\begin{wniosek}
	Jeśli $X$ nie ma izolatorów, to $g' = 0$.
\end{wniosek}

\begin{twierdzenie}[Kaplansky]
	Jeżeli zbiór $X \subseteq \cialo$ jest zwarty, zaś $f \colon X \to \cialo$ ciągła, to istnieje wielomian $P \colon \cialo \to \cialo$, że dla każdego $x \in X$, $|P(x) - f(x)| < \varepsilon$.
\end{twierdzenie}

Twierdzenie Kaplansky'ego nie jest wybredne co do ciała $\cialo$, natomiast twierdzenie Weierstraßa (o jednostajnej granicy wielomianów) staje się fałszywe (!) dla $\C$ zamiast $\R$.

\begin{proof}
	Fakt połączony ze zwartością $X$ pokazują, że można ograniczyć się do indykatorów kul.

	Bez straty ogólności, niech $0 \in X$, $f(0) = 1$.
	Wybierzmy $c_1, \ldots, c_m \in X$, by $|c_1| \le \ldots \le |c_m|$ i $X \subseteq \bigsqcup_{k \le m} \kula[c_k, \delta]$ ($c_0 = 0$).
	Wtedy $\delta <|c_1|$.
	Dla pewnego $s$ mamy $(\delta / |c_1|)^s < \varepsilon$.
	Indukcyjnie wskażemy $n_1, \ldots, n_m$, by funkcja
	\[
		P(x) = \prod_{j=1}^m \left[1 - \frac{x^s}{c_j^s}\right]^{n_j}
	\]

	miała potrzebne własności: $|P(x) - 1| < \varepsilon$ dla $x$ z $\kula[0, \delta]$ i $|(x)| < \varepsilon$ dla $x$ z kolejnych kul.

	Niech $x \in \kula[0, \delta]$.
	Wtedy $1 - (x/c_j)^s \in \kula[1, \varepsilon]$ dla każdego $j$.
	Skoro $\kula[1, \varepsilon]$ jest grupą z mnożeniem (co najmniej dla $\varepsilon < 1$), wnioskujemy, że $|P(x) - 1| < \varepsilon$ niezależnie od $n_1$, $\ldots$, $n_m$.

	Niech teraz $x \in \kula[c_i, \delta]$ dla $i \ge 1$.
	Wtedy $|x - c_i| \le \delta$ i $|x| = |c_i|$, zatem
	\[
		|1 - (x/c_j)^s| \le \begin{cases} \max(1, |x/c_j|^s) \le |c_i / c_j|^s & j < i \\
		|1 - x/c_j| \le \delta / |c_1| & j = i \\
		\max(1, |x / c_j|^s) \le 1 & j > i\end{cases}
	\]

	Chcemy, żeby $|P(x)| < \varepsilon$, czyli
	\[
		\left|\frac{c_i}{c_1}\right|^{sn_1} \cdot \ldots \cdot \left|\frac{c_i}{c_{i-1}}\right|^{sn_{i-1}} \cdot \left(\frac{\delta}{|c_1|}\right)^{n_i} < \varepsilon
	\]
	dla ustalonych $n_1, \ldots, n_{i-1}$.
	Dobre $n_i$ zawsze znajdziemy.
\end{proof}

Rozpatrujemy funkcje $f \colon \Z_p \to \Q_p$.

\begin{fakt}
	Zbiór antypochodnych $f$ jest gęsty w zbiorze $\mathcal C(\Z_p, \Q_p)$.
\end{fakt}

\begin{proof}
	Jeśli $g' = 0$ (z wniosku) i $F' = f$, to $(F + g)' = f$.
\end{proof}

Funkcja $\sum_{n \ge 0} p^n x^{p^n- 1}$ jest analityczna, ale dla $x = 1$ jej antypochodna nie zbiega.
Natomiast antypochodna dowolnej funkcji lokalnie analitycznej też jest taka.

\begin{fakt}
	Antypochodna funkcji lokalnie analitycznej jest lokalnie analityczna, rzędu o jeden wyższego.
\end{fakt}

Funkcja $\sum_{n \ge 0} qx^{q-1}$, $q = p^n$, pokazuje, że słowa ,,lokalnie'' nie można pominąć.
Zastanowimy się teraz, czy każda ciągła funkcja $f \colon \Z_p \to \Q_p$ ma antypochodną.

\begin{fakt}
	Niech funkcje $f_1, f_2, \ldots$ będą ograniczone na $\Z_p$, tak by $f = \sum_{n \ge 1} f_n$ zbiegał jednostajnie.
	Jeśli $F_n$ jest antypochodną $f_n$ i $\|F_n\|_\triangle := \max (\|F_n\|, \|\Phi_1 F_n\|) \le \|f_n\|$, to szereg $\sum_n F_n$ także zbiega jednostajnie, do antypochodnej $f$.
\end{fakt}

\begin{proof}
	Z jednostajnej zbieżności $\|f_n\|$ dąży do zera, więc $\|F_n\|$ także (i $F$ jest dobrze określoną funkcją ciągłą). Ustalmy $\varepsilon > 0$ i $N$, takie że $\|f_n\| <\varepsilon$ dla $n > N$.

	Wtedy $|\Phi_1 F_n(s,t) - f_n(t)|_p \le \max(\|\Phi_1 F_n\|, \|f_n\|) < \varepsilon$ jednostajnie w $s \neq  t \in \Z_p$.
	Dla $x$ bliskich $a$, lewa strona jest mniejsza od $\varepsilon$ również dla $n \le N$ ($x, a$ zamiast $s, t$), można zatem ,,zgubić'' indeksy $n$.
\end{proof}

Jeżeli funkcje $F_n$ są klasy $\mathcal C^1$, dowód daje się poprawić tak, by $F$ też była.

\begin{twierdzenie}[Dieudonné]
	Każda ciągła $f \colon \Z_p \to \Q_p$ posiada antypochodną.
\end{twierdzenie}

\begin{proof}
	Lemat ze spostrzeżeniem pokazują, iż wystarczy nam ograniczyć się do lokalnie stałych $f$ i pokazać antypochodną $F$ klasy $\mathcal C^1$, $\|F\|_\triangle \le \|f\|$.
	Niech $f = \sum_{m=0}^{q} \lambda_m [x \in m + p^n\Z_p]$ dla $q = p^n-1$.
	Wtedy naturalnym kandydatem na funkcję $F$ jest $\sum_m \lambda_m (x-m) [x \in m + p^n\Z_p]$.
	Jej pochodna to $f$, zaś $\|F\| \le \|f\|$.

	Pokażemy, że $|\Phi_1 F(x,y)|_p \le \|f\|$ dla różnych $x, y \in \Z_p$.
	Niech $x \in m + p^n\Z_p$, $y \in m' + p^n \Z_p$.
	Jeśli liczby $m$ i $m'$ są równe, to $F(x) - F(y) = \lambda_m (x - y)$.
	Jeśli nie, to $|x-y|_p$ można szacować z dołu przez większą z $|x - m|_p$, $|y - m'|_p$, ale wtedy $|F(x) - F(y)|_p = |\lambda_m(x - m) + \lambda_{m'}(y - m')|_p \le \|f\| \cdot |x -y|_p$, co kończy dowód.
\end{proof}

\begin{fakt}
	Funkcja $f \colon X \to \cialo$ jest pierwszej klasy Baire'a (jest punktową granicą f. ciągłych), wtedy i tylko wtedy gdy posiada ona antypochodną.
\end{fakt}

Twierdzenie to nie ma rzeczywistego odpowiednika, ale wiadomo, że tylko implikacja w lewo jest prawdziwa.
Dowód w lewo (ultrametryczny) również jest prosty.
W prawo trzeba skorzystać z kilku lematów.

\begin{lemat}
	Jeśli $f \colon X \to \cialo$ jest pierwszej klasy Baire'a, to jest sumą przeliczalnie wielu lokalnie stałych $f_n \colon X \to \cialo$, takich że $\|f_n\| \le \|f\|$ (dla ograniczonych $f$).
\end{lemat}

\begin{lemat}
	Ustalmy $\varepsilon > 0$ oraz domkniętą kulę $\kula \subseteq X$.
	Istnieje wtedy liniowe odwzorowanie $F \colon X \to \cialo$, antypochodna indykatora $\kula$, normy co najwyżej $\varepsilon$, takie że $|F(x-y)| \le |x- y|$ (dla $x, y \in \kula$) i lub ,,$\le |x-y|$'' (w pozostałych przypadkach), zerowe poza kulą $\kula$.
\end{lemat}

\begin{lemat}
	Jeśli $f \colon X \to \cialo$ jest lokalnie stała, zaś $\varepsilon > 0$, to istnieje lokalnie liniowa antypochodna $F \colon X \to \cialo$, że $\|F\| \le \varepsilon$ i $|F(x) - F(y)| \le \max (|f(x)|, \varepsilon) |x-y|$.
\end{lemat}	