\section{Algebra Tate'a}
Pierścień szeregów formalnych $\cialo[[X]]$ zawiera podprzestrzeń wektorową złożoną z szeregów, których współczynniki zbiegają do zera, zwaną $\cialo\{X\}$.

Podprzestrzeń ta jest izomorficzna z p. Banacha $c_0(\cialo)$ i jest uzupełnieniem $\cialo[x]$ z normą Gaußa (to znaczy supremum współczynników).
Zachodzi nierówność $\|fg\| \le \|f\| \cdot \|g\|$, z której wynika (jednostajna) ciągłość mnożenia w $\cialo[x]$ i jego ciągłość w $\cialo\{x\}$.

Z tego względu $\cialo\{x\}$ jest pierścieniem i algebrą Banacha, algebrą Tate'a jednej zmiennej nad $\cialo$.
\textbf{Algebra Tate'a} pozostaje chwilowo poza naszym zasięgiem. 