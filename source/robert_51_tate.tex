\section{Algebra Tate'a}
\begin{definicja}
	$\cialo\{x\} = \{\sum_n a_n x^n : \lim_{n \to \infty} a_n = 0\}$ jest podprzestrzenią $\cialo[[x]]$.
\end{definicja}

\begin{fakt}
	Zbiór $\cialo\{x\}$ jest izomorficzny z p. Banacha $c_0(\cialo)$ i stanowi uzupełnienie $\cialo[x]$ z normą Gaußa (supremum współczynników).
\end{fakt}

Zachodzi nierówność $\|fg\| \le \|f\| \cdot \|g\|$, z której wynika (jednostajna) ciągłość mnożenia w $\cialo[x]$ i jego ciągłość w $\cialo\{x\}$.

\begin{fakt}
	Przestrzeń $\cialo\{x\}$ jest nawet algebrą Banacha, \emph{algebrą Tate'a} jednej zmiennej.
\end{fakt}

Algebra Tate'a pozostaje chwilowo poza naszym zasięgiem. 