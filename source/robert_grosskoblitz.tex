\section{Eksponens (Dworka)}
Pierwiastkami równania $x + x^p:p = 0$ są zero i pierwiastki $x^{p-1} + p = 0$, ,,$\pi$''
Problem w tym, że nie możemy dokonać tu podstawienia: $\exp(\pi + \pi^p:p) \neq \exp(0) = 1$!

W klasycznym $\C$-przypadku pierwiastki jedności dają się określić jako specjalne wartości eksponensy.
Okazuje się, że tutaj również, poniższa obserwacja pochodzi od Dworka.

\begin{fakt}
	Niech $\pi^{p-1} + p = 0$ oraz $\zeta_\pi := \exp(\pi + \pi^p:p)$.
	Wtedy $\zeta_\pi$ jest $p$-tym pierwiastkiem jedności równym $1 + \pi$ mod $\pi^2$.
\end{fakt}

Zauważmy, że zazwyczaj takie $\pi$ żyje w $\Q_p^a$.

\begin{definicja}
	Szereg Dworka to $E_\pi(x) := \exp(\pi(x-x^p))$, jest elementem $\Q_p(\pi)[[x]]$.
\end{definicja}

\begin{twierdzenie}[Dwork]
	Rozszerzenie $\Q_p(\pi)$ jest całkowicie oraz poskromienie rozgałęzione, stopnia $p-1$, Galois: to $\Q_p(\mu_p)$.
\end{twierdzenie}

Dokładniej:
\begin{enumx}
	\item dokłanie jeden element $\zeta_\pi \in \Q_p(\pi)$ jest x pierwiastkiem jedności ($p$-tym), że $\zeta_\pi \equiv 1 + \pi$ mod $\pi^2$
	\item promień zbieżności $E_\pi(x)$ to $p^\beta > 1$, $\beta = 1/p - 1/p^2$.
	\item jeśli $a \in \Q_p$ i $a^p = a$, to $E_\pi(a)^p = 1$, $E_\pi(a) \equiv 1 + a\pi$ mod $\pi^2$.
\end{enumx}

Nastał czas sum Gaußa.

\begin{definicja}
	Morfizm $\psi \colon \mathbb F_q \to \cialo^\times$ to addytywny charakter $\mathbb F_q$.
	Morfizm $\chi \colon \mathbb F_q^\times \to \cialo^\times$, $\chi(0) = 0$ to multiplikatywny charakter $\mathbb F_q$, gdzie $q$ jest potęgą $p$.
	Nazewnictwo jest tradycyjne.
\end{definicja}

\begin{fakt}
	Każda rodzina homomorfizmów $\grupa \to \cialo^\times$ jest liniowo niezależna w $\cialo$-liniowej p. funkcji $\grupa \to \cialo$ (grupa, ciało).
\end{fakt}

\begin{fakt}
	Jeśli $\tau \colon \mathbb F \to \cialo^\times$ to nietrywialny charakter (addytywny) skończonego ciała, nie ma innych charakterów niż $\psi(x) = \tau(ax)$ dla różnych $a \in \mathbb F$.
\end{fakt}

Przejdźmy do tw. Grossa-Koblitza.

% R = B_\le 1 \sup P 		A
% P = B_<   1 				M

Wybierzmy pierwotny $p$-ty pierwiastek jedności $\zeta_p \in \C_p$, niech $\cialo = \Q_p(\zeta_p)$.
Jak widzieliśmy, ideał $\mathfrak p$ jest maksymalny w $\mathcal O_p$.
Istnieje generator (jedyny!) $\pi$ dla $\mathfrak p$, że $\pi^{p-1} = -p$ oraz $\pi \equiv \zeta_p - 1$ mod $(\zeta_p-1)^2$.
Odwrotnie, wybór $\pi$, pierwiastka $\pi^{p-1} = -p$ w $\C_p$, daje ciało $\cialo = \Q_p(\pi)$, czyli rozszerzenie Galois dla $\Q_p$, ze wszystkimi pierwiastkami jedności rzędu $p$.
Tylko jeden z nich spełnia $\zeta_p \equiv 1 + \pi \mbox{ mod } \pi^2$ (szereg Dworka $E_\pi(1) = \zeta_p$).

Addytywny charakter $\mathbb F_p$ jest określony przez $\psi(1) \in \mu_p$.
Wybierzmy $\psi(1) = \zeta_{\pi}$, wtedy $\psi(v) = \zeta_\pi^v$ dla $v \in \mathbb F_p$.
Od tej chwili rozpatrywać możemy sumy Gaußa postaci
\[
	G(\chi, \psi) = \sum_{x \in \mathbb F_p} \chi(x) \zeta_\pi^x, \hfill \chi(0) = 0,
\]
$\chi$ jest multiplikatywnym charakterem $\mathbb F_p$ z wartościami w $\cialo$.
Dokładniej, wartości $\chi$ to pierwiastki jedności rzędu dzielącego $p-1$ (i $0$): $G(\chi, \psi) \in \Q(\mu_p, \mu_{p-1}) = \Q(\mu_{p(p-1)})$.
Interesujące są sumy Gaußa postaci
\[
	\sum_{x \in \mathbb F_p^\times} \omega(x)^{-a} \psi(x) = \sum_{x \in \mathbb F_p} \omega(x)^{-a} \zeta_\pi^x, \hfill \omega(0) = 0
\]
gdzie $\omega(x) \in \mu_{p-1}$ to jedyny pierwiastek jedności w $\cialo$ mający redukcję $x$ w ciele residuów $\mathcal O/\mathfrak p$ dla $\cialo$.
Tutaj, całkowita $a$ liczy się tylko mod $p-1$: lepiej wziąć $\alpha \in \frac{1}{p-1} \Z/\Z$ i położyć
\[
	G_\alpha = - \sum_{x \in \mathbb F_p} \omega(x) ^{-(p-1)a} \zeta_\pi^x, \hfill \omega(0) = 0
\]
Wybieramy taki znak, by $G_0$ było równe $1$.
%Powodem dla tego wyboru znaku jest to, że teraz $G_0 = 1$.
%\[ \sum_{v=0}^{p-1} \zeta_\pi^v = 0 \implies \sum_{v=1}^{p-1} \zeta_\pi^v = -1.\]
Warto nadmienić, że te sumy Gaußa związane są z $p$-adyczną funkcją $\Gamma$ Mority: kiedy $\alpha = a/(p-1)$ dla $0 \le a < p-1$, możemy napisać wprost: $G_\alpha = \pi^a \Gamma_p (\alpha)$ (szczególny przypadek tw. Grossa-Koblitza).

%\[ G_\alpha = \pi^a \Gamma_p \left( \frac a {p-1}\right).\]
%To szczególny przypadek twierdzenia Grossa-Koblitza.
Wartości $\Gamma_p$ są jednościami w $\Z_p$, poprzedni wzór daje więc dokładny rząd $|G_\alpha| = |\pi|^a = r_p^a = |p|^{a/(p-1)}$.
Przytoczone wyżej twierdzenie pokazuje, że $\Gamma_p(a/(p-1)) \in \Q(\pi,\mu_{p^2-p})$, a ta wartość jest algebraiczna, bo $\pi^{p-1} = -p$.

Spróbujemy uogólnić to wszystko.
Niech $\alpha \in \Z_{(p)}$, $\Q \cap \Z_p$, będzie wymierną o mianowniku $N$ względnie pierwszym z $p$.
Dla wysokiej potęgi $q = p^f$ rozszerzenie $\mathbb F_q$ stopnia $f$ swego  prostego ciała zawiera $N$-ty pierwiastek jedności.
Będziemy więc pracować w poskromionym rozszerzeniu rozgałęzionym
${\cialo = \Q_p(\pi, \mu_{q-1}) \subset \C_p}$ o indeksie rozgałęzienia $e = p-1$, stopniu  residuum $f$, zatem stopnia $n = ef$ nad $\Q_p$.

Dla $\alpha \in \frac{1}{N} \Z/\Z \subset \frac{1}{q-1} \Z/\Z$ wybieramy reprezentację:
\[
	0 \le \langle \alpha \rangle = a/(q-1) < 1
\]
i piszemy $p$-adyczne rozwinięcie licznika:
\[
	a = a_0 + a_1p + \ldots + a_{f-1}p^{f-1} < q-1 < q =  p^f
\]

Niech $S_p(a) = a_0 + a_1 +\ldots + a_{f-1}$ oznacza sumę cyfr $a$.
Niech całkowite $a^{(i)}$ za rozwinięcie mają cykliczną permutację $a = a^{(0)}$.
Przykładowo $a^{(1)} = a_{f-1} + a_0p + \dots + a_{f-2}p^{f-1}$.

Z drugiej strony, jeśli nietrywialny addytywny charakter $\psi$ ciała prostego $\mathbb F_p$ jest nadal taki sam, to złożenie $\psi$ ze śladem
$\textrm{Tr} \colon \mathbb F_q \to \mathbb F_p$, $x \mapsto x + x^p + \dots + x^{*}$, (gdzie $* = p^{f-1}$) jest nietrywialnym addytywnym charakterem $\mathbb F_q$.

Nietrywialność śladu wynika z tego, że rozszerzenie $\mathbb F_q / \mathbb F_p$ jest rozdzielcze ($\mathbb F_q$ jest skończone).

Czas na niespodziankę.

\begin{twierdzenie}[Gross-Koblitz, 1979] 
	Wartość sumy Gaußa $G_\alpha$ dla $0 \le \alpha = a / (q-1) < 1$ to
	\[
		-\sum_{x \in \mathbb F_p}^{x\neq 0} \omega(x)^{-a} \psi (\textrm{Tr}(x)) = \pi^{S_p(a)} \prod_{j=0}^{f-1} \Gamma_p \left( \frac{a^{(j)}}{q-1} \right).
	\]
\end{twierdzenie}
