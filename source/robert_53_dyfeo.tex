\section{Dyfeomorfizmy}
\begin{definicja}
	Dyfeomorfizmem otwarniętych zbiorów jest każdy dwustronnie różniczkowalny homeomorfizm.
\end{definicja}

\begin{fakt}
	Niepuste \prawo{Schik\\74.1} otwarnięte podzbiory $\C_p$ są dyfeomorficzne.
\end{fakt}

\begin{proof}
	Jako że $\C_p$ nie jest lokalnie zwarte, ale jest ośrodkowe, wystarczy zapisać obydwa jako unię przeliczalnie wielu dysków ,,domkniętych'' i skleić liniowe mapy między nimi.
\end{proof}

\begin{fakt}
	Każde \prawo{Schik\\74.A} nieograniczone i otwarnięte podzbiory p. lokalnie zwartej są dyfeomorficzne.
	Jeśli jeden z homeomorficznych podzbiorów jest nieograniczony, to drugi też.
\end{fakt}

\begin{fakt}[,,Peano'']
	$\Q_p$ \prawo{Schik\\74.3} i $\Q_p^2$, $\Z_p$ i $\Z_p^2$ są homeomorficzne.
\end{fakt}

\begin{fakt}
	$\Z_p$ i $\Z_p \setminus p \Z_p$ są dyfeomorficzne tylko dla $p = 2$.
\end{fakt}
