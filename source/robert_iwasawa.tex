\section{Logarytm (japoński)}
Pracujemy \prawo{Rbrt\\5.4.5} w $\cialo = \C_p$.

\begin{fakt}
	$\log \exp x = x$, $\exp \log (1+x) = 1 +x$, o ile $|x| < r_p$.
\end{fakt}

(To było tylko przypomnienie).

\begin{wniosek}
	Morfizm $\log \colon 1 + \mathfrak p \to \cialo$ jest ,,na''.
	Jego jądro to $\mu(p^\infty)$.
	Obcięty do $1 + \kula(0, r_p)$ jest (injektywną) izometrią.
\end{wniosek}

\begin{fakt}
	Dokładnie jeden morfizm $f \colon \cialo^\times \to \cialo$ ma poniższe własności (logarytm Iwasawy $\iwasawa$): $f(p) = 0$ (normalizacja), zaś obcięcie $f$ do $\kula(1,1)$ pokrywa się ze zwykłym logarytmem (to znaczy szeregiem potęgowym).
\end{fakt}

\begin{proof}
	(Jednoznaczność) Podgrupy $p^\Q\mu_{(p)}$ i $1 + \mathfrak p$ generują całe $\cialo^\times$.
	Jest to oczywiste dla $\mu$, gdyż ciało $\cialo$ (charakterystyki 0) nie ma addytywnej torsji.
	Z drugiej strony, jeśli $x^a = p^b$, to $af(x) = bf(p) = 0$, co daje $f(x) = 0$.

	(Istnienie) Niech $f$ będzie zerem na $p^\Q\mu$, gdyż zgadza się to z logarytmem na przekroju $p^\Q\mu$ z $1 + \mathfrak p$, $\mu(p^\infty)$.
	Ale podgrupa $\mu(p^\infty)$ to dokładnie jądro logarytmu.
\end{proof}

\begin{fakt}
	Logarytm Iwasawy jest lokalnie analityczny: dla $a \neq 0$ i $|x - a| < |a|$ mamy
	$\iwasawa x = \iwasawa a - \sum_{k \ge 1}^\infty [1 - x/a]^k / k$.
\end{fakt}

\begin{fakt}
	Dla $x \in \Z_p^\times$, $(1-p)\iwasawa x = \sum_{k \ge 1} (1-x^{p-1})^k / k$.
\end{fakt}

\begin{fakt}
	Dla zupełnych podciał $K \subseteq \C_p$, $\iwasawa[K^\times] \subseteq K$.
\end{fakt}

\begin{fakt}
	Dla każdego ciągłego automorfizmu $\sigma$ ciała $\C_p$ prawdą jest $\iwasawa(x^\sigma) = (\iwasawa x)^\sigma$.
\end{fakt}

W swojej pracy doktorskiej (,,Prolongement de la fonction exponentielle en dehors de son cercle de convergence'') M. C. Sarmant-Durix pokazała, że funkcja wykładnicza przedłuża się na całe $\C_p$.
Robert zaś postanowił podążać za Schikhofem (ale nie widać tego tutaj).

\begin{fakt}
	Ustalmy $a \in \C_p$, że $|1 - a|_p <1$.
	Wtedy
	\[
		\iwasawa a = \lim_{n \to \infty}\frac{a^{p^n} - 1}{p^n}.
	\]

	Jeżeli $b \in \C_p$, $|b|_p = 1$ i $\omega$ jest charakterem Teichmüllera, to
	\[
		\iwasawa b = \lim_{n \to \infty}\frac{b^{p^{n!}} - \omega_p(b)}{\omega_p(b) p^{n!}}.
	\]

\end{fakt}

\color{black}