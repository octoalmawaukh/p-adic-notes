\section{Rozszerzenia kwadratowe}
Rozszerzymy teraz $\Q_p$ o pierwiastek z $\varepsilon \not \in \Q_p^\times$. Otrzymany tak zbiór, $\Q_p(\sqrt{\varepsilon}) = \{x + y \sqrt{\varepsilon} : x, y \in \Q_p\}$, jest ciałem.

\begin{lemat}
	Równanie $x^2 = a \in \Z_p^\times$, ma rozwiązanie $x \in \Q_p$, wtedy i tylko wtedy gdy $a_0$ jest kwadratem w $\mathbb F_p$ (dla $p \neq 2$) lub $a_ 1$ i $a_2$ są zerami (dla $p = 2$): $a = a_0 + a_1p + a_2p^2 + \ldots$.
\end{lemat}

Volovich z kolegami podaje nie do końca właściwy dowód, jako że nie chce skorzystać z lematu Hensela.
Ustalmy  jedność $\eta$, która nie jest kwadratem.

\begin{wniosek}
	Dla $p \neq 2$, liczby $\eta$, $p$, $p \eta$ nie są kwadratami.
\end{wniosek}

\begin{wniosek}
	Liczby $p$-adyczne są postaci $\varepsilon y^2$, gdzie $y \in \Q_p$, zaś $\varepsilon = 1, \eta, p$ lub $p\eta$ ($p \neq 2$).
	Istnieją trzy nieizomorficzne rozszerzenia stopnia dwa dla $\Q_p$: o pierwiastek z $\eta$, $p$ i $p \eta$.
\end{wniosek}

\begin{wniosek}
	Liczby $2$-adyczne są postaci $\varepsilon y^2$, gdzie $y \in \Q_p$, zaś $\varepsilon = \pm 1, \pm 2, \pm 3$ lub $\pm 6$.
	Istnieje zatem siedem nieizomorficznych rozszerzeń kwadratowych: o pierwiastki z $-1$, $\pm 2$, $\pm 3$ lub $\pm 6$.
\end{wniosek}

\begin{wniosek}
	Dla $p = 4k + 3$, $|x^2 + y^2|_p = \max \{|x|^2_p, |y|_p^2\}$.
\end{wniosek}

Jeśli $z = x + \sqrt{\varepsilon} y$, $x, y \in \Q_p$, to parę $(x, y)$ dla $z$ nazwiemy współrzędnymi kartezjańskimi.
Zbiór punktów, które spełniają równanie $z \overline z = c$, będzie dla nas ,,okręgiem''.
Niech $\Q_p^\varepsilon \le \Q_p^\times$ składa się z liczb $c$ tej postaci: $c = r^2$ lub $c = \kappa r^2$, gdzie $r \in \Q_p^\times$, $\kappa \in \Q_p^\varepsilon$ nie jest kwadratem.
Para $(\rho, \sigma)$, gdzie $\rho = r$ lub $\rho = \nu r$ ($\nu \in \Q_p^\times (\sqrt \varepsilon)$), $z = \rho \sigma$ i $\sigma\overline \sigma = 1$ to współrzędne biegunowe.
,,Okrąg'' $z \overline z = 1$ to $\{(1 + \varepsilon t^2, 2t) / (1 - \varepsilon t^2) : t \in \Q_p\}$, jest on zbiorem zwartym.

Obraz funkcji $\varphi \colon \Q_p \to \R_+$, $\varphi(x) = |x|_p \sum_{k = 0}^\infty x_k p^{-2k}$ jest przeliczalną unią rozłącznych, nigdzie gęstych zbiorów o mierze Lebesgue'a zero, które są doskonałe.

Przestrzeń $\Q_p^n$ z iloczynem skalarnym $\langle x \mid y \rangle$ danym przez $\sum_{i=1}^n x_i y_i$ zachowuje nierówność Cauchy'ego Schwarza.
Kula $B_\gamma(a)$ jest produktem kul $B_\gamma(a_i)$ (rozpatrujemy tu metrykę od normy $|x|_p = \max_i |x_i|_p$).