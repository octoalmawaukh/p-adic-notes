Liczby $p$-adyczne do matematyki wprowadził Kurt Hensel.
Oto, co chyba mogło być jego główną motywacją: pary $\Z$, $\Q$ i $\C[x]$, $\C(x)$ (pierścień -- ciało ułamków) są do siebie podobne.
Zarówno $\Z$ jak i $\C[x]$ są pierścieniami z jednoznacznością rozkładu: liczby pierwsze $p \in \Z$ odpowiadają wielomianom $x - x_0 \in \C[x]$.
Każdemu wielomianowi $P(x) \in \C[x]$ można przypisać jego rozwinięcie Taylora wokół $x_0$: $P(x) = \sum_{0 \le i \le n} a_i (x - x_0)^k$.

Elementy $\N$ również mają tę własność: jeżeli $p$ jest l. pierwszą, to $m = a_0 + \ldots + a_n p^n$, przy czym $a_i \in \Z \cap [0, p-1]$ jest dobrze znanym rozwinięciem w systemie o podstawie $p$.
Kodujemy tak lokalne informacje (rząd $x_0$ jako pierwiastka $P$, stopień podzielności $m$ przez $p$).
Analogia nie umiera tak łatwo. 
W $\C(x)$ istnieją szeregi Laurenta, zazwyczaj zawierające nieskończenie wiele wyrazów.

Spróbujemy stworzyć coś na ich kształt w $\Q$.
Oto przykład, który wyraża więcej niż tysiąc słów.
Gdy $p = 3$, to $24 : 17 = (2p+2p^2) : (2+2p+p^2) = p + p^3 + 2p^5 + p^7(\ldots)$.
Wszystkie szeregi Laurenta w potęgach $p$ o skończonym ogonie tworzą ciało ($\Q_p$).
Taka definicja jest jednak do niczego.
Później rozwiniemy tę analogię i uwypuklimy kilka różnic.

Oto tematyka kolejnych rozdziałów.
Zaczynamy od analizy rzeczywistej i kombinatoryki, by przejść potem do topologii i algebry.
Z pomocą teorii Galois i algebry liniowej budujemy niearchimedesowe ciało liczb zespolonych ($\C_p$) oraz jego sferyczne uzupełnienie ($\Omega_p$).
W połowie kończymy zwiedzanie i wyruszamy w naukową ekspedycję, chociaż to chyba wciąż za mało, by poprowadzić poważne badania.
Mam nadzieję, że Czytelnik znajdzie po lekturze tego skryptu ulubioną gałąź matematyki w $p$-adycznej odmianie.
Oby się tylko na niej nie powiesił.

Notatki te w żadnym razie nie próbują udawać przesadnie poważnego tekstu.
Niestety, ale nie są najprawdopodobniej wolne od błędów.
Dla uprzyjemnienia życia podaję, kto jako pierwszy (lub najrozsądniej) podał dane stwierdzenie, na marginesie.

{\color{white}.} \hfill \\
{\color{white}.} \hfill Leon Aragonés\\
{\color{white}.} \hfill Wrocław, Polandia\\
{\color{white}.} \hfill \today
