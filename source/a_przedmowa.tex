Chociaż Kurt Hensel odkrył liczby $p$-adyczne ponad sto lat temu, do dzisiaj wydają się one nieco tajemnicze i niezrozumiane.
Podczas sporządzania notatek im poświęconych nawet nie starałem się o formalny wydźwięk, żywię przy tym nadzieję, iż okażą się one użyteczne dla przynajmniej jednej osoby. % wielu osób.
%Wykraczają bowiem daleko poza wprowadzenie do tak zwanej analizy ultrametrycznej, sięgając do teorii liczb, geometrii i topologii, algebry, funkcjonalnej, zespolonej czy rzeczywistej analizy (teorii miary). 
%Dobry Bóg stworzył liczby naturalne, ja sam wolę nie znać rodowodu pozostałych.

Dokument oparty jest na kilku książkach, jakie zdążyły się ukazać, przed zabraniem się za lekturę nie trzeba jednak znać zbyt dużo matematyki.

Najważniejsze pozycje umieszczone są w bibliografii na końcu, mniej ważne odniesienia do istniejącej literatury można znaleźć w uwagach historycznych zamykających poszczególne rozdziały.
%Na kolejnych stronicach starałem się uwypuklić różnice pomiędzy analizą $p$-adyczną i rzeczywistą.
%Tym, co łączy ciała $\Q_p$ z lepiej znanymi $\R$ (lub $\C$), jest wartość bezwzględna.
%W przypadku $p$-adycznym udaje ona rząd wielkości: z nierówności $|p| < 1$ wynika zbieżność ciągu $p^n$ do zera, co stanowi łącznik z ciałami charakterystyki zarówno $p$, jak i $0$.

%Rozdziały pierwszy i drugi są zrozumiałe po standardowym kursie analizy rzeczywistej.
%Pojawia się ważny lemat Hensela o podnoszeniu rozwiązań równań ze skończonego ciała $\mathbb F_p$ do $\mathbb Z_p$ oraz reguła lokalno-globalna, która orzeka, iż zachowanie (globalnego) ciała $\Q$ można lepiej wyjaśnić analizując lokalne ciała $\Q_p$ oraz $\R$.
%Już tam widać nieoczekiwane znalezionych wiek temu liczb oraz przewagę szeregów potęgowych nad pochodną.
%Analiza z plusem (rozdział trzeci) wymaga włożenia większego wysiłku, dodatkowo niektóre stwierdzenia musiałem niestety obedrzeć z dowodów.
%Pozostawiłem za to przy życiu sekcję poświęconą rachunkowi cienistemu (\emph{umbral calculus}), choć jego związek z resztą książki jest wątpliwy.
%Określam także pierwszy odpowiednik całki Riemanna, ważną dla zastosowań całkę Volkenborna oraz zwracam uwagę na dalsze trudności wychodzące na światło dzienne podczas różniczkowania.

%Do imperium z czwartego rozdziału można wkroczyć po kursie topologii skupiony tylko na przestrzeniach metrycznych, aczkolwiek im więcej się wcześniej przeżyło, tym lepiej.
%Nieznajomość grup topologicznych nie stanowi przeszkody, jako że przywołuję je jedynie na potrzeby ustępu o solenoidzie, algebraiczno-topologicznej struktury, w której żyją wszystkie lokalne ciała dla $\Q$.
%Rozdział czwarty wymaga jeszcze mniej, nie trzeba znać teorii kategorii, gdyż granice rzutowe nie pojawiają się nigdzie poza piątą sekcją.

%Rozsądnie rozszerzamy ciała, aż do uzyskania liczb zespolonych z $p$-adyczną topologią ($\C_p$) oraz ich sferycznego domknięcia.

%Funkcje specjalne są chyba najciekawszym z punktu widzenia zastosowań.
%Dołączyłem m.in. opis analogonu funkcji $\zeta$ Riemanna, to znaczy $p$-adycznego przedłużenia Kuboty-Leopoldta, w oparciu o książkę Koblitza.
%W przyszłości zamierzam wspomnieć o wydajnych obliczeniach z pakietem \texttt{Pari/GP}.

%Potem zajmujemy się już tylko analizą: funkcjonalną nad ciałami innymi niż $\R$ i $\C$, równaniami różniczkowymi, analizą zespoloną i mechaniką kwantową (a właściwie jej zalążkiem).

%Zajmujący tylko jedną stronę (ale nadal zajmujący!) rozdział dziesiąty to zachęta do przeczytania jedynej w swoim rodzaju monografii trzech rosyjskich uczonych (\cite{volovich94}).
%Tu i ówdzie widać też inspirację Schikhofem (\cite{schikhof85}).
%W przyszłości planuję dopisanie dodatkowych ustępów o zastosowaniu metod $p$-adycznych w kombinatoryce czy teorii macierzy (na przykład w oparciu o nieco przestarzałą ,,Local fields'' Casselsa, \cite{cassels86}), nie wiem jednak, kiedy to nastąpi.

Liczby $p$-adyczne do matematyki wprowadził Kurt Hensel.
Oto, co chyba mogło być jego główną motywacją: pary $\Z$, $\Q$ i $\C[X]$, $\C(x)$ (pierścień -- ciało ułamków) są do siebie podobne.
Zarówno $\Z$ jak i $\C[X]$ są pierścieniami z jednoznacznością rozkładu: liczby pierwsze $p \in \Z$ odpowiadają wielomianom $X - \alpha \in \C[X]$.
Każdemu wielomianowi $P(X) \in \C[X]$ można przypisać jego rozwinięcie Taylora wokół $\alpha$: $P(X) = \sum_{0 \le i \le n} a_i (X - \alpha)^k$.

Elementy $\N$ również mają tę własność: jeżeli $p$ jest l. pierwszą, to $m = a_0 + \ldots + a_n p^n$, przy czym $a_i \in \Z \cap [0, p-1]$ jest dobrze znanym rozwinięciem w systemie o podstawie $p$.
Kodujemy tak lokalne informacje (rząd $\alpha$ jako pierwiastka $P$, stopień podzielności $m$ przez $p$).
Analogia nie umiera tak łatwo. 
W $\C(x)$ istnieją szeregi Laurenta, zazwyczaj zawierające nieskończenie wiele wyrazów.

Spróbujemy stworzyć coś na ich kształt w $\Q$.
Oto przykład, który wyraża więcej niż tysiąc słów.
Gdy $p = 3$, to $24 : 17 = (2p+2p^2) : (2+2p+p^2) = p + p^3 + 2p^5 + p^7(\ldots)$.
Wszystkie szeregi Laurenta w potęgach $p$ o skończonym ogonie tworzą ciało ($\Q_p$).
Taka definicja jest jednak do niczego.
Później rozwiniemy tę analogię i uwypuklimy kilka różnic.

Oto tematyka kolejnych rozdziałów.
Zaczynamy od analizy rzeczywistej i kombinatoryki, by przejść potem do topologii i algebry.
Z pomocą teorii Galois algebry liniowej budujemy niearchimedesowe ciało liczb zespolonych ($\C_p$) oraz jego sferyczne uzupełnienie ($\Omega_p$).
W połowie kończymy zwiedzanie i wyruszamy w naukową ekspedycję, chociaż to chyba wciąż za mało, by poprowadzić poważne badania.
Mam nadzieję, że Czytelnik znajdzie po lekturze tego skryptu ulubioną gałąź matematyki w $p$-adycznej odmianie.
Oby się tylko na niej nie powiesił.


{\color{white}.} \hfill \\
{\color{white}.} \hfill Leon Aragonés\\
{\color{white}.} \hfill Wrocław, Polandia\\
{\color{white}.} \hfill \today
