Dobry Bóg stworzył liczby naturalne, reszta jest dziełem człowieka, ale $p$-adyczne szkarady mają jeszcze inny rodowód.
Nie należą bowiem do standardowej matematyki poznawanej na studiach, choć odgrywają ważną rolę we współczesnej teorii liczb.
Stanowią wygodny język do opisu kongruencji, a jednocześnie umożliwiają stosowanie typowo analitycznych narzędzi (jak szeregi potęgowe).

Rozdziały: pierwszy, drugi, szósty, a także częściowo: czwarty z piątym przypominają bardziej zwiedzanie niż naukową ekspedycję i nie są wystarczające do prowadzenia poważnych badań.
Ich treść oparta jest na książce F. Gouvea \cite{gouvea97}, podaje ona rozsądny przepis na rozszerzanie ciał, a przy tym nie odstrasza mniej doświadczonych podróżników formalizmem.

Bardziej wymagające jest dzieło Roberta \cite{robert99}, które nadal podąża za analogiami z klasyczną analizą.
Fałszywość twierdzenia o wartości średniej utrudnia życie, ale nowa wartość bezwzględna nie jest bezużyteczna.
Autor opisuje bardziej skomplikowane struktury: solenoidy, ,,właściwsze'' ciała $\C_p$ i $\Omega_p$, a także przedstawia analizę zespoloną, funkcjonalną i niektóre funkcje specjalne.
Chodzi tu o funkcję $\Gamma$ Mority, eksponens Artina-Hassego i logarytm Iwasawy.
Bez niego rozdziały: czwarty z piątym, siódmy, ósmy i dziewiąty wyglądałyby zupełnie inaczej.

Jeszcze trudniejsza w lekturze jest ,,Analytic Elements in $p$-adic Analysis'' spod pióra Escassuta; w tych notatkach nie ma jednak odniesień do niej. 
Dołączyłem jednak opis analogonu funkcji $\zeta$ Riemanna, to znaczy $p$-adycznego przedłużenia Kuboty-Leopoldta do $\Q_p$ w oparciu o książkę Koblitza \cite{koblitz84} (który to korzysta z nieopublikowanych notatek Mazura).
Brakuje w niej twierdzenia Hassego-Minkowskiego (można je znaleźć w ,,Number Theory'' Borewicza i Szafarewicza), pracy dyplomowej Tate'a (do znalezienia w podręczniku ,,Algebraic Number Theory'' Langa).
Temat $p$-adycznych $L$-funkcji odpowiadających charakterom Dirichleta też jest nie do końca zgłębiony.

Zajmujący tylko jedną stronę (ale nadal zajmujący!) rozdział dziesiąty to zachęta do przeczytania jedynej w swoim rodzaju monografii trzech rosyjskich uczonych (\cite{volovich94}).
Tu i ówdzie widać też inspirację Schikhofem (\cite{schikhof85}).
W przyszłości planuję dopisanie dodatkowych ustępów o zastosowaniu metod $p$-adycznych w kombinatoryce czy teorii macierzy (na przykład w oparciu o nieco przestarzałą ,,Local fields'' Casselsa, \cite{cassels86}), nie wiem jednak, kiedy to nastąpi.

Liczby $p$-adyczne do matematyki wprowadził Kurt Hensel.
Oto, co chyba było jego główną motywacją: pary $\Z$, $\Q$ i $\C[X]$, $\C(x)$ (pierścień -- ciało ułamków) są do siebie podobne.
Zarówno $\Z$ jak i $\C[X]$ są pierścieniami z jednoznacznością rozkładu: liczby pierwsze $p \in \Z$ odpowiadają wielomianom $X - \alpha \in \C[X]$.
Każdemu wielomianowi $P(X) \in \C[X]$ można przypisać jego rozwinięcie Taylora wokół $\alpha$: $P(X) = \sum_{0 \le i \le n} a_i (X - \alpha)^k$.

Elementy $\N$ również mają tę własność: jeżeli $p$ jest liczbą pierwszą, to $m = a_0 + \ldots + a_n p^n$, przy czym $a_i \in \Z \cap [0, p-1]$ jest dobrze znanym rozwinięciem w systemie o podstawie $p$.
Jest to dobre, gdyż zawiera lokalne informacje (rząd $\alpha$ jako pierwiastka $P$, stopień podzielności $m$ przez $p$).
Analogia nie umiera tak łatwo. 
W $\C(x)$ istnieją szeregi Laurenta, zazwyczaj zawierające nieskończenie wiele wyrazów.
Spróbujemy stworzyć coś na ich kształt w $\Q$.
Oto przykład, który wyraża więcej niż tysiąc słów.
Niech $p = 3$.
Wtedy $24 : 17 = (2p+2p^2) : (2+2p+p^2) = p + p^3 + 2p^5 + p^7 + p^8 + 2p^9 + \ldots$.
Wszystkie szeregi Laurenta w potęgach $p$ o skończonym ogonie tworzą ciało ($\Q_p$).
Taka definicja jest jednak do niczego.

Mam nadzieję, że Czytelnik znajdzie po lekturze tego skryptu ulubioną gałąź matematyki w $p$-adycznej odmianie.
Oby się tylko na niej nie powiesił.

\textbf{Notacja}.
Oznaczamy ciała przez $\cialo$, pierścienie: $\pierscien$, grupy: $\grupa$, przestrzenie liniowe: $\liniowa$, kule: $\kula$.
Indeks sumowania to zazwyczaj $k$ lub $i$, pod literą $j$ zawsze kryje się jednostka urojona, zaś $m$ i $n$ to najczęściej jakieś liczby całkowite.
Elementy ciała to $x, y, z$, promienie kul: $r, R$, bliżej nieokreślone stałe: $C$.
Kule domknięte odróżniamy od otwartych nawiasami ($\kula[\cdot, \cdot]$ kontra $\kula(\cdot, \cdot)$), jako że domknięcie kuli otwartej zazwyczaj nie jest kulą domkniętą o tym samym promieniu.
Wykładnikami są greckie litery: $\alpha, \beta$, niekoniecznie naturalnymi, ale $\lambda$ to tylko mnożnik.

{\color{white}x} \hfill \\
{\color{white}x} \hfill Leon Aragonés\\
{\color{white}x} \hfill Wrocław, Polandia\\
{\color{white}x} \hfill \today
\fancyhead[LE, RO]{\nouppercase{\leftmark}}