\section{Topologia teoriomnogościowa}
\begin{twierdzenie}[Sierpiński]
	Przeliczalna przestrzeń metryczna bez izolatorów jest homeomorficzna z $\Q$.
\end{twierdzenie}

Stąd bierze się zaskakująca własność $\Q$.

\begin{wniosek}
	Topologia na $\Q$ ,,nie zależy'' od normy.
\end{wniosek}

\begin{twierdzenie}[Vaughan, 1937]
	Przestrzeń metryczną $X$ można zmetryzować tak, by zwarte były dokładnie domknięte i ograniczone podzbiory, wtedy i tylko wtedy gdy $X$ jest ośrodkowa i lokalnie zwarta.
\end{twierdzenie}

\begin{przyklad}
	 $X = \Q_p$.
\end{przyklad}

Wymiar \prawo{Vlad.\\???} zupełnej przestrzeni metrycznej $X$ to najmniejsza liczba całkowita $n$, że w każde pokrycia $X$ można wpisać inne pokrycie krotności $n + 1$ (krotność to największa całkowita $m$, dla której można wybrać $m$ zbiorów o niepustym przekroju).
Przykładowo $\dim \R^n = n$.

\begin{fakt}
	Przestrzeń $\Q_p$ jest zerowymiarowa.
\end{fakt}

\begin{proof}
	Każdy otwarty podzbiór $X \subseteq \Q_p$ jest przeliczalną unią rozłącznych dysków.
\end{proof}	

