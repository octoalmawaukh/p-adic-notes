\section{Pierścień adeli}
W \prawo{Cnrd\\???} rozdziale poświęconym mechanice kwantowej pojawiają się charaktery ciała $\Q_p$.
Można je analogicznie określić na przykład dla $\Q$ i tym właśnie się zajmiemy -- a dokładniej ich klasyfikacją.
Do tego celu użyjemy pierścienia adelicznego.

\begin{definicja}
	Pierścień adeli $A_\Q \subsetneq \R \times \prod_p \Q_p$ składa się z tych ciągów, których wyrazy od pewnego miejsca leżą w (stosownych) $\Z_p$.
	Działania (dodawanie i mnożenie) określone są punktowo.
\end{definicja}

Adele leżą między sumą $\bigoplus_p \Q_p$ i produktem $\prod_p \Q_p$.
Ich elementy oznaczamy przez $(a_\infty, a_2, a_3, \ldots)$.
Ciało $\Q$ wkłada się w $A_\Q$ przekątniowo, $r \mapsto (r, r, r, \ldots)$.

\begin{definicja}
	Jeśli $a$ jest adelem, to $\Psi_a(r) = \chi_\infty(r a_\infty) \prod_p \chi_p(r a_p)$ jest charakterem $\Q$.
\end{definicja}

\begin{fakt}
	Odwzorowanie $a \mapsto \Psi_a$ jest surjekcją, którego jądro to wymierne adele.
\end{fakt}

\begin{proof}
	,,The character group of $\Q$'', Keith Conrad.
\end{proof}