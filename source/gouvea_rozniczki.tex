\section{Bezmyślne różniczkowanie}	
Metryka \prawo{Gouvea\\4.2} zadaje ciągłość.
Niestety, w $\Q_p$ nie można pracować z przedziałami (bo ich nie ma); można jednak definiować funkcje na kulach (otwar...niętych). 
Upośledzona definicja pozwoli nam udawać, że różniczkujemy, chociaż do przyszłego rozdziału nie będziemy tego potrafić.

\begin{definicja}
	Niech $U \subseteq \Q_p$ będzie zbiorem otwartym.
	Funkcja $f \colon U \to \Q_p$ jest ciągła w punkcie $y \in U$, jeśli dla każdego $\varepsilon > 0$ istnieje $\delta > 0$, że ,,$|x-y| < \delta$ pociąga $|f(x) - f(y)| < \varepsilon$''.
\end{definicja}

%Jeżeli zbiór $U$ jest zwarty, to $f$ jest jednostajnie ciągła (dlaczego?).

Użyteczność pochodnej (jako granicy ilorazów różnicowych) jest jednak ograniczona, jako że nie mamy do dyspozycji twierdzenia o wartości średniej, nawet dla wielomianów.

\begin{fakt} [fałszywy]
	Jeśli funkcja $f$ jest różniczkowalna na $\Q_p$ i ma ciągłą pochodną, zaś $x,y \in \Q_p$, to istnieje taka liczba $z \in \Q_p$ postaci $\lambda x + (1-\lambda)y$ z $|\lambda| \le 1$, że $f(y) - f(x) = f'(z) (y-x)$.
\end{fakt}

\begin{proof}
	Niech $f(t) = t^p - t$, $x = 0$, $y = 1$.
	Nie ma takiego $z_\lambda = 1 - \lambda$ z $\lambda \in \Z_p$, żeby $f'(z_\lambda) = 0$: w takiej stuacji pochodna się odwraca (!) i nie może być zerem.
\end{proof}

Smutku wiele.
Łańcuchowa reguła (o różniczkowaniu złożenia) pozostaje prawdziwa w nowym ciele.
Jeżeli funkcja $f$ ma zerową pochodną, zaś $g$ jest ciągle różniczkowalna, to pochodna zarówno $f \circ g$, jak i $g \circ f$ zeruje się w całej dziedzinie.
To tłumaczy, skąd bierze się całe multum funkcji $\Q_p \to \Q_p$, które mają zerową pochodną, ale nie są lokalnie stałe.

Twierdzenie o wartości średniej uratujemy później, w ślad za Robertem (po delikatnym wzmocnieniu założeń).