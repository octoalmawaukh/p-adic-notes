\section{Logarytm i eksponens \label{sekcjalog}}
Zajmijmy \prawo{Gouv\\4.5} się grupą $\Z_p^\times$.
Lemat Hensela pokazał, że zawiera $(p-1)$-sze pierwiastki z jedynki.
Dla prostoty ,,$q_2 = 4$ i $q_p = p$''.
% Eksponensa $p$-adyczna będzie określona dla $x \in q \Z_p$, zaś logarytm dla $x \in 1+ p \Z_p$.
Podzbiory $U = 1 + p\Z_p$ i $U_1 = 1 + q\Z_p$ są podgrupami $\Z_p^\times$, $U_1 \subset U$. Jeśli $p = 2$, to $U = \Z_p^\times$, jeśli nie, to $U_1 = U$.
Niech $W = (q\Z_p, +)$.

\begin{definicja}
	Elementy $U$ nazywa się \emph{jeden-jednościami}.
\end{definicja}

\begin{fakt}
	Logarytm $p$-adyczny zadaje homomorfizm $U \to \Z_p^+$ oraz izomorfizm $U_1 \to W$ (funkcja odwrotna: eksponens).
	Wtedy $U_1 \cong W \cong \Z_p^+$ są beztorsyjne, $\log_p(U) \subseteq p\Z_p$.
\end{fakt}

\begin{proof}
	Oczywisty po ustępie \ref{parum}.
\end{proof}

\begin{wniosek}\label{hostis}
	Mamy $\Z_p^\times \cong V \times U$, gdzie $U \cong \Z_p^+$ to beztorsyjna pro-$p$-grupa, zaś $V \le \Z_p^\times$ składa się z pierwiastków jedności w $\Q_p$ (część torsyjna).
	Jest cykliczna, rzędu $\varphi(q)$.
\end{wniosek}

\begin{proof}
	Łatwo widać, że istnieje ciąg dokładny $1 \to U_1 \to \Z_p^\times \stackrel \pi \to  (\Z/q\Z)^\times \to 0$ (jest to definicja $U$).
	Chcemy pokazać, że się ,,rozdziela''.
	Lemat Hensela z twierdzeniem Straßmana mówią, że $\Z_p^\times$ zawiera grupę $V$ pierwiastków jedności, cykliczną i rzędu $\varphi(q)$.
	Elementy $V$ są różne modulo $q$ (lemat Hensela dla $p > 2$).
	Jeżeli $\pi(\zeta_1) = \pi(\zeta_2)$, to $\zeta_1 / \zeta_2 = 1 + qx \in U$ dla $x \in \Z_p$, czyli $\zeta_1 = \zeta_2$ mod $q$, a jedyny przypadek, kiedy to jest możliwe, to $\zeta_1 = \zeta_2$.

	Strzałka $\pi$ indukuje izomorfizm $V \cong (\Z/q\Z)^\times$.
	Reszta jest prosta: dla $u \in \Z_p^\times$ istnieje $\zeta \in V$, takie że $\pi(\zeta) = \pi(u)$, wtedy $u \mapsto (\zeta, u \zeta^{-1})$ uzasadnia $\Z_p^\times \cong V \times U$.
\end{proof}

\section{Charakter Teichmüllera\label{sekcjateich}}
Z faktu \ref{hostis} możemy wywnioskować dla pierwszej $p \neq 2$ istnienie $\omega \colon \mathbb F_p^\times \cong V \hookrightarrow \Z_p^\times$ przedłużonej przez $\omega(0) = 0$.

\begin{definicja}
	\emph{Charakter Teichmüllera} to funkcja $\omega$.
\end{definicja}

Czasem tej nazwy używa się dla charakteru Dirichleta, to jest złożenia $\omega$ z redukcją mod $p$: $\Z \to \mathbb F_p \to \Z_p$.

Żeby nie było zbyt łatwo, często przez $\omega$ oznacza się rzut z $\Z_p^\times$ na $V$, by każdy $x\in \Z_p^\times$ zapisywał się jednoznacznie jako $x = \omega(x) \cdot x_1$ z $x_1 \in 1 + q\Z_p$.
Taka definicja jest sensowna: rozszerzając rzut na $\Z_p$ (do zera na niejednościach) i obcinając do $\Z$, dostaniemy charakter Dirichleta.
Niech ,,$(x) = x_1$''.

\begin{fakt}
	Jeżeli $p \neq 2$ i $x \in \Z_p^\times$, to $\omega(x)$ dane jest wzorem
	\[
		\omega(x) = \lim_{n \to \infty} x^{p^n}.
	\]
\end{fakt}

\begin{proof}
	Skoro $\omega(x)^{p-1} = 1$, to $\omega(x)^{p^n} = \omega(x)$ dla każdego $n$.
	Z drugiej strony, $x_1 = 1 + qy$, więc $(1 + qy)^p = 1 + pqy + q^2 \cdot$ reszta, a stąd wynika $x_1^p \in 1 + p^2 \Z_p$.
	Można to powtórzyć: indukcyjnie pokazuje się, że $x_1^{r} \in 1 + p^{n+1} \Z_p$ i $x_1^r \to 1$ ($r = p^n$).
\end{proof}

Jeżeli $p = 2$, to część odkryć trzeba poprawić: $\mathbb F_2^\times = \{1\}$.