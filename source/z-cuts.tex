\newcommand{\prawo}[1]{\mymarginpar{\small{#1}}}

% Zbiory, N < Z < Q < R < C.
\newcommand{\N}{\mathbb N}
\newcommand{\Z}{\mathbb Z}
\newcommand{\Q}{\mathbb Q}
\newcommand{\R}{\mathbb R}
\newcommand{\C}{\mathbb C}

\newcommand{\grupa}{\mathcal G}
\newcommand{\pierscien}{\mathcal R}
\newcommand{\cialo}{\mathcal K}
\newcommand{\liniowa}{\mathcal V}
\newcommand{\kula}{\mathcal B}

\newcommand{\dysk}[2]{\mathrel{\langle\mkern-3.85mu\mid\mkern-4mu} {#1}\mathrel{\mkern-4mu\mid\mkern-3.85mu\rangle}_{#2}}  












































\DeclareFontFamily{U}{wncy}{}
\DeclareFontShape{U}{wncy}{m}{n}{<->wncyr10}{}
\DeclareMathOperator{\findif}{\nabla}
\DeclareMathOperator{\iwasawa}{Log}
\DeclareMathOperator{\logrob}{Log}
\DeclareMathOperator{\suma}{{\textstyle \reflectbox{S}}}

\DeclareSymbolFont{mcy}{U}{wncy}{m}{n}
\newcommand{\convulsion}{\oslash}
\newcommand{\intzp}{\int_{\Z_p}}
\newcommand{\kobint}{\oint}
\newcommand{\koblitzint}{\oint}


\newcommand{\uroj}{\textrm{i}}
\usepackage{upgreek} % uptau -- funkcja Tate'a	
\makeatletter % sha - obcięty produkt
\def\shaprod{\operatornamewithlimits{\mathchoice{\vcenter{\hbox{\huge $\Sh$}}}{\vcenter{\hbox{\Large $\Sh$}}}{\mathrm{\Sh}}{\mathrm{\Sh}}}}
\makeatother

\DeclareMathSymbol{\Sh}{\mathord}{mcy}{"58} 