\section{Rozszerzenia jako przestrzenie wektorowe z normą}
Dotychczas \prawo{Gouvea\\5.1} byliśmy skupieni na ciele $\Q$ i jego uzupełnieniach.
Czasami odczuwaliśmy jednak potrzebę, by rozważyć inne ciała: tak było na przykład wtedy, kiedy polowaliśmy na zera szeregów potęgowych.
Wiemy, że ciało liczb rzeczywistych można algebraicznie domknąć do ciała liczb zespolonych, przeprowadzimy podobną konstrukcję.
Okaże się, iż będzie to dużo zabawniejsze niż można było przypuszczać, jako że dołożenie jednego elementu nie wystarczy ($\C \cong \mathbb R[i]$).

Zaczniemy od przestrzeni wektorowych oraz norm, następnie wykorzystamy mocniej teorię ciał i przejdziemy od rozszerzeń skończonych do algebraicznego domykania.

Wybraliśmy jedną z dwóch dróg: zamiast rozszerzać $\Q_p$, można postąpić tak samo z ciałem $\Q$, jednak wymaga to więcej teorii Galois.
Nie zrobimy tego, natomiast ciekawego czytelnika odeślemy do książek poświęconych algebraicznej teorii liczb.
Przyjmujemy ściśle lokalną perspektywę, choć być może któregoś dnia to się zmieni.

Niech ciało $\cialo$ z wartością bezwzględną będzie zupełne.
Wprawdzie można zajmować się także ciałami charakterystyki $p$, dla świętego spokoju załóżmy jednak, że $\operatorname{char} \cialo = 0$.
Niech $\liniowa$ będzie przestrzenią wektorową nad ciałem $\cialo$.

\begin{definicja}
	Norma to funkcja $\|\cdot\| \colon \liniowa \to \R_+$ spełniająca:
	\begin{enumx}
	\item $\|v\| = 0$, wtedy i tylko wtedy gdy $v = 0$.
	\item jeśli $v, w \in \liniowa$, to $\|v+w\| \le \|v\| + \|w\|$.
	\item jeśli $v \in \liniowa$, $\lambda \in \cialo$, to $\|\lambda v\| = |\lambda| \cdot \|v\|$.
	\end{enumx}
\end{definicja}

Nie wprowadzamy pojęcia niearchimedesowej przestrzeni liniowej, gdyż taka definicja byłaby równie skomplikowana co bezużyteczna.
Przestrzeń nad niearchimedesowym ciałem $\cialo$ sama po prostu też taka jest.

\begin{definicja}
	Dwie normy na jednej przestrzeni są równoważne, gdy istnieją rzeczywiste stałe $C$ i $D$, obie większe od zera, że $\|v\|_1 \le C\|v\|_2 \le CD \|v\|_1$.
\end{definicja}

\begin{fakt}
	Dwie normy są równoważne, wtedy i tylko wtedy gdy zadają tę samą topologię.
	Wtedy ciągi Cauchy'ego względem nich pokrywają się.
\end{fakt}

\begin{proof}
	Dla dowodu implikacji w prawo wystarczy pokazać, że kula otwarta względem jednej normy jest też otwarta względem drugiej.
	Można ograniczyć się do jednej kuli, bo to wektorowa przestrzeń z normą.

	Dla $x \in \kula = \{x \in \liniowa : \|x\|_1 < 1\}$ przyjmijmy, że $r = \|x\|_1$ i weźmy $R < (1-r) / C$.
	Zbiór $N = \{y \in \liniowa : \|y-x\|_2 < R\}$, otwarta względem $\|\cdot\|_2$ kula, zawiera się w $\kula$, która (dzięki temu) jest otwarta względem $\|\cdot\|_2$.

	W drugą stronę można zaszaleć.
	Identyczność $i \colon \liniowa \to \liniowa$ (obie z różnymi normami) oraz odwrotna do niej są ciągłe i liniowe.
\end{proof}

\begin{fakt}
	Przestrzeń wektorowa $\liniowa$ nad zupełnym ciałem z normą i bazą $v_1, \ldots, v_m$ jest zupełna (z normą supremum).
	Ciąg jej wektorów $w_n = \sum_{k=1}^m a_{kn}v_k$ jest Cauchy'ego, wtedy i tylko wtedy gdy takie są ciągi jego współczynników $(a_{kn})$ w ciele $\cialo$.
\end{fakt}

\begin{proof}
	Norma to największy ze współczynników ,,bazowych'', zatem $\|w_{n_1} - w_{n_2}\|$ dąży do zera dokładnie wtedy, gdy do zera dążą wszystkie $a_{i n_1} - a_{i n_2}$.
\end{proof}

\begin{fakt}
	Weźmy $\liniowa = \Q_p[x]$ i ustalmy rzeczywiste $c > 0$.
	Wtedy $\|\cdot\|$ jest (multiplikatywną) normą na $\liniowa$, z którą ta jest zupełna.
	\[
		\left\|\sum_{k=0}^n a_k x^k \right\| = \max_{0 \le i \le n} |a_i| c^i
	\]
\end{fakt}

\begin{proof}
	Ciało $\C_p$ zna odpowiedź na wszystkie pytania.
\end{proof}