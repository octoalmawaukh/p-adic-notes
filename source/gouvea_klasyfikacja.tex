\section{Klasyfikacja wymiernych norm}
Znamy \prawo{Gouvea\\3.1} już trzy różne rodzaje norm na ciele liczb wymiernych: rodzinę norm $p$-adycznych, klasyczną oraz trywialną wartość bezwzględną.
Naturalne pytanie, jakie powinniśmy sobie postawić, brzmi: czy są jeszcze jakieś?

Przypomnijmy: dwie normy na ciele są równoważne dokładnie wtedy, gdy ich metryki generują dokładnie takie same topologie.
Łatwiej powiedzieć niż sprawdzić, ale na szczęście mamy poniższy lemat.

\begin{lemat}
	Następujące warunki są równoważne dla dwóch norm na jednym ciele $\cialo$:
	\begin{enumx}
		\item topologie od norm pokrywają się, 
		\item $\|x\|_1 < 1$, wtedy i tylko wtedy gdy $\|x\|_2 < 1$, 
		\item istnieje stała $\alpha > 0$, że dla $x \in \cialo$ jest $\|x\|_1 = \|x\|_2^\alpha$.
	\end{enumx}
\end{lemat}

\begin{proof}
	Pokażemy ciąg implikacji $3 \Rightarrow 1 \Rightarrow 2 \Rightarrow 3$.
	
	($3 \Rightarrow 1$) $\|x-y\|_1 < r$, wtedy i tylko wtedy gdy $\|x-y\|_2 < r^{1:\alpha}$.
	Otwarte kule są nadal otwarte, choć nie zawsze o takim samym promieniu. 

	($1 \Rightarrow 2$) Z każdą topologią związane jest pojęcie zbieżności.
	Zauważmy, że $x^n$ dąży do zera dokładnie dla $\|x\| < 1$.
	
	($2 \Rightarrow 3$) Wybierzmy $y \in \cialo^\times$, takie że $\|y\|_1 < 1$.
	Warunek nr 2 mówi, że $\|y\|_2$ też jest mniejsze od jeden.
	Wskazujemy więc $\alpha > 0$ takie, by $\|y\|_1 = \|y\|_2^\alpha$.

	Weźmy $x \in \cialo^\times$.
	Jeśli $\|x\|_1 = \|y\|_1$, to drugie normy także są równe, w przeciwnym przypadku jedna z liczb $x/y$ lub $y/x$ miałaby drugą normę większą niż jeden, przeczyłoby to założeniu.
	Jeśli $\|x\|_1 = 1$, to (patrząc na $x$ lub $1/x$) widzimy $\|x\|_2 = 1$, wtedy równość $\|x\|_1 = \|x\|_2^\alpha$ zachodzi w próżni.

	Zastępując (w miarę potrzeby) $x$ przez jego odwrotność, możemy założyć $\|x\|_1 < 1$ i wziąć $\beta > 0$, by $\|x\|_1 = \|x\|_2^\beta$.
	Ustalmy naturalne $n, m$.
	Wtedy $\|x\|_1^n < \|y\|_1^m$, wtedy i tylko wtedy gdy $\|x\|_2^n < \|y\|_2^m$.
	Wzięcie logarytmów daje (po drobnych przekształceniach)
	\[
		\frac nm < \frac{\log \|y\|_1}{\log \|x\|_1} \iff \frac n m < \frac{\log \|y\|_2}{\log \|x\|_2}.
	\]

	Prawostronne ułamki są równe, więc $\|y\|_1 = \|y\|_2^\alpha$ i rzeczywiście $\alpha = \beta$.
\end{proof}

\begin{wniosek}
	Dwie normy, z których dokładnie jedna jest archimedesowa, nie są równoważne.
\end{wniosek}

\begin{wniosek}
	Normy $p$- i $q$-adyczne są równoważne, wtedy i tylko wtedy gdy $p = q$.
\end{wniosek}



\begin{definicja}
	Dwie normy spełniające dowolny z trzech warunków lematu nazywamy równoważnymi.
\end{definicja}

\begin{twierdzenie}[Ostrowski, 1916]
	Każda norma na $\Q$ jest dyskretna lub równoważna z $\|\cdot\|_p$, gdzie $p \le \infty$ jest liczbą pierwszą.
\end{twierdzenie}

\begin{proof}
	Niech $\|\cdot\|$ będzie nietrywialną normą na $\Q$.
	Pierwszy przypadek: archimedesowa (odpowiada normie $|\cdot|_\infty$).
	Weźmy więc najmniejsze dodatnie całkowite $n_0$, że $\|n_0\| > 1$.
	Wtedy $\|n_0\| = n_0^\alpha$ dla pewnej $\alpha > 0$.
	Wystarczy uzasadnić, dlaczego $\|x\| = |x|_\infty^\alpha$ dla każdej $x \in \Q$, a właściwie tylko dla $x \in \N $ (gdyż norma jest multiplikatywna).
	Dowolną liczbę $n$ można zapisać w systemie o podstawie $n_0$: $n = a_0 + a_1 n_0 + \dots + a_mn_0^m$, gdzie $a_m \neq 0$ i $0 \le a_j \le n_0-1$.
	\begin{align*}
		\|n\| & = \left\|\sum_{i=0}^m a_in_0^i\right\| \le \sum_{i=0}^m \left\|a_i\right\| n_0^{i \alpha} \le n_0^{m \alpha} \sum_{i = 0}^m n_0^{-i \alpha} \le n_0^{m \alpha} \sum_{i = 0}^\infty n_0^{-i \alpha} = C n_0^{m \alpha} %  n_0^{m \alpha} \frac{n_0^\alpha}{n_0^\alpha - 1}
	\end{align*}

	Pokazaliśmy $\|n\| \le Cn_0^{m \alpha} \le C n^\alpha$ dla każdego $n$, a więc w szczególności dla liczb postaci $n^N$ (gdyż $C$ nie zależy od $n$): $\|n\| \le C^{1/n}n^\alpha$.
	Idziemy z $N$ do nieskończoności, wtedy $C^{1/n}$ zbiega do jedynki i otrzymujemy półrówność $\|n\| \le n^\alpha$.

	Teraz trzeba pokazać nierówność w drugą stronę.
	Skorzystamy jeszcze raz z rozwinięcia.
	Skoro $n_0^{m+1} > n \ge n_0^m$, to nie kłamczymy pisząc $\|n+n_0^{m+1} - n\| \le \|n\| + \|n_0^{m+1} - n\|$, 
	
	Wnioskujemy stąd, że $\|n\| \ge n_0^{(m+1) \alpha} - \|n_0^{m+1} - n\| \ge n_0^{(m+1)\alpha} - (n_0^{m+1} - n)^\alpha$.
	Skorzystaliśmy tutaj z nierówności udowodnionej wyżej.
	Wiemy, że $n \ge n_0^m$, zatem
	\begin{align*}
		\|n\| & \ge n_0^{(m+1)\alpha} - (n_0^{m+1} - n_0^m)^\alpha = n_0^{(m+1) \alpha} [1 - (1 - 1 : n_0)^\alpha] = C' n^\alpha.
	\end{align*}

	Od $n$ nie zależy $C' = 1 - (1-1:n_0)^\alpha$, jest dodatnia i przez analogię do poprzedniej sytuacji możemy pokazać $\|n\| \ge n^\alpha$.
	Wnioskujemy stąd, że $\|n\| = n^\alpha$ i $\|\cdot\|$ jest równoważna najzwyklejszej wartością bezwzględną.

	Załóżmy, że $\|\cdot\|$ jest niearchimedesowa.
	Wtedy $\|n\| \le 1$ dla całkowitych $n$.
	Ponieważ $\|\cdot\|$ jest nietrywialna, musi istnieć najmniejsza l. całkowita $n_0$, że $\|n_0\| < 1$.
	Zacznijmy od tego, że $n_0$ musi być l. pierwszą: gdyby zachodziło $n_0 = a \cdot b$ dla $1 < a,b < n_0$, to $\|a\| = \|b\| = 1$ i $\|n_0\| < 1$ (z minimalności $n_0$) prowadziłoby do sprzeczności.
	Chcemy pokazać, że $\|\cdot\|$ jest równoważna z normą $p$-adyczną, gdzie $p := n_0$.
	W następnym kroku uzasadnimy, że jeżeli $n \in \Z$ nie jest podzielna przez $p$, to $|n| = 1$.
	Dzieląc $n$ przez $p$ z resztą dostajemy $n = ap + b$ dla $0 < b < p$.
	Z minimalności $p$ wynika $\|b\| = 1$, zaś z $\|a\| \le 1$ ($\|\cdot\|$ jest niearchimedesowa) i $\|p\| < 1$: $\|ap\| < 1$.
	,,Wszystkie trójkąty są równoramienne'', więc $\|n\| = 1$.
	Wystarczy więc tylko zauważyć, że dla $n \in \Z$ zapisanej jako $n = p^v n'$ z $p \nmid n'$ zachodzi $\|n\| = \|p\|^v \|n'\| = \|p\|^v \le 1$.
\end{proof}

\begin{historia}[Ostrowski Aleksander]\end{historia}

Często w starciu z  teorioliczbowymi problemami warto pracować ze wszystkimi liczbami pierwszymi jednocześnie, to znaczy korzystać z informacji, jakich dostarczają różne normy na $\Q$.
Oto fundamentalny przykład, który pokazuje przy okazji, że $\infty$ też zasługuje być nazywana liczbą pierwszą.

\begin{fakt}[,,adelizm'' 1]
	Gdy $x \in \Q^\times$, to $\prod_{p \le 2} |x|_p = 1$.
\end{fakt}

\begin{proof}
	Wystarczy użyć zasadniczego twierdzenia arytmetyki.
\end{proof}

Podobne stwierdzenie jest prawdziwe dla skończonych rozszerzeń $\Q$, chociaż wtedy trzeba dołożyć po jednej ,,nieskończoności'' za każde włożenie w $\R$ (albo $\C$).
Zajmiemy się tym przypadkiem nieco później.