Rzeczywiści analitycy znają różne sposoby, by zdefiniować potrzebne im funkcje specjalne: mają do dyspozycji rozwijanie w szereg potęgowy, równania różniczkowe, parametryczne całki, równania funkcyjne i inne.
W analizie $p$-adycznej można użyć innej sztuczki: wziąć zwykłą funkcję $f \colon [a, \infty) \to \R$, która przyjmuje tylko wymierne wartości dla całkowitych argumentów i popatrzeć na jej ciągłe przedłużenie do $\Z_p \to \C_p$.
Liczby całkowite leżą gęsto w $\Z_p$, nie ma więc mowy o tym, by przedłużenie nie było jednoznaczne.

Poszukiwanie $p$-adycznych wariantów funkcji określanych normalnie na podzbiorach $\C$ zaczniemy właśnie od logarytmu i eksponensa.

\section{Logarytm ($\Q_p$)}
Klasyczny \prawo{Gouv\\4.5}  logarytm rozwija się w szereg potęgowy wokół $x = 1$:
\begin{equation}
		\label{eq:plogarytm}
	f(x) = \log(1+x) = \sum_{n = 1}^\infty - \frac{(-x)^n}{n}.
\end{equation}

Ze względu na wymierność wszystkich współczynników jest sens myśleć o szeregu, jakby żył w $\Q_p$.
Pierwszym krokiem do jego zrozumienia jest wyznaczenie dysku zbieżności.
Zanim przejdziemy do rachunków, należy podkreślić kontrast między analizą klasyczną i $p$-adyczną: dla nas rosnące mianowniki stanowią utrudnienie, kiedy nie dzielą się przez $p$.

Logarytm $p$-adyczny zbiega ,,gorzej'' niż jego odpowiednik $\log \colon \R_+ \to \R$, ale wciąż dużo lepiej od eksponensa.
Żywimy przy tym nadzieję, że logarytm $p$-adyczny nie zostanie nigdy pomylony ze zwykłym, przy podstawie $p$, rzeczywistym.

\begin{fakt}
	Logarytm jest dobrze określoną funkcją na dysku $1 + p\Z_p = \kula(1,1)$.
\end{fakt}

\begin{proof}
		Skoro $0 \le v_p(n) \le \log_p (n)$ i $|a_n| = p^{v_p(n)}$, wykorzystamy wzór Cauchy'ego-Hadamarda.
		Powie nam, że promieniem zbieżności jest $1$.

		Aby zadecydować, co dzieje się dla $|x| = 1$, wystarczy zauważyć, że $|a_nx^n| = |1/n|$ nie zbiega do zera.
\end{proof}

\begin{definicja}
	Logarytm $p$-adyczny to funkcja $1 + p\Z_p \to \Q_p$ zadana wzorem \ref{eq:plogarytm}.
\end{definicja}

By funkcja $\log_p \colon \mathcal B \to \Q_p$ zasługiwała na bycie logarytmem, musi mieć jego własności, czyli zachowywać się jak homomorfizm.
Tak rzeczywiście jest.

\begin{fakt}
	Dla $a, b \in 1+p\Z_p$ jest $\log_p ab = \log_p a + \log_p b$.
\end{fakt}

\begin{proof}
	W literaturze często dowodzi się tego stwierdzenia przez zauważenie stosownej tożsamości dla szeregów potęgowych.
	Problem w tym, iż sprawdzenie założeń złotego faktu jest naprawdę bolesne, przedstawimy więc bezpośrednie uzasadnienie udające klasyczny dowód.

	Przyjmijmy $f(x) = \log_p(1+x)$ dla $x \in \Z_p$.
	Z naszą wiedzą o pochodnych szeregów potęgowych piszemy
	\[
		f'(x) = \sum_{n \ge 0} (-1)^nx^n = \frac{1}{1+x}.
	\]
	Ustalmy $y \in p\Z_p$ i określmy $g(x) = f(y + (1+y)x)$.
	Jest to szereg potęgowy zbieżny dla $|x| < 1$.
	Reguła łańcucha pozwala policzyć pochodną:
	\[
		g'(x) = (1+y) f'(y + (1+y)x) = \frac{(1+y)}{1+y + (1+y)x}  = \frac{1}{1+x} = f'(x)
	\]
	To oznacza, że $g(x), f(x)$ różnią się o stałą, a skoro $g(0) = f(y)$, przekonujemy się do równości $g(x) = f(x) + f(y)$, którą wystarczy przetłumaczyć na język logarytmów.
\end{proof}

Jeśli $p = 2$, to $-1 \in \mathcal B$, a to umożliwia obliczenie $\log_p(-1)$, $0$, co nie powinno szokować.

Lemat Hensela pozwalał określić, dla jakich $m$ istnieją $m$-te pierwiastki jedności w $\Q_p$ pod warunkiem, że $p \nmid m$.
W $\Q_p$ nie ma takich dla $m = p^n$, co można pokazać w trzech krokach (poza patologicznym przypadkiem $p = 2$ i $n = 1$).

\begin{fakt}
	Logarytm \prawo{Gouvea\\Prob. 161}$p$-adyczny ma dokładnie jedno lub dwa miejsca zerowe: $1$, jeśli $p > 2$, $\pm 1$ w przeciwnym razie.
	Wynika stąd, że pierwiastki $p$-tego (czwartego) stopnia nie istnieją w $\Q_p$.
\end{fakt}

\begin{proof}
	Twierdzenie Strassmana dla $\log(1+px)$.
\end{proof}

\begin{wniosek}
	W $\Q_p$ żyje $\max\{2, p - 1\}$ pierwiastków jedności.
\end{wniosek}