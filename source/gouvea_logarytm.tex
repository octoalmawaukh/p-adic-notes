W analizie nad $\R$ funkcje specjalne można definiować na wiele sposobów, przez rozwinięcia w szereg, równania różniczkowe, całki parametryczne, równania funkcyjne i tak dalej.
%W $\C_p$ do zdefiniowania logarytmu i eksponensa najlepiej użyć szeregu potęgowego.

Inną metodą jest wzięcie zwykłej funkcji $f$ określonej na $[a, \infty) \subset \R$ o wymiernych wartościach dla całkowitych $n \ge a$ i patrzenie na ciągłą funkcję $\Z_p \to \C_p$ przedłużającą $n \mapsto f(n)$.
$\Z \cap [a, \infty)$ jest gęste w $\Z_p$, więc nie może być dwóch przedłużeń.

Poszukiwanie $p$-adycznych wariantów funkcji określanych normalnie na podzbiorach $\C$ zaczniemy właśnie od logarytmu i eksponensa.

\section{Logarytm ($\Q_p$)}
\begin{definicja}
	Logarytm \prawo{Gouv\\4.5} to formalny szereg
	\[
		f(x) = \log(1+x) = \sum_{n = 1}^\infty - \frac{(-x)^n}{n}.
	\]
\end{definicja}

Logarytm zbiega ,,gorzej'' niż funkcja $\log \colon \R_+ \to \R$.

Wypadałoby znać jego promień zbieżności: współczynniki tutaj nie maleją w normie.

\begin{fakt}
	Logarytm ma sens dla $x \in 1 + p\Z_p = \kula(1,1)$.
\end{fakt}

\begin{proof}
Skoro $0 \le \frac 1 n v_p (n) \le \frac 1 n \log_p n \to 0$ i $|a_n| = p^{v_p(n)}$, to $p^{-0} = 1$ jest szukanym promieniem.
Łatwo widać, że nie ma dla $|x| = 1$ zbieżności.
\end{proof}

Mamy nadzieję, że logarytm $p$-adyczny nie zostanie nigdy pomylony ze zwykłym, przy podstawie $p$, rzeczywistym.

By funkcja $\log_p \colon \mathcal B \to \Q_p$ zasługiwała na bycie logarytmem, musi mieć jego własności. Tak rzeczywiście jest.

\begin{fakt}
	Dla $a, b \in 1+p\Z_p$ jest $\log_p ab = \log_p a + \log_p b$.
\end{fakt}

\begin{proof}
	Przyjmijmy $f(x) = \log_p(1+x)$ dla $x \in \Z_p$.
	Z naszą wiedzą o pochodnych szeregów potęgowych piszemy
	\[
		f'(x) = \sum_{n \ge 0} (-1)^nx^n = \frac{1}{1+x}.
	\]
	Ustalmy $y \in p\Z_p$ i określmy $g(x) = f(y + (1+y)x)$.
	Jest to szereg potęgowy zbieżny dla $|x| < 1$.
	Reguła łańcucha pozwala policzyć pochodną:
	\begin{align*}
		g'(x) & = (1+y) f'(y + (1+y)x) = \frac{(1+y)}{1+y + (1+y)x}  = \frac{1}{1+x} = f'(x) \\
		g(x)& = f(x) + C.
	\end{align*}
	%To oznacza, że $g(x), f(x)$ różnią się o stałą.
	Widać, że $g(0) = f(y)$, zatem $g(x) = f(x) + f(y)$, wystarczy przetłumaczyć to na język logarytmów.
\end{proof}

Jeśli $p = 2$, to $-1 \in \mathcal B$, a to umożliwia obliczenie $\log_p(-1)$, $0$, co nie powinno szokować.

Lemat Hensela pozwalał określić, dla jakich $m$ istnieją $m$-te pierwiastki jedności w $\Q_p$ pod warunkiem, że $p \nmid m$.
W $\Q_p$ nie ma takich dla $m = p^n$, co można pokazać w trzech krokach (poza patologicznym przypadkiem $p = 2$ i $n = 1$).

\begin{fakt}
	Logarytm $p$-adyczny ma dokładnie jedno ($x = 1$ jeśli $p > 2$) lub dwa ($x = \pm 1$ dla $p = 2$) miejsca zerowe.
\end{fakt}

\begin{proof}
	Twierdzenie Strassmana dla $\log(1+px)$.
\end{proof}

\begin{wniosek}
	Dla $x \in 1 + p\Z_p$ oraz $p \neq 2$ ($p = 2$), które spełnia $x^p = 1$ ($x^4 = 1$), mamy $x = 1$ ($x = \pm 1$) -- zatem $p$-te (czwarte) pierwiastki jedności w $\Q_p$ nie istnieją.
\end{wniosek}

\begin{wniosek}
	W $\Q_p$ żyje $\max\{2, p - 1\}$ pierwiastków jedności.
\end{wniosek}