\begin{fakt}
	Funkcje $\exp x$, $\cos x$, $\frac 1x\sin x$, $\frac 1 x \log 1 + x$ są tam, gdzie $\exp x$ jest określona, kwadratami funkcji $p$-adycznych. % Vladimirov, 42/337
\end{fakt}

\section{Stałe matemagiczne}
\begin{fakt}
	W żadnym z ciał $\Q_p$ nie ma stałej Nepera.
\end{fakt}

\begin{proof}
Szereg dla $\exp_p(1)$ jest rozbieżny w $\Q_p$, zaś znalezienie pierwiastka dla $\exp_2(4)$ lub $\exp_p(p)$ ($p \ge 3$) wymaga  wyboru (niestety, niekanonicznego).
\end{proof}

\begin{fakt}
	Kosinus nie ma zera, sinus ma tylko jedno ($0$), w dodatku nie istnieje takie $t \in \cialo$, że $\sin (x) = \cos (x + t)$.
\end{fakt}

Nie określimy więc tak odpowiednika $\pi \approx 3,1415926535$.
Warto zwrócić uwagę na punkt piąty faktu \ref{impuls} dla $x = 1/2$.

\begin{fakt}
	Zwykła funkcja $\Gamma$ spełnia $\Gamma (z) \Gamma(1-z) = \pi : \sin \pi z$, a stąd $\Gamma (1/2) = \sqrt{\pi}$.
\end{fakt}

Zatem odpowiednikiem $\pi$ w $\Q_p$ powinna być $\Gamma_p(1/2)^2$, $-1$ (w $\Q_{4k+1}$) lub $1$ (w $\Q_{4k+3}$).

%Funkcja $\Gamma_p$ jest lokalnie analityczna.
Różniczkując $\Gamma_p(x + 1) = \Gamma_p(x) h_p(x)$ dostaniemy
\[
	\frac{\Gamma_p'(x+1)}{\Gamma_p(x+1)} - \frac{\Gamma_p'(x)}{\Gamma_p(x)} = \frac{h_p'(x)}{h_p(x)}.
\]

Istnieje stała $c$, że dla każdego $x \in \Z_p$ jest
\[
	\frac{\Gamma_p'(x)}{\Gamma_p(x)} = c + L_p(x),
\]
gdzie $L_p(x)$ to nieoznaczona suma poprzedniej prawej strony.
Po wstawieniu do wzoru $x = 1$ mamy (dla $m \in \N$):
\[
	\frac{\Gamma_p'(m)}{\Gamma_p(m)} = \frac{\Gamma_p'(1)}{\Gamma_p(1)} + \sum_{j < m}' \frac 1 j.
\]

To przypomina wzór, który jest prawdziwy też dla zwykłej funkcji $\Gamma$ i zachęca do zdefiniowania nowej stałej.

\begin{definicja}
	Stała ($p$-adyczna) Eulera to $\gamma_p := - \Gamma_p'(0)$.
\end{definicja}

Schikhof nie wie nic na temat jej (nie)wymierności.

\begin{fakt}
	$|\gamma_p|_p \le 1$, a w $\Q_p$
	\[
		\gamma_p = \lim_{n \to \infty} \frac 1 {p^n} \left[1 - \frac{(-1)^p p^n!}{p^{n-1}! p^{p^n-1}} \right].
	\]
\end{fakt}