\section{Łatanie podziurawionych ciał}
Przypomnienie: $\R$ jest uzupełnieniem $\Q$, to znaczy norma $|\cdot|_\infty$ przedłuża się na $\R$, $\R$ jest zupełne z metryką od niej i $\Q$ leży gęsto w $\R$. Uzupełnianie jest konieczne, gdyż

\begin{lemat}
	Ciało $\Q$ z nietrywialną normą nie jest zupełne.
\end{lemat}

\begin{proof}
	Dzięki twierdzeniu Ostrowskiego wystarczy sprawdzić $p$-adyczne normy.
	Niech $p \neq 2$ będzie pierwsza, zaś $y \in \Z$ taka, że nie jest kwadratem, nie dzieli się przez $p$ i równanie $x^2 = y$ ma rozwiązanie w $\Z/p\Z$.
	Stosowne $y$ zawsze istnieje: wystarczy powiększyć jakiś kwadrat z $\Z$ o krotność $p$.
	
	Niech $y_0$ będzie dowolnym rozwiązaniem równania, $y_n$ ma być równe $x_{n-1}$ modulo $p^n$ oraz $y_n^2 = y$ (modulo $p^{n+1}$).
	Tak skonstruowany ciąg Cauchy'ego nie ma granicy, oto stosowne rachunki:
	\begin{align*}
		y_n & = y_{n-1} + \lambda_n p^n \\
		y_n^2 & = y_{n-1}^2 + 2 y_{n-1} \lambda_n p^n + \lambda_n^2 p^{2n} \\
		\lambda_n & = (y - y_{n-1}^2)(2y_{n-1} p^n)^{-1} \pmod p
	\end{align*}
	
	Jest Cauchy'ego ($|y_{n+1} - y_n| \le p^{-n-1}$) i nie ma granicy ($|y_n^2 - y| \le p^{-n-1}$, ale pierwiastek z $y$, jedyny kandydat, nie istnieje).
	Gdy $p = 2$, to zastępujemy pierwiastek kwadratowy sześciennym.
\end{proof}

Zbiór ciągów Cauchy'ego oznaczmy przez $C$.
Można na nim zadać strukturę pierścienia (przemiennego i z jedynką) przez punktowe dodawanie oraz mnożenie.
Wprowadzamy ideał $N$, do którego należą ciągi zbieżne do zera.

\begin{lemat}
	Ideał $N \trianglelefteq C$ jest maksymalny.
\end{lemat}

\begin{proof}
Ustalmy ciąg $(x_n) \in C \setminus N$ oraz ideał $I = \langle (x_n), N \rangle$.
Od pewnego miejsca $x_n$ nie jest zerem, zatem $y_n = 1/x_n$ od tego miejsca, $0$ wcześniej ma sens.
Ciąg $y_n$ jest Cauchy'ego:
\[
	|y_{n+1} - y_n| = \frac{|x_{n+1} - x_n|}{|x_nx_{n+1}|} \le \frac{|x_{n+1}-x_n|}{\varepsilon^2} \to 0.
\]
Ale $(1) - (x_n)(y_n) \in N$, więc $I = C$.
\end{proof}

\begin{definicja}
	Ciało liczb $p$-adycznych to $\Q_p := C / N$.
\end{definicja}

\begin{lemat}
	Ciąg $|x_n|_p$ jest stacjonarny, gdy $(x_n) \in C \setminus N$.
\end{lemat}

\begin{proof}
	Można znaleźć takie liczby $\varepsilon, N_1$, że $n \ge N_1$ pociąga $|x_n| \ge \varepsilon > 0$.
	Z drugiej strony istnieje taka $N_2$, że $n, m \ge N_2$ pociąga $|x_n - x_m| < \varepsilon$.
	Połóżmy więc $N = \max\{N_1, N_2\}$.
	Wtedy $n, m \ge N$ pociąga $|x_n - x_m| < \max\{|x_n|, |x_m|\}$, a to oznacza, że $|x_n| = |x_m|$.
\end{proof}

Dzięki temu następująca definicja nie jest bez sensu:

\begin{definicja}
	Gdy $(x_n) \in C$ reprezentuje $x \in \Q_p$, przyjmujemy $|x|_p := \lim_{n \to \infty} |x_n|_p$.
\end{definicja}

\begin{lemat}
	Obraz $\Q \hookrightarrow \Q_p$ po włożeniu jest gęsty.
\end{lemat}

\begin{proof}
	Chcemy pokazać, że każda otwarta kula wokół $x \in \Q_p$ kroi się z obrazem $\Q$, czyli zawiera ,,stały ciąg''.
	Ustalmy kulę $\kula(x, \varepsilon)$, ciąg Cauchy'ego $(x_n)$ dla $x$ i $\varepsilon' < \varepsilon$.
	Dzięki temu, że ciąg jest Cauchy'ego, możemy znaleźć dla niego indeks $N$, że $n, m \ge N$ pociąga $|x_n - x_m| < \varepsilon'$.
	Rozpatrzmy stały ciąg $(y)$ dla $y = x_N$.
	Wtedy $|x - (y)| < \varepsilon$, gdyż $x - (y)$ odpowiada ciąg $(x_n-y)$.
	Ale $|x_n - x_N| < \varepsilon'$ i $\lim_{n \to \infty}|x_n - y| \le \varepsilon' < \varepsilon$.
\end{proof}

\begin{fakt}
	Ciało $\Q_p$ jest zupełne.
\end{fakt}

\begin{proof} 
	Ustalmy $x_n$, ciąg Cauchy'ego elementów $\Q_p$.
	Obraz $\Q$ w $\Q_p$ jest gęsty, a zatem można znaleźć liczby wymierne $q_n$, że $|x_n - (q_n)| \to 0$ (w ciele $\Q_p$).
	Okazuje się, że liczby $q_n$ same tworzą ciąg Cauchy'ego i to właśnie on jest granicą $x_n$.
\end{proof}

\begin{fakt}
	Własności pierścienia waluacji $\{x \in \Q_p : |x|_p \le 1\}$:
	\begin{enumx}
		\item pierścień ,,$\Z_p$'' jest lokalny; ideał $p \Z_p$ jest maksymalny
		\item $\Q \cap \Z_p = \Z_{(p)} = \{\frac yz \in \Q: p \nmid z\}$
		\item włożenie $\Z \hookrightarrow \Z_p$ ma gęsty obraz: jeśli $x \in \Z_p$ i $n \ge 1$, to istnieje jedyna $x_n \in \Z \cap [0, p^n- 1]$, że $|x-x_n| \le p^{-n}$.
		\item każdy $x \in \Z_p$ jest granicą ciągu Cauchy'ego $x_n \in \Z$, którego wyrazy spełniają $0 \le x_n \le p^n-1$, $p^{n-1} \mid (x_n - x_{n-1})$.
	\end{enumx}
\end{fakt}

\begin{proof}
	Pierścień $\Z_p$ jest lokalny, jak inne pierścienie waluacji.
	Ideał waluacji ma $p$ za generator, bo $|x| < 1$ wtedy i tylko wtedy gdy $|x/p| \le 1$, czyli $x \in p\Z_p$.
	Ideał waluacji zawiera się w $p\Z_p$ i jest maksymalny, czyli jest nim po prostu $\Z_p$.

	Niech $x \in \Z_p$, $n \ge 1$.
	Wskażmy $\frac yz \in \Q$, że $|x-\frac yz| \le p^{-n}$.
	Skoro $|y/z| \le \max (|x|, |x-y/z|) \le 1$ (czyli $p \nmid z$), to istnieje $z' \in \Z$, że $zz' \equiv 1$ mod $p^n$.
	To oznacza, że $|y/z-yz'| \le p^{-n}$ i $yz' \in \Z$.
	Zastąpiliśmy ułamek liczbą całkowitą.

	Wybierając $x_n$, jedyną całkowitą, że $0 \le x_n \le p^n-1$ i $x_n = yz'$ modulo $p^n$, dostajemy $|x - x_n| \le p^{-n}$.
	Ostatni punkt wynika z przedostatniego.
\end{proof}

\begin{wniosek}
	Zbiory $p^n\Z_p$ to układ otoczeń dla zera kryjący $\Q_p = \Z_p[1/p]$ ($n \in \Z$).
	Ciąg $0 \to \Z_p \to \Z_p \to \Z/p^n\Z \to 0$ (najpierw mnożymy przez $p^n$, później rzutujemy) jest dokładny, a strzałki ciągłe, więc $\Z_p^+$ jest beztorsyjna i ${\Z_p}/{p^n \Z_p} \cong {\Z}/{p^n \Z}$.
\end{wniosek}
