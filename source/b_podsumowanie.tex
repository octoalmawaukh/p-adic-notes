\section{Preludium (arytmetyka)}
Pojęcie (oraz sama nazwa) \emph{waluacji} pochodzi z pracy Kürschaka z 1913, stanowi ono uogólnienie \emph{wartości bezwzględnej}.
Przy ich użyciu pokazał, że każde ciało z waluacją można rozszerzyć do ciała zupełnego oraz algebraicznie domkniętego; zainspirowała go książka Hensela z 1908 (przykład: $\Q_p$ i $\C_p$).
Świat dowiedział się o twierdzeniu Ostrowskiego w roku 1918, chociaż Ostrowski potrafił je uzasadnić kwietniem dwa lata wcześniej.
My przedstawiliśmy dowód Artina (1932).
Lemat Hensela (przedstawiony później w dużo większej ogólności) jest wnioskiem z lematu Hensela-Rychlika, o którym nikt nie pamięta.
Reguła lokalno-globalna (dla form kwadratowych) pojawia się w rozprawie doktorskiej Hassego z 1921 roku.
Hasse rok wcześniej przeniósł się do Marburga z Göttingen po tym, jak odkrył w antykwariacie książkę ,,Zahlentheorie'' Hensela z 1913 roku.
Minkowski zmarł w 1909 r., jednak podstawy geometrii liczb wyłożył trzynaście lat wcześniej.

\section{Analiza}
Większość z uzyskanych w tym rozdziale wyników jest klasyczna i nie wymaga dodatkowego komentarza.
Zrozumienie przykładu \label{leniwy} wymaga poznania eksponensy Artina-Hassego.
Problem, czy liczba $\sum_n n!$ jest wymierna, czy nie, pochodzi jeszcze od Schikhofa, zatem pozostaje otwarty od ponad trzech dekad.
Twierdzenie Straßmanna pojawia się w jego pracy z 1928 roku, ,,Über den Wertevorrat von Potenzreihen im Gebiet der $\mathfrak p$-adischen Zahlen''.
Formułę pozwalającą na wyznaczenie waluacji $p$-adycznej silnii znał już A. Legendre w 1830 (,,Théorie des Nombres'').
Dyskusja szeregów dwumiennych podąża za Koblitzem.
Konstrukcja dziwnie zbieżnego szeregu różni się od pracy Bürgera i Struppecka jedynie notacją.

Sztywne przestrzenie analityczne są odpowiednikiem zespolonych, ale nad niearchimedesowym ciałem.
Wprowadził je J. Tate w roku 1962 jako efekt uboczny prac nad ujednoliceniem $p$-adycznych krzywych eliptycznych o złej redukcji.
Ich zaletą jest to, że sens mają dla nich analityczne przedłużanie i spójność, ale ceną za to jest pojęciowa złożoność.