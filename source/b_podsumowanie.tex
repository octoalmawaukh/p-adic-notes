\begin{enumx}
\item Preludium (arytmetyka)
\begin{enumx}
	\item Fernando Q. Gouvea -- $p$-adic numbers, an introduction (od 2.1 do 3.5)
	\item Alain M. Robert -- A course in $p$-adic analysis (1.6.7)
	\item John W. S. Cassels -- Local fields (2.3)
\end{enumx}
\item{Analiza}
\begin{enumx}
	\item Edward B. Burger, Thomas Struppeck -- Does $\sum_n 1/n!$ really converge? 
	\item Fernando Q. Gouvea -- $p$-adic numbers, an introduction (od 4.1 do 4.4),
	\item Helmut Hasse -- Number theory (17),
	\item Neal Koblitz -- $p$-adic numbers, analysis and $\zeta$-functions (4.1),
	\item Wim Schikhof -- Ultrametric calculus (3, 7, 23), 
	\item Ram Murty, Sarah Sumner -- on the $p$-adic series $\sum_{n \ge 1} n^k \cdot n!$
	%\item Alain M. Robert -- A course in $p$-adic analysis (1.6.7)
	%\item John W. S. Cassels -- Local fields (2.3)
\end{enumx}
\item{Analiza z plusem}
Niezweryfikowano.
\item{Topologia}
\begin{enumx}
	\item \end{enumx}
\item{Algebra}
\begin{enumx}
	\item Jean-Pierre Serre -- Local fields (2.6)
	\item Keith Conrad -- The character group of $\Q$, 
	\item Daniel Finkel -- An overview of Witt vectors
	\item Fernando Q. Gouvea -- $p$-adic numbers, an introduction (4.5),
	\item Alain M. Robert -- A course in $p$-adic analysis (1.1, 1.4, 1.5, 2.1, 3.2)
	%\item Wim Schikhof -- Ultrametric calculus (3, 7, 23), 
	%\item Neal Koblitz -- $p$-adic numbers, analysis and $\zeta$-functions (4.1),
	%\item John W. S. Cassels -- Local fields (2.3)
\end{enumx}
\item{Rozszerzenia ciał}
\begin{enumx}
	\item Fernando Q. Gouvea -- $p$-adic numbers, an introduction (5),
	\item Alain M. Robert -- A course in $p$-adic analysis (3.2, 3.A)
	%\item Wim Schikhof -- Ultrametric calculus (3, 7, 23), 
	%\item Neal Koblitz -- $p$-adic numbers, analysis and $\zeta$-functions (4.1),
	%\item John W. S. Cassels -- Local fields (2.3)
\end{enumx}
\item{Funkcje specjalne}
\begin{enumx}
	\item Fernando Q. Gouvea -- $p$-adic numbers, an introduction (4.5),
	%\item Alain M. Robert -- A course in $p$-adic analysis (1.1, 1.4, 1.5, 2.1, 3.2)
	%\item Wim Schikhof -- Ultrametric calculus (3, 7, 23), 
	\item Neal Koblitz -- $p$-adic numbers, analysis and $\zeta$-functions (4.1)
	%\item John W. S. Cassels -- Local fields (2.3)
\end{enumx}
\end{enumx}





%Większość z uzyskanych w tym rozdziale wyników jest klasyczna i nie wymaga dodatkowego komentarza.
%Zrozumienie przykładu \label{leniwy} wymaga poznania eksponensy Artina-Hassego.
%Problem, czy liczba $\sum_n n!$ jest wymierna, czy nie, pochodzi jeszcze od Schikhofa, zatem pozostaje otwarty od ponad trzech dekad.
%Twierdzenie Straßmanna pojawia się w jego pracy z 1928 roku, ,,Über den Wertevorrat von Potenzreihen im Gebiet der $\mathfrak p$-adischen Zahlen''.
%Formułę pozwalającą na wyznaczenie waluacji $p$-adycznej silnii znał już A. Legendre w 1830 (,,Théorie des Nombres'').
%Dyskusja szeregów dwumiennych podąża za Koblitzem.
%Konstrukcja dziwnie zbieżnego szeregu różni się od pracy Bürgera i Struppecka jedynie notacją.

%Sztywne przestrzenie analityczne są odpowiednikiem zespolonych, ale nad niearchimedesowym ciałem.
%Wprowadził je J. Tate w roku 1962 jako efekt uboczny prac nad ujednoliceniem $p$-adycznych krzywych eliptycznych o złej redukcji.
%Ich zaletą jest to, że sens mają dla nich analityczne przedłużanie i spójność, ale ceną za to jest pojęciowa złożoność.