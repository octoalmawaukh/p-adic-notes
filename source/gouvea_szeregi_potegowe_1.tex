\section{Szeregi potęgowe}
Szeregi \prawo{Gouvea\\4.3} potęgowe pozwalają wygodnie opisywać funkcje trygonometryczne albo eksponens.
Tak jak można przypuszczać, teoria $p$-adyczna będzie nieco bardziej zakręcona.

Największą niespodzianką jest zapewne nieregularne zachowanie złożeń, poświęcimy jej zatem najwięcej uwagi.
Później zdefiniujemy logarytm oraz funkcję wykładniczą pod wpływem Casselsa, Hassego oraz Conrada.
Będziemy rozważać szeregi postaci $\sum_n a_n x^n$, nigdy nie odróżniając $x$ od $X$ (jako wyrażenia formalnego).

Z analitycznego punktu widzenia wyrażenie $f(x)$ ma sens, o ile $|a_n x^n|$ zbiega do zera.
Zbiór punktów, dla których tak właśnie jest, nazywamy dyskiem (obszarem) zbieżności.

\begin{fakt}
	Szereg $\sum_n a_n x^n$ zbiega na różnych dyskach, których promień zależy od $R$, odwrotności $\limsup |a_n|^{1:n}$.
	\begin{enumx}
	\item jeśli $R = 0$, to $f$ zbiega tylko w $x = 0$.
	\item jeśli $R = \infty$, to $f$ zbiega wszędzie na $\Q_p$.
	\item jeśli $R > 0$ i $\lim_{n \to \infty} |a_n| R^n = 0$, to $f$ zbiega dla $|x| \le R$.
	\item w przeciwnym przypadku $f$ zbiega dokładnie dla $|x| < R$.
	\end{enumx}
\end{fakt}

\begin{proof}
	Chcemy opisać kształt zbioru $\{x \in \Q_p : \lim_{n \to \infty} |a_n x^n| = 0\}$.
	Oczywiście wartość $f(0)$ istnieje.
	Jeśli $|x| < R $, to (rzeczywisty) szereg potęgowy $\sum_n |a_n| |x|^n$ jest zbieżny.
	Jeśli zaś $|x| > R $, to $|a_n| |x|^n$ nie może zbiegać do zera przy $n$ uciekającym do nieskończoności: wtedy $|a_n|$ jest nieskończenie często blisko $R ^{-n}$, więc $(|x|/R )^n$ może być dowolnie duże.
	
	Przypadek $|x| = R $ jest konsekwencją faktu \ref{ingentis}.
\end{proof}

Szeregi $p$-adyczne szeregi zachowują się porządniej niż ich zespoleni koledzy.
Tam zbieżność na brzegu dysku $\{|x| = R \}$ jest nieprzewidywalna, tutaj brzegu po prostu nie ma.
Formalne szeregi potęgowe można dodawać i mnożyć, jak w fakcie \ref{ludmori}.

\begin{fakt}\label{decoris}
	Jeżeli szeregi potęgowe $f, g$ nad $\Q_p$ zbiegają w punkcie $x$, to $f+g$ oraz $fg$ również, odpowiednio do $f(x) + g(x)$ i $f(x)g(x)$.
\end{fakt}

Przyjrzymy się formalnym złożeniom, które (o dziwo) zachowują się zaskakująco często gorzej niż źle.
Będziemy więc pracować z szeregami: $f(x) = \sum_n a_n x^n$ i $g(x) = \sum_n b_n x^n$, przy czym $b_0 = 0$, by napis $f(g(x))$ miał sens (niezależnie od topologii).
Przez formalne złożenie rozumiemy
\[
	h(x) = (f \circ g)(x) = \sum_{n = 0}^\infty a_n g(x)^n = \sum_{n = 0}^\infty c_n x^n.
\]

Współczynniki $c_n$ są jawnie opisane przez wielomiany Bella, ale te akurat nie będą dla nas przesadnie przydatne.

\begin{fakt}[złoty] \label{auctoris}
	Niech $f(x) = \sum_n a_n x^n$, $g(x) = \sum_n b_n x^n$ będą formalnymi szeregami potęgowymi, $g(0) = 0$ i $h = f \circ g$ (jako formalne złożenie).
	Załóżmy, że $g(x_0)$ oraz $f(g(x_0))$ są zbieżne, zaś dla każdego $n$ mamy $|b_n x_0^n| \le |g(x_0)|$.
	Wtedy $h(x_0)$ także zbiega i przyjmuje wartość $f(g(x_0))$.
\end{fakt}

\begin{proof}
	Podamy dowód za książką Hassego.
	Niech
	\[
		g(x)^m = \sum_{n=m}^\infty d_{m, n} x^n = \sum_{n=m}^\infty \sum_{*} \prod_{k=1}^m b(i_k) x^n,
	\]

	gdzie wewnętrzna suma rozciąga się na te indeksy, dla których $i_1 + \ldots + i_m$ jest równe $n$.
	Pozwala to na napisanie jawnego wzoru dla $h(x)$:
	\[
		h(x) = a_0 + \sum_{n = 1}^\infty \sum_{m = 1}^n a_m d_{m, n} x^n.
	\]

	Pomyślmy o zbieżności.
	Szereg $g(x_0)$ jest zbieżny, więc fakt \ref{decoris} pozwala powiedzieć, że $g(x)^m$ zbiega do $g(x_0)^m$ po odpowiednim podstawieniu (jeden szereg jest formalny, drugi nie!).
	Co ciekawsze, dla każdego $n$ mamy $|d_{m,n}x^n| \le |g(x)^m|$.
	Jeżeli $n \ge m$, to nierówność ultrametryczna daje
	\[
		|d_{m,n}x^n| \le \max_{\mbox{,,}i\mbox{''}} \prod_{k \le m} |b_{i_k} x^{i_k}| \le \prod_{k \le m} |g(x)| = |g(x)^m|,
	\]
	kiedy $i_1 + \ldots + i_m = n$ (dzięki $|b_{ij} x^{ij}| \le |g(x)^m|$).
	Jeżeli $n < m$, to nie ma czego dowodzić: $d_{m,n}x^n = 0$.
	Wiemy już, że $g(x)$, $g(x)^m$ oraz $f(g(x))$ zbiegają.
	Zapiszmy w takim razie
	\begin{align*}
	 	f(g(x))	& = a_0 + \sum_{m \ge 1} \sum_{n \ge m} a_m d_{m,n}x^n,\\
		h(x) & = a_0 + \sum_{n \ge 1} \sum_{m \ge 1} a_md_{m,n} x^n.
	\end{align*}

	Aby uzasadnić poprawność zamiany kolejności sumowania powołamy się na fakt \ref{caedis} i oszacujemy $a_md_{m,n}x^n$.

	Wiemy przede wszystkim, że $|a_md_{m,n}x^n| \le |a_mg(x)^m|$: prawa strona nie zależy od $n$.
	Ustalmy $\varepsilon > 0$.
	Możemy wybrać indeks $N$, taki że $m \ge N$ pociąga $|a_mg(x)^m| < \varepsilon$.
	To pokazuje, że $a_md_{m,n}x^n \to_m 0$ jednostajnie względem $n$.

	Z drugiej strony, dla każdego $m$ szereg $g(x)^m$ jest zbieżny, zatem jego wyraz ogólny zbiega do zera: $a_m d_{m,n}x^n \to 0$.
\end{proof}

Leniwi mogą nie sprawdzić założeń i nadepnąć na minę.
Zjawisko to świetnie ilustruje poniższy przykład.
To intrygujące, jak zwykła analiza łatwiej radzi sobie z tym problemem: jeśli promieniem zbieżności $f(x)$ jest $R $ i $|g(x)| < R $, to $h(x)$ zbiega do $f(g(x))$.

\begin{przyklad}\label{leniwy}
	Niech $g(x) = 2x^2 - 2x$ i $h = \exp_2 \circ g$.
	Funkcja $g$ zbiega wszędzie, jako że jest wielomianem.
	Pokażemy później, iż dyskiem zbieżności dla $\exp_2$ jest dokładnie $4\Z_2$.
	Oczywiście $f(g(1)) = 1$.
	Niech $h(x) = \sum_n a_n x^n$.
	Jeżeli $n \ge 2$, to $v_2(a_n)$ wynosi co najmniej $1 + n / 4$, czyli $h$ zbiega na $\Z_2$.
	Niestety, $h(1) \equiv 3 \pmod {4}$ i $h(1) \neq f(g(1))$.
\end{przyklad}

\begin{fakt}
	Formalna pochodna spełnia regułę Leibniza $\mathfrak D[fg] = \mathfrak D[f]g + f \mathfrak D[g]$, a także regułę łańcuchową: $\mathfrak D[f \circ g](x) = \mathfrak D [f](g(x)) \mathfrak D [g](x)$.
\end{fakt}