\section{Zbiór Cantora jako pierścień $\Z_p$}
Liczby \prawo{Rbrt\\1.2} $p$-adyczne to formalne szeregi $\sum_{i=0}^\infty a_i p^i$, o współczynnikach $a_i \in [0, p) \cap \Z$.
Takie szeregi można utożsamiać z ciągiem $(a_i)$.
Kartezjański produkt $X_p = \{0, \dots, p-1\}^\N$ ma topologię produktową (czynniki są dyskretne), zatem jest zwarty.
Jest też całkowicie rozłączny.
Topologia na $X_p$ pochodzi od wielu metryk, na przykład \[d(x, y) = \sup_{i \ge 0} \delta(x_i, y_i) p^{-i} = p^{-v_p(x-y)}\]
albo $\sum_{i \ge 0} \delta(x_i,y_i)p^{-i-1}$, ale nie tylko.
Wszystkie metryka na zwartej p. metryzowalnej są jednostajnie równoważne, ale my będziemy sobie cenić te, które dają wierny obraz warstwowej struktury $\Z_p$: dla każdej $k \in \N$, warstwy $p^k \Z_p$ mają być izometryczne.

\begin{fakt}
	Z $p$-adyczną metryką mnożenie jest kontrakcją, zatem ciągłe: $d(px, py) = \frac 1p d(x, y)$.
\end{fakt}

Zbiór Cantora $C \subseteq [0,1]$ składa się z punktów, w których trójkowych rozwinięciach nie znajdziemy cyfr(y) jeden.
Jest homeomorficzny z $\Z_2$ za sprawą ciągłej bijekcji $\psi \colon \Z_2 \to C$, $\sum_i a_i2^i \mapsto \sum_i 2a_i/3^{i+1}$ między zwartymi zbiorami.

\begin{fakt}
	Całkowicie \prawo{Engel} niespójna, zwarta p. metryczna bez izolatorów jest homeomorficzna z $\Z_2$.
	Zwarta przestrzeń metryczna jest obrazem $\Z_2$.
\end{fakt}

W przestrzeni $\Z_2$ nie ma zatem nic niezwykłego.

Ciekawa jest też funkcja $\varphi \colon \Z_2 \to [0,1]$ zadana przez $\sum_i a_i2^i \mapsto \sum_i a_i/2^{1+i}$.
Diagram przemienny $\varphi = g \circ \psi$ (funkcja $g$ zszywa krańcowe punkty zbioru Cantora) zachęca nas do rozważenia liniowych modeli $\Z_p$.

\begin{fakt}
	Funkcje $\psi_b \colon \Z_p \to [0,1]$ określone dla $b > 1$ są ciągłe.
	Dla $b > p$, są injektywnym homeomorfizmem na obraz.
	Kiedy $b = p$, $\psi_b$ jest tylko nieróżnowartościową surjekcją: \[\psi_b \Bigl(\sum_{i = 0}^\infty a_i p^i\Bigr) = \frac {b - 1 }{p-1} \sum_{i = 0}^\infty \frac{a_i}{b^{i+1}}.\]
\end{fakt}
