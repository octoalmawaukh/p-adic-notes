\section{Dołączanie $p$-tego pierwiastka}
Wcześniejsze \prawo{Gouvea\\5.6} osiągnięcia teoretyczne tylko czekają, by użyć ich do czegoś konkretnego.
Rozpatrujemy ciało $\cialo = \Q_p(\zeta)$, gdzie $\zeta$ to $p$-ty pierwiastek jedności, zaś $p$ nie jest dwójką.
Przypadek $p = 2$ jest, delikatnie mówiąc, trywialny.
Zatem $\zeta$ zeruje
\[
	\Phi_p(X) = \frac{X^p-1}{X-1} = \sum_{k=0}^{p-1} X^k,
\]
,,$p$-ty wielomian cyklotomiczny''.

\begin{lemat}
	Wielomian $\Phi_p(X)$ jest nierozkładalny nad $\Q_p$.
\end{lemat}

\begin{proof}
	Niech $F(X) = \Phi_p(X+1)$. Jest on nierozkładalny tak samo jak $\Phi_p(X)$; sprawdzimy założenia kryterium Eisensteina.
	Mamy
	\[
		F(X)  = \frac{(X+1)^p - 1}{X} = \frac{X^p + 1 - 1}{X} \stackrel {p} \equiv X^{p-1},
	\]
	więc wszystkie (poza pierwszym) współczynniki $F(X)$ dzielą się przez $p$.

	Ostatni współczynnik to $F(0) = \Phi_p(1) = p$ i z pewnością nie dzieli się przez $p^2$.
\end{proof}

Możemy stąd wywnioskować kilka rzeczy.
\begin{enumx}
	\item $\cialo = \Q_p(\zeta)$ jest rozszerzeniem $\Q_p$ stopnia $p-1$.
	\item $\mathfrak N_{\cialo/ \Q_p}(\zeta) = 1$, więc $|\zeta| = 1$.
	\item Wielomian $F(X) = \Phi_p(X+1)$ jest minimalny dla $\zeta - 1$, zatem $\mathfrak N_{\cialo/\Q_p}(\zeta - 1) = p$ i $|\zeta-1| = p^{1/(1-p)}$.
	\item $\cialo$ jest całkowicie rozgałęzione, z jednolitością $\pi = \zeta- 1$.
	\item $\zeta \equiv 1 \pmod \pi$; tzn. $\zeta$ jest $1$-jednością w $\mathcal O_{\cialo}$.
	\item $\Z_p[\zeta] \subseteq \mathcal O_{\cialo}$.
\end{enumx}

Skoro $\cialo$ jest całkowicie rozgałęzione, to $e = p-1$, $f = 1$ i ciało residuów $\mathcal O_{\cialo} / \pi \mathcal O_{\cialo}$ dla $\cialo$ to $\mathbb F_p$.
Wybieramy liczby $0, 1, \dots, p-1$ jako reprezentantów warstw.
Wynika stąd, że elementy $\cialo$ mają $\pi$-adyczne rozwinięcia postaci
\[
	a_{-n}\pi^{-n} + a_{-n+1}\pi^{-n+1} + \ldots + a_0 + a_1 \pi + \ldots,
\]
gdzie $a_i \in [0, p-1] \cap \Z$.
Jest tylko jeden mały kłopot: jak z $p$-adycznego rozwinięcia $x \in \Q_p$ uzyskać rozwinięcie $\pi$-adyczne?
Już $x = p$ zapewnia koszmarne rachunki.

\begin{fakt}
	Tak naprawdę $\Z_p[\zeta] = \mathcal O_{\cialo}$.
\end{fakt}

\begin{proof}
	Pokazaliśmy kiedyś, że elementy $\mathcal O_{\cialo}$ to $\Z_p$-liniowe kombinacje $\pi^{l}\alpha_i$ dla $0 \le l < e$ oraz $1 \le i \le f$, gdzie $\alpha_i$ to elementy $\mathcal O_{\cialo}$, które redukują się do bazy dla $\mathfrak K$ nad $\mathbb F_p$.

	W naszym przypadku $f = 1$, więc wystarcza nam $\alpha_1 = 1$, a przy tym $e = p-1$.
	Przypomnijmy sobie, że $\pi = \zeta - 1$, to koniec.
\end{proof}

A teraz niespodzianka, własne uogólnienie dla $\zeta = 2$.

\begin{fakt}
	$\sum_{n \ge 1} (1 - \zeta)^n : n = 0$.
\end{fakt}

\begin{proof}
	Skoro $|\zeta - 1| < 1$, szereg dla logarytmu zbiega.
	Z drugiej strony $\zeta^p = 1$, więc $p \log_p \zeta = \log_p 1 = 0$, co można zapisać w postaci
	\[
		\sum_{n = 1}^\infty (-1)^{n+1} \frac{(\zeta - 1)^n} n = 0. \qedhere
	\]
\end{proof}

Co jeszcze dziwniejsze, w $\mathcal O_{\cialo}$ można doszukać się takiego $\pi_1$, że $\pi_1^{p - 1} + p = 0$.
Jest to możliwe dzięki współpracy algebry z analizą.