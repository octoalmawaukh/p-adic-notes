\section{Teoria reprezentacji Blondela}
Teoria reprezentacji w tym rozdziale wyłożona jest w naprawdę telegraficznym skrócie.
Tak długo, jak to możliwe, będziemy pracować z przestrzenią wektorową nad ciałem, zakładając, że nie ma charakterystyki $p$.
Naszym ,,węgielnym kamieniem'' będzie następujące stwierdzenie: gładkie oraz nierozkładalne reprezentacje $p$-adycznej grupy reduktywnej nad ciałem, które ma charakterystykę różną od $p$, są osiągalne.

% \section{Gładkie reprezentacje}
% \subsection{Grupy lokalnie proskończone}
\begin{definicja}
	Grupa topologiczna $\grupa$ jest {proskończona}, gdy jest zwarta i całkowicie niespójna.
\end{definicja}

\begin{fakt}
	Proskończone grupy topologiczne $\grupa$ są izomorficzne z $\varprojlim \grupa / U$, gdzie $U$ przebiega otwarte podgrupy normalne.
\end{fakt}

\begin{fakt}
	Jeśli ilorazy $\grupa / U$ są $p$-grupami, to granica odwrotna również nią jest.
\end{fakt}

\begin{przyklad}
	$\Z_p = \varprojlim \Z / p^n \Z$.
\end{przyklad}

\begin{definicja}
	Grupa topologiczna $\grupa$ jest {lokalnie proskończona}, gdy każde otoczenie $e \in G$ zawiera otwartą i zwartą podgrupę
\end{definicja}

\begin{fakt}
	 Lokalna proskończoność $\grupa$ jest równoważna lokalnej zwartości i całkowitej niespójności.
\end{fakt}

\begin{przyklad}
	$(\Q_p, +)$.
\end{przyklad}

% \subsection{Podstawy teorii}

\begin{definicja}
	{Reprezentacja} $(\pi, \liniowa)$ grupy $\grupa$ w przestrzeni $\liniowa$ nad ciałem $\cialo$ to homomorfizm $\pi \colon G \to \textrm{Aut}_{\cialo}(\liniowa)$.
\end{definicja}

\begin{definicja}
	Niezerowa reprezentacja jest {nieredukowalna}, gdy właściwe podprzestrzenie $\liniowa$ są niestabilne względem automorfizmów $\pi(G)$.
\end{definicja}

\begin{definicja}
	Reprezentacja jest {nierozkładalna}, jeśli $\liniowa$ nie da się rozłożyć jako sumę prostą właściwych podprzestrzeni, które same są stabilne względem tych samych automorfizmów $\pi(G)$.
\end{definicja}

% \subsection{Reprezentacje gładkie}
\begin{definicja}
	Reprezentacja $(\pi, \liniowa)$ dla $\grupa$ jest {gładka}, jeżeli wszystkie zbiory postaci $G_v = \{g \in \grupa : \pi(g)(v) = v\} \subseteq \grupa$ dla $v$, stabilizatory w $\grupa$, są otwartymi podzbiorami $\grupa$.
\end{definicja}

% \subsection{Reprezentacje indukowane}

% \section{Reprezentacje osiągalne}
\begin{definicja}
	Gładka reprezentacja $(\pi, \liniowa)$ dla $\grupa$ jest {osiągalna}, o ile $\liniowa^K$ ma skończony wymiar dla każdej zwartej podgrupy otwartej $K \le \grupa$.
\end{definicja}

\begin{enumerate}
\item Miara Haara. 
\item Algebra Heckego. 
\item Koniezmienniki. 
\item Charaktery. 
\item Lemat Schura, charakter centralny. 
\item Reprezentacje $Z$-zwarte. 
\item Paraboliczna indukcja, restrykcja, para.
\item Rozkład Iwahoriego. 
\item Reprezentacje szpiczaste. 
\end{enumerate}

% \subsection{Miara Haara}

% \subsection{Algebra Heckego}

% \subsection{Koniezmienniki}

% \section{Lemat Schura i $Z$-zwartość}
% \subsection{Charaktery}

% \subsection{Lemat Schura, charakter centralny}

% \subsection{Reprezentacje $Z$-zwarte}

% \subsection{Przykład}

% \section{Reprezentacje szpiczaste}

% \subsection{Paraboliczna indukcja, restrykcja}

% \subsection{Pary paraboliczne}

% \subsection{Reprezentacje szpiczaste}

% \subsection{Rozkład Iwahoriego}

% \subsection{Główne twierdzenie}

\begin{twierdzenie}
	Każda gładka reprezentacja nieredukowalna dla $\grupa$ nad $\cialo$ o charakterystyce różnej od $p$ jest osiągalna.
\end{twierdzenie}