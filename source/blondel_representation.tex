\section{Reduktywne grupy $p$-adyczne}
Wyłożymy jedynie podstawy teorii reprezentacji reduktywnych grup $p$-adycznych na tyle elementarnie, by treść tego rozdziału była przystępna dla początkujących.
Aby nie rozrosły się przesadnie, konieczne były drastyczne cięcia.
Mimo to staraliśmy się zachować jak najdłużej ogólność, to jest minimalne założenia: od drugiej podsekcji żądamy mianowicie, by ciało $\cialo$ nie było charakterystyki $p$.
Po drugie, chcieliśmy jednak być w stanie udowodnić cokolwiek sensownego i to się nam udało.

\subsection{Gładkość}
\subsubsection{Grupy lokalnie proskończone}
Przypomnijmy (nie do końca wiadomo skąd), że grupa topologiczna $\grupa$ jest \emph{proskończona}, jeśli jest zwarta i całkowicie niespójna (składowe spójności liczą równo jeden element).
W takim przypadku $\grupa$ jest izomorficzna z $\varprojlim \grupa / U$, gdzie $U$ przebiega otwarte dzielniki normalne $\grupa$.
Zauważmy, że wszystkie ilorazy $\grupa / U$ są skończone.
Odwrotnie, taka granica odwrotna jest zwarta i całkowicie rozłączna, zatem proskończona.
Jeśli ilorazy były $p$-grupami, to $\grupa$ jest pro-$p$-grupą.
Przykładem takiej sytuacji jest znana nam już równość $\Z_p = \varprojlim \Z/p^n\Z$.

\begin{definicja}
	Grupa topologiczna $\grupa$ jest lokalnie proskończona, gdy każde otoczenie $e \in G$ zawiera otwartą i zwartą podgrupę lub (równoważnie) $\grupa$ jest lokalnie zwarta i całkowicie niespójna.
\end{definicja}

\begin{przyklad}
	Addytywna grupa $\Q_p$ lub ogólniej, lokalnego ciała niearchimedesowego.
	Grupa multiplikatywna $\Q_p^\times$.
	Przestrzenie liniowe nad $\Q_p$ skończonego wymiaru.
	Grupa macierzy $n \times n$ nad tym ciałem, podgrupa (otwarta) macierzy odwracalnych.
\end{przyklad}

% \subsection{Podstawy teorii}
\subsubsection{Podstawy teorii reprezentacji}
\begin{definicja}
	Reprezentacja $(\pi, \liniowa)$ grupy $\grupa$ w przestrzeni $\liniowa$ nad ciałem $\cialo$ to homomorfizm $\pi \colon G \to \textrm{Aut}_{\cialo}(\liniowa)$.
\end{definicja}

Morfizm między reprezentacjami to liniowy homomorfizm $\phi \colon \liniowa_1 \to \liniowa_2$, że $\phi \circ \pi_1(g) = \pi_2(g) \circ \phi$ dla wszystkich $g \in \grupa$.
Zbiór morfizmów to $\textrm{Hom}_\grupa(\pi_1, \pi_2)$.
Jeśli zawiera liniowy izomorfizm, to reprezentacje nazwiemy izomorficznymi.

Reprezentacja jest \emph{nierozkładalna}, jeśli $\liniowa$ nie jest sumą prostą właściwych $\grupa$-stabilnych podprzestrzeni (stabilnych względem automorfizmów z $\pi(\grupa)$).

\emph{Nieskracalna}, jeśli jest niezerowa i żadna właściwa niezerowa podprzestrzeń $\liniowa$ nie jest stabilny względem automorfizmów z $\pi(\grupa)$.

\subsubsection{Gładkie reprezentacje}
Wracamy do lokalnie proskończonych grup $\grupa$ i wymagamy więcej od reprezentacji zależnie od dodatkowej struktury na $\grupa$.
Chcemy brać pod uwagę topologię na $\grupa$: $(g, v) \mapsto \pi(g) v$ powinno być ciągłą funkcją $\grupa \times \liniowa \to \liniowa$, kiedy $\liniowa$ jest dyskretną przestrzenią.

\begin{definicja}
	Reprezentacja $(\pi, \liniowa)$ dla $\grupa$ jest gładka, jeżeli $\grupa_v = \{g \in \grupa : \pi(g)(v) = v\}$, stabilizatory $v$ w $\grupa$, są otwartymi podzbiorami $\grupa$.
\end{definicja}

\subsubsection{Reprezentacje dziedziczone}

\subsection{Osiągalność}
\subsubsection{Reprezentacje osiągalne}
\begin{definicja}
	Gładka reprezentacja $(\pi, \liniowa)$ dla $\grupa$ jest {osiągalna}, o ile $\liniowa^K$ ma skończony wymiar dla każdej zwartej podgrupy otwartej $K \le \grupa$.
\end{definicja}

\subsubsection{Miara Haara}

\subsubsection{Algebra Heckego}

\subsubsection{Koniezmienniki}

\subsection{Lemat Schura}
\subsubsection{Quasi-charaktery}
\subsubsection{Charakter centralny}
\subsubsection{Reprezentacje $Z$-zwarte}
\subsubsection{Przykład}

\subsection{Szpiczastość}
\subsubsection{Paraboliczna indukcja, restrykcja}
\subsubsection{Paraboliczna para}
\subsubsection{Reprezentacje szpiczaste}
\subsubsection{Rozkład Iwahoriego}
\subsubsection{Gładkie reprezentacje nieskracalne są osiągalne}

\begin{twierdzenie}[Vigneras, 2005]
	Każda gładka reprezentacja nieredukowalna dla $\grupa$ nad $\cialo$ o charakterystyce różnej od $p$ jest osiągalna.
\end{twierdzenie}