\section{Normowa niezależność}
Zaprezentujemy \prawo{Cassels\\2.3} teraz pogląd Casselsa na temat niezależności nierównoważnych norm.
Co dokładnie przez to rozumiemy, stanie się jasne natychmiast po udowodnieniu lematu.

\begin{lemat}
	Niech nietrywialne normy $|\cdot|_1, \ldots, |\cdot|_m$ będą parami nierównoważne (na ciele $\cialo$).
	Istnieje wtedy $x \in \cialo$, że $|x|_1 > 1$, ale $|x|_2, \ldots, |x|_m < 1$.
\end{lemat}

\begin{proof}
	Indukcyjny względem $m$.
	Gdy $m = 2$, istnieją $y, z \in \cialo$, takie że $|y|_1, |z|_2 < 1$ oraz $|y|_2, |z|_1 \ge 1$.
	Poszukiwanym jest wtedy $x = zy^{-1}$.
	
	Jeżeli $m > 2$, to z założenia indukcyjnego mamy $y \in \cialo$, że $|y|_1 > 1$, natomiast dla $2 \le i \le m-1$ zachodzi $|y|_i < 1$.
	Z drugiej strony istnieje $z \in \cialo$, że $|z|_1 > 1$, $|z|_m < 1$.
	Rozpatrujemy trzy przypadki.

	Jeżeli $|y|_m < 1$, to $x = y$.
	Jeżeli $|y|_m = 1$, to $x = y^nz$ dla dużego $n$.
	Jeżeli $|y|_m > 1$, to $x = y^n z (1 + y^n)^{-1}$ dla dużego $n$.
	Mamy bowiem
	\[
		\frac{y^n}{1 + y^n} \to \begin{cases} 
		1 & \text{dla } |\cdot|_1 \text{ oraz } |\cdot|_m, \\
		0 & \text{w przeciwnym razie.}
		\end{cases} \qedhere
	\]
\end{proof}

\begin{fakt}
	Przy założeniach lematu, $x_1, \ldots, x_m \in \cialo$ oraz $\varepsilon > 0$ (rzeczywistym), istnieje $x \in \cialo$, że jednocześnie spełniona jest każda z nierówności $|x-x_i|_i < \varepsilon$.
\end{fakt}

\begin{proof}
	Z lematu wynika istnienie takich $y_i \in \cialo$, że $|y_i|_i > 1$, $|y_i|_k < 1$ ($k \neq i$).
	Wystarczy położyć
	\[
		x = \lim_{n \to \infty} \sum_{i=1}^m \frac{y_i^n}{1 + y_i^n} x_i. \qedhere
	\]
\end{proof}

Związane jest to z chińskim twierdzeniem o resztach.
Mówi ono, że gdy $x_i \in \Z$ są dane, $p_i$ parami różne (i pierwsze), zaś $m_i$ naturalne, to układ ,,kongruencji'' $|x - x_i|_i \le m_i$
ma rozwiązanie nie tylko w $\Q$, ale także $\Z$.
Nasz fakt można jednak wzmocnić, gdy $\cialo$ jest algebraicznym ciałem liczbowym (uczynimy to, ale jeszcze nie teraz).

Przedstawimy teraz obrazowo niezależność.

\begin{fakt}
	Niech $\cialo_i$ będzie uzupełnieniem ciała $\cialo$ względem parami nierównoważnych norm.
	Wtedy odwzorowanie przekątniowe $\Delta: \cialo \hookrightarrow \prod_i \cialo_i$ ma gęsty obraz.
\end{fakt}

\begin{proof}
	Ustalmy elementy $x_i \in \cialo_i$ dla $1 \le i \le n$.
	Istnieją wtedy $y_i \in \cialo$, że $|x_i-y_i|_i < \varepsilon$ dla ustalonego $\varepsilon > 0$.
	Mamy takie $z \in \cialo$, że $|z - y_i|_i < \varepsilon$, zatem $|z - x_i|_i < 2 \varepsilon$ (na mocy poprzedniego faktu).
\end{proof}

\section{Podsumowanie i uwagi historyczne}
Pojęcie (oraz sama nazwa) \emph{waluacji} pochodzi z pracy Kürschaka z roku 1913.
Przy ich użyciu pokazał, że każde ciało z waluacją można rozszerzyć do ciała zupełnego oraz algebraicznie domkniętego; zainspirowała go książka Hensela z 1908 (przykład: $\Q_p$ i $\C_p$).

Świat dowiedział się o twierdzeniu Ostrowskiego w roku 1918, chociaż Ostrowski potrafił je uzasadnić kwietniem dwa lata wcześniej.
My przedstawiliśmy dowód Artina (1932).

Lemat Hensela (przedstawiony później w dużo większej ogólności) jest wnioskiem z lematu Hensela-Rychlika, o którym nikt nie pamięta.

Reguła lokalno-globalna (dla form kwadratowych) pojawia się w rozprawie doktorskiej Hassego z 1921 roku.
Hasse rok wcześniej przeniósł się do Marburga z Göttingen po tym, jak odkrył w antykwariacie książkę ,,Zahlentheorie'' Hensela z 1913 roku.
Minkowski zmarł w 1909 r., jednak podstawy geometrii liczb wyłożył trzynaście lat wcześniej.

Zalecana literatura:

\begin{enumerate}
	\item Fernando Q. Gouvea -- $p$-adic numbers, an introduction (od 2.1 do 3.5)
	\item Alain M. Robert -- A course in $p$-adic analysis (1.6.7)
	\item John W. S. Cassels -- Local fields (2.3)
\end{enumerate}