\section{Eksponens (klasyczny)}
W $\R$ szereg $\exp(x) = \sum_{n = 0}^\infty x^n/n!$ zbiega wszędzie, bo $1/n!$ bardzo szybko maleje: ale nie w $\Q_p$.
Trzeba więc określić tempo wzrostu tych współczynników.

\begin{lemat}
	Zachodzi $(p-1) v_p(n!) < n$.
\end{lemat}

\begin{proof}
	Prawdziwość nierówności wynika z
	\[
		v_p(n!) = \sum_{i = 1}^\infty \left \lfloor \frac n {p^i} \right \rfloor \le \sum_{i = 1}^\infty \frac n {p^i} = \frac n {p-1}. \qedhere
	\]
\end{proof}

\begin{lemat}
	Szereg $\sum_{n = 0}^\infty {x^n}:{n!}$, protoplasta eksponensa, zbiega wtedy i tylko wtedy gdy $v_p(x) > 1 / (p-1)$. 
\end{lemat}

\begin{proof}
	Promieniem zbieżności ,,protoplasty'' jest co najmniej $p^{-1/(p-1)}$, gdyż $|1/n!| = p^{v_p(n!)} < p^{n/(p-1)}$.
	
	Z drugiej stony, gdy $n = p^m$, to $v_p(n!) = (n-1)/(p-1)$.
	Jeśli ustalimy $x$ o waluacji równej $1 / (p-1)$, to $x^n/n!$ nie dąży do zera (a sam szereg nie jest zbieżny), gdyż poniższe wyrażenie nie zależy od $m$.
	\[
		v_p\left ( \frac{x^n}{n!}\right) = \frac{p^m}{p-1} - \frac{p^m-1}{p-1} = \frac{1}{p-1}. \qedhere
	\]
\end{proof}

Nierówność z lematu jest trochę dziwna, przecież dla $p \neq 2$ i $x \in \Z_p$ wartość bezwzględna, $|x|$, może być albo większa lub równa $1$ (więc większa niż $p^{-1/(p-1)}$) lub mniejsza lub równa $1/p$ (ta zaś liczba jest mniejsza), nie ma wartości ,,pomiędzy''.
A zatem dysk z lematu jest po prostu otwartym o promieniu jeden: lemat nie jest jednak bezsensowny, w szczególności dla ciał, które zawierają $\Q_p$ (np. $\C_p$).
%Podsumujmy: funkcja $g(x)$ zbiega dla $x \in p\Z_p$ (jeśli $p\neq 2$) i dla $x \in 4 Z_2$ (dla $p = 2$).

\begin{definicja}
	Eksponens $\exp_p \colon \mathcal B \to \Q_p$ jest określona na $p\Z_p$ (dla $p \neq 2$) lub $4\Z_2$ przez podany wcześniej szereg.
\end{definicja}

\begin{fakt}
	Jeżeli $x, y, x+y \in \mathcal B(0, p^{-1/(p+1)})$, to $\exp(x+y)$ jest równe $\exp x \exp y$.
\end{fakt}

\begin{proof}
	Dowód to po prostu formalna manipulacja szeregów.
	\begin{align*}
		L & = \exp_p(x+y) = \sum_{n \ge 0} \frac{(x+y)^n}{n!} = \\
		% & = \sum_{n = 0}^\infty \frac{1}{n!} \sum_{k = 0}^n C^n_k x^{n-k} y^k \\
		& = \sum_{n \ge 0} \sum_{k \le n} \frac{1}{n!} \frac{n!}{k!(n-k)!} x^{n-k}y^k \\
		& = \sum_{n \ge 0} \sum_{k \le n} \frac{x^{n-k}}{(n-k)!} \frac{y^k}{k!} = \sum_{m \ge 0} \frac{x^m}{m!} \cdot \sum_{k \ge 0} \frac{y^k}{k!} \\
		& = \exp_p(x) \exp_p(y) = R\qedhere
	\end{align*}
\end{proof}

Zwykła eksponensa i logarytm są do siebie odwrotne, czy $\exp_p(\log_p(1+x)) = 1+x$ tam, gdzie jest zbieżność?
Fakt \ref{auctoris} ma założenia, które trzeba sprawdzić.

\begin{fakt}
	Załóżmy, że jest $|x| < p^{-1/(p-1)}$ ($x \in \Z_p$). Zachodzi wtedy $\log_p (\exp_p x) = x$ oraz $\exp_p(\log_p (1+x)) = 1 + x$.
\end{fakt}

\begin{proof}
	Bez straty ogólności niech $x \neq 0$.
	Podczas składania $\log_p(\exp_p x)$, wstawiamy $\exp_p x - 1$ do wzoru na $\log_p (1 + x)$.
	Skoro tak, korzystamy z założenia i piszemy:
	\[
		|\exp_p x - 1| = \left|\sum_{n=1}^\infty \frac {x^n}{n!} \right| < \frac{|x|^n}{p^{n / (p-1)}} < 1
	\]

	Można jeszcze lepiej oszacować waluację $v = v_p(x^{n-1} / n!)$:
	\[
		v = (n-1)v_p(x) - v_p(n!) > \frac{n-1}{p-1} - \frac{n-s}{p-1} > 0,
	\]
	gdzie $s$ to suma cyfr $n$ w rozwinięciu $p$-adycznym i $n \ge 2$.
	Stąd wynika, że $v_p(x) < v_p(x^n / n!)$ i wreszcie korzystamy z lematu \ref{auctoris} dla $\log_p \circ \exp_p$, gdyż
	\[
		p^{-1/(p-1)} > |\exp_p(x) - 1| = |x| > |x^n/n!|.
	\]

	Złożymy teraz szeregi w drugą stronę.
	Niech $n > 1$ oraz $a_n = - (-x)^n / n$.
	Wtedy (dla $v = v_p(a_n) - v_p(x)$) mamy
	
	\begin{align*}
		v & = (n-1) v_p(x) - v_p(n) \\
		& > \frac{n-1}{p-1} - v_p(n) = (n-1) \left[ \frac{1}{p-1} - \frac{v_p(n)}{n-1}\right].
	\end{align*}

	Wykażemy nieujemność tego, co pozostało w nawiasach.
	Niech $n = p^v n'$ z $n ' \nmid p$, czyli
	
	\begin{align*}
		\frac{v_p(n)}{n-1} & = \frac{v}{p^v n' -1 } \le \frac{v}{p^v - 1} \\
		& = \frac{1}{p-1} \cdot \frac{v}{p^{v-1} + \ldots + p + 1} \le \frac{1}{p-1}.
	\end{align*}

	A zatem $|(-1)^{n+1} x^n/n| < |x|$ i do akcji wkracza fakt \ref{ingentis}: $|\log_p(x)| = |x| < p^{-1/(p-1)}$ daje żądaną równość.
\end{proof}

Ostrożność była potrzebna: dla $p = 2$, $x = -2$ ,,wszystko'' zbiega, ale $\exp(\log_p(1+x)) = \exp(0) = 1 \neq -1$.

W ciałach charakterystyki $p$ dzieje się coś niedobrego.
\begin{fakt}
	Analityczna $f$ z wypukłego otoczenia zera spełniająca jeden z warunków: $f' = f$ lub $f(x+y) = f(x)f(y)$ jest zerem.
\end{fakt}

Szereg z definicji eksponensa wcale nie ma sensu, jako że $n!$ nie odwraca się dla $n \ge p$.