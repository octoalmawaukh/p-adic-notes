\section{Klasyfikacja lokalnie zwartych ciał}
Podejmiemy \prawo{Rbrt\\2.App} się klasyfikacji czegoś więcej niż samych tylko ciał ultrametrycznych: lokalnie zwartych ciał charakterystyki zero.

Zakładamy po cichu, że do dyspozycji mamy miarę Haara: na każdej lokalnie zwartej grupie $G$ istnieje dodatnia miara Radona $\mu$, która jest niezmiennicza na lewe przesunięcia.
Na taką miarę patrzymy jak na dodatni ciągły funkcjonał liniowy na przestrzeni zwarcie niesionych funkcji ciągłych $G \to \R$ (lub inaczej).
Brzmi nieźle i pozwala na udowodnienie, że lokalnie zwarte ciała są metryzowalne.
Ich klasyfikacja: $\R$, $\C$, skończone rozszerzenia $\Q_p$ (chyba?).
%Chaotyczne to wszystko.
