\section{Ciągłość na $\Z_p$}
Przed lekturą tego ustępu warto przypomnieć sobie definicję i podstawowe własności jednostajnej zbieżności.

Punktowa granica ciągłych funkcji z $X$ (topologicznej) w $M$ (zupełną metryczną) jest ciągła, jeśli jednostajna.

Jeśli ustalimy ciągłą injekcję $\varphi \colon \Z_p \to \R$ (choćby liniowy model $\Z_p$), to możemy przybliżać jednostajnie wielomianami od $\varphi$ ciągłą $f \colon \Z_p \to \R$. 
Istotnie, algebra wielomianów od $\varphi$ jest podalgebrą wszystkich ciągłych $\Z_p \to \R$, która rozdziela punkty ($\Z_p$ jest zwarta).
Tw. Stone'a-Weierstraßa orzeka, że ta podalgebra jest gęsta z jednostajną zbieżnością.

Niech $f \colon \Z_p \to \C_p$ będzie ciągła.
Funkcja $|f| \colon \Z_p \to \R$ też jest ciągła i osiąga supremum.
Dokładniej, zbiór $f[\Z_p] \subseteq \C_p$ jest zwarty, zaś $\{|f(x)| \neq 0 : x \in \Z_p\} \subseteq \R_+$: dyskretny.

Definicja pierścienia topologicznego $\pierscien$ pokazuje, że każdy wielomian $f \in \pierscien[x]$ zadaje ciągłą funkcję $\pierscien \to \pierscien$.
Kolejnymi źródłami ciągłych funkcji są:
\begin{enumx}
	\item wielomiany z $\C_p[x]$ po obcięciu do $\Z_p$
	\item szeregi potęgowe $\sum_{i \ge 0} a_i x^i$ z $a_i \in \C_p$, $|a_i| \to 0$.
\end{enumx}

\begin{definicja}
	Dla ciągłej funkcji $f \colon \Z_p \to \C_p$ przyjmijmy, że $\|f\| = \sup_{x \in \Z_p} |f(x)| = \max_{x \in \Z_p} |f(x)|$.
\end{definicja}
	
Jest jasne, że wielomiany dwumianowe wyznaczają ciągłe funkcje $f_k \colon \Z_p \to \Z_p$, $x \mapsto (x \mbox{ nad } k)$.
Zbiór $\N$ jest gęsty w $\Z_p$, zatem $\|f_k\| = \sup_{\N}|(n \mbox{ nad } k)| \le 1$.
Ponieważ $(k \mbox{ nad } k) = 1$, mamy nawet równość.

Zanim pójdziemy śladami Mahlera, żeby odwrócić proste spostrzeżenie sprzed akapitu, określimy użyteczne szeregi, które nazwano zresztą jego nazwiskiem.

\begin{definicja}
	Szereg Mahlera dla $a_k \in \C_p$ ($\Omega_p$), że $|a_k| \to 0$ to $\sum_{k \ge 0} a_k (x \mbox{ nad } k)$.
\end{definicja}

Jeśli szereg dwumianowy zbiega dla wszystkich $x \in \Z_p$ (lub dla samego $x = -1$), to czyni to jednostajnie.
Ze zbieżności w $-1$ wynika, że $a_k (-1 \mbox{ nad } k) = \pm a_k \to 0$ i $|a_k| \to 0$.

\begin{twierdzenie}[Mahler]
	Niech funkcja $f \colon \Z_p \to \C_p$ będzie ciągła, $a_k = \findif^k f(0)$.
	Wtedy $|a_k| \to 0$, zaś szereg $\sum_{k \ge 0} a_k (x \mbox{ nad } k)$ zbiega jednostajnie do $f(x)$.
	Co więcej, $\|f\| = \sup_{k \ge 0} |a_k|$.
\end{twierdzenie}

\begin{proof}
	Bez straty ogólności, zastąpmy $f \neq 0$ przez $f / f(x_0)$, gdzie $x_0 \in \Z_p$ maksymalizuje $|f(x)|$.
	Teraz obraz $f$ leży w $\mathcal O$.

	Rozważmy iloraz $E = \mathcal O / p \mathcal O$ jako przestrzeń liniową nad ciałem prostym $\mathbb F_p$.
	Złożenie $\varphi = f \mbox{ mod } p \colon \Z_p \to \mathcal O_p \to E$, jest ciągłe (przyjmuje skończenie wiele wartości, jest lokalnie stałe), ale nie jest stale zerem.
	Jest jednostajnie ciągłe, a także jednostajnie lokalnie stałe ($\Z_p$ jest zwarte).

	To oznacza, że $\varphi$ jest stała na warstwach modulo $p^t\Z_p$ dla dużych $t$, czyli $p^t$-okresowa na $\Z$.
	Skorzystamy więc z faktu \ref{vietoris}.
	Niech $T = p^t$.
	Zapiszmy $\varphi$ tak, jak niżej, przy czym znaczenie sztyletu $\dagger$ jest nieznane: $\varphi(x) = \sum_{k < T} \alpha_k (x \mbox{ nad } k)^\dagger$.

	Weźmy reprezentantów $a_k^0 \in \mathcal O$ dla $\alpha_k$.
	Przynajmniej raz $|a_k^0| = 1$, gdyż różnica $\sum_{k < T} a_k^0 f_k - f$ przyjmuje wartości w $p \mathcal O$.
	Wiemy, że $|a_k^0| \le 1$.
	Z naszej konstrukcji wynika, że jest $\|f(x) - \sum_{k < T} a_k^0 (x \mbox{ nad } k)\| = r \le |p|$.
	Jeśli różnica nie jest $0$, możemy powtórzyć proces: znaleźć $S > T$ i współczynniki $a_k^1$, że $|a_k^1| \le r$, $\max |a_k^1| = r$.
	Drobne nagięcie oznaczeń prowadzi przez $a_k^0 = 0$ dla $k \ge T$ do
	\[
		\left|f(x) - \sum_{k=0}^{S-1} (a_k^0 + a_k^1) \cdot {x \choose k} \right| = r' \le |p^2|.
	\]

	Jest jasnym, że po nieskończenie wielu krokach otrzymamy zbieżne szeregi $a_k = a_k^0 + a_k^1 + \ldots \in \C_p$, że $|a_k^n| \le |p^n| \to 0$.
	Zachodzi przy tym $\sup_{k > 0} |a_k| = \sup_{k<T} |a_k| = 1 = \|f\|$ i to już koniec: $\|f(x) - \sum_{k \ge 0} a_k (x \mbox { nad } k) \| < |p|^m$, $m \in \N$.
\end{proof}

\begin{wniosek}
	Ciągłe funkcje $\Z_p \to \C_p$ to dokładnie jednostajne granice wielomianów z $\C_p[X]$.
\end{wniosek}

Znajomość jednostajnej zbieżności, zwartych przestrzeni metrycznych, funkcji ciągłych i twierdzenia Mahlera pozwala przeprowadzić częściowo indukcyjny dowód następującego faktu.

\begin{fakt}
	Następujące warunki są sobie równoważne dla funkcji $f \colon \N \to \C_p$ oraz $a_k = \findif^k f(0)$:
	$|a_k| \to 0$;
	$\|\findif^kf\| \to 0$;
	$f$ ma ciągłe przedłużenie do $\Z_p \to \C_p$;
	$f$ jest jednostajnie ciągła (na $\N$ z topologią od $\Z_p$);
	szereg Mahlera dla $f$ zbiega jednostajnie. %  $\sum_{k\ge0}a_k(x \mbox{ nad } k)$
\end{fakt}

Twierdzenie Mahlera ma ciekawe zastosowania dla splotów (przesuniętych).
Okazuje się, że dzięki temu można oszacować resztę w skończonym rozwinięciu Mahlera.
Przypomnijmy,
\[
	|(f \convulsion g)(n)| \le \max |f(i) g(n-i-1)| \le \|f\| \cdot \|g\|
\]

\begin{fakt}
	Przesunięty splot $f \convulsion g$ ciągłych funkcji $\Z_p \to \C_p$ daje się przedłużyć do ciągłej funkcji $\Z_p \to \C_p$.
\end{fakt}

\begin{proof}
Pokażemy, że $\findif^k (f \convulsion g)(0) \to 0$.
Wróćmy do 
\[
	\findif^{2n+1} (f \convulsion g) = \sum_{i + j = 2n} \findif^i f \cdot \findif^j g(0) + f \convulsion \findif^{2n+1} g
\]

Dla ograniczonej funkcji $h$ ultrametryka daje $\|\findif h\| \le \|h\|$.
Rozbijemy lewą stronę powyższego równania na trzy człony.
\begin{align*}
\left|\sum_{i = n}^{2n} \findif^i f (0) \cdot \findif^{2n-i} g(0) \right| & \le \|\findif^n f\| \cdot \|g\| \\
\left|\sum_{i = 0}^{n-1} \findif^i f(0) \cdot \findif^{2n - i} g(0) \right| & \le \|f\| \cdot \|\findif^n g\| \\
\left|(f \convulsion \findif^{2n+1}g)(0)\right| & \le \|f\| \cdot \|\findif^{2n+1} g\| \\
& \le \|f\| \cdot \|\findif^n g\|
\end{align*}

Prawe strony nierówności dążą do $0$, gdy $n$ rośnie.
Można podać podobne oszacowania dla $\findif^{2n}$ miast $\findif^{2n+1}$.
\end{proof}

\begin{wniosek}
	Twierdzenie van Hamme'a jest prawdziwe także dla $f \colon \Z_p \to \C_p$, z oszacowaniem $\|R_{n+1} f\| \le \|\findif^{n+1}f\| \to 0$.
\end{wniosek}

%Poniższy wniosek pokrzyżuje nam wkrótce szyki.

\begin{wniosek}\label{glorreichetraume}
	Jedyną liniowa formą $C(\Z_p, \mathcal K) \to \cialo$, która jest odporna na przesuwanie, jest forma zerowa: $\varphi \equiv 0$.
\end{wniosek}

\begin{proof}
	Ustalmy $f \in C(\Z_p, \cialo)$ z pierwotną $F = \suma f$.
	Wtedy $\varphi(f(x)) = \varphi(F(x+1)) - \varphi(F(x)) = 0$.
\end{proof}

\begin{przyklad}
	Funkcja $f \colon \Z_p \setminus \{1\} \to \Q_p$ jest nieograniczona, ale ciągła: $f(x) = \sum_{n \ge 0} (x \mbox{ nad } p^{2n}-1)p^{-n}$.
\end{przyklad}

Człowiek może się zastanawiać, dlaczego w definicje szeregu Mahlera pojawiają się symbole Newtona, a nie zwykłe potęgi $x$.

\begin{fakt}
	Funkcje $f_n(x) = x^n$ nie tworzą ortonomalnej bazy p. funkcji ciągłych, ograniczonych z $\mathcal O \subseteq \cialo$ (zupełnego) w $L$, ,,$BC$''.
\end{fakt}

\begin{proof}
	Liniowo niezależne funkcje $f_n$ mają normę $1$.
	%(supremum) jeden.
	Jeśli $L$ nie jest lokalnie zwarta, zbiór $\{f_n\}$  jest ortonormalny, ale jego $L$-liniowa powłoka nie jest gęsta w ,,$BC$''.
	
	Jeśli jednak jest, to $f_n$ nie są zbiorem ortogonalnym (!), choć ich $L$-powłoka jest gęsta wśród ciągłych $X \to L$.
\end{proof}