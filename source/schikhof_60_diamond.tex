\section{Logarytm (diamentowy)}
Poznamy zaraz funkcję, która choć nie jest równa $\log_p \Gamma_p$, i tak zasługuje na swoją nazwę: log gamma Diamonda.

\begin{definicja}
	Dla $x \in \C_p \setminus \Z_p$ niech
	\[
		G_p(x) = \intzp (x+u) (\log_p (x+u) - 1) \,\textrm{d}u.
	\]
\end{definicja}

\begin{fakt}
	$G_p(x+1) - G_p(x) = \log_p x$.
\end{fakt}

\begin{proof}
	Niech $f(x, u) = (x+u)(\log p(x+u) - 1)$.
	Wtedy to $f(x + 1, u) - f(x, u) = f(x, u + 1) - f(x, u)$, zatem prawdą jest też $G_p(x+1) - G_p(x) = \partial_uf(x, 0) = \log_p(x)$.
\end{proof}

\begin{fakt}
	$G_p(1-x) = - G_p(x)$.
\end{fakt}

\begin{proof}
	Skoro $\log_p(-1) = 0$, to $f(-x, u) = - f(x, -u)$.
	\begin{align*}
		- G_p (-x) & = \intzp f(x, -u) \,\textrm{d}u
		= \intzp -f(x, u+1) \,\textrm{d}u \\
		& =  \intzp -f(x+1, u) \,\textrm{d}u
		= G_p(1+x). \qedhere
	\end{align*}
\end{proof}

\begin{fakt}
	Dla $m \in \N$,
	\[
		G_p(x) = \frac{2x-1}{2} \cdot \log_p m + \sum_{j=0}^{m-1} G_p \frac{x+j}{m}.
	\]
\end{fakt}

\begin{proof}
	Mamy $f(y, mu) = (y + mu) \log_p m + m f(y/m, u)$, a do tego równość
	\[
		G_p(x) = \intzp f(x, u) \,\textrm{d}u = \frac 1m \sum_{j = 0}^{m-1} f(x, j+mu) \,\textrm{d}u.
	\]
	Podkładając $y = x+j$ dostajemy
	\[
		\intzp f(x, j + mu) \,\textrm{d} u = (y - \frac m 2) \log_p m + m G_p \frac{y}{m}.
	\]
	Ale
	\[
		\frac 1m \sum_{j=0}^{m-1} (x + j - \frac m2) \log_pm = (x - \frac 12) \log_p m. \qedhere
	\]
\end{proof}

\begin{fakt}
	Funkcja $G_p$ jest lokalnie analityczny.
\end{fakt}

\begin{fakt}
	Związek z $\Gamma_p$ Mority:
	\[
		\log_p \Gamma_p (x) = \sum_{j=0}^{p-1} G_p \frac{x+j}{p},
	\]
	ale sumujemy jedynie po tych $j$, że $|x + j|_p = 1$.
\end{fakt}