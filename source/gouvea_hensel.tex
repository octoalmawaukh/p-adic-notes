\section{Lemat Hensela o podnoszeniu}
,,Lemat Hensela'' opisuje jedną z ważniejszych algebraicznych cech ciał udających $\Q_p$ (zupełnych oraz z niearchimedesową normą).
Orzeka mianowicie, że w pewnych warunkach można łatwo sprawdzić, czy wielomian ma pierwiastki w $\Z_p$.

\begin{twierdzenie}[lemat Hensela]
	Każde z zer $x_1 \in \Z_p$ (modulo $p \Z_p$) dla wielomianu $f(x) \in \Z_p[x]$, że $f'(x_1) \not \equiv 0$ mod $p \Z_p$ można podnieść do prawdziwego zera $x$, które przystaje do $x_1$ mod $p\Z_p$.
	Co więcej, zero to jest jednoznacznie wyznaczone.
\end{twierdzenie}

\begin{proof}
	Wskażemy ciąg Cauchy'ego zbieżny do $x$ przy użyciu ,,metody Newtona'' ($x_n$), taki że $f(x_n) \equiv 0$ mod $p^n$ i $x_n \equiv x_{n+1}$ mod $p^n$.
	Mamy $x_1$, chcemy $x_2 = x_1 + y_1 p$ dla $y_1 \in \Z_p$.

	Widzimy, że $f(x_2) = f(x_1) + f'(x_1) y_1p + p^2 \cdot r_2$ (gruz).
	Szukamy $y_1$, dla którego $f(x_1) + f'(x_1) y_1p \equiv 0$ mod $p^2$, czyli $z_1 + f'(z_1) y_1 \equiv 0$ mod $p$, gdzie $f(x_1) = pz_1$. 
	Rozwiązaniem jest $y_1 \equiv -z_1 f'(x_1)^{-1}$ mod $p$.
	Uważny Czytelnik zauważy, że skoro z $x_1$ można dostać $x_2$, to z $x_n$ można dostać $x_{n+1}$.
\end{proof}

W dowodzie skorzystaliśmy ze wzoru Taylora:

\begin{fakt}
	Dla wielomianu $f(x)$ nad ciałem $\cialo$ charakterystyki zero jest $f(x+h) = f(x) + f'(x)h + f''(x)h^2 / 2 + \ldots$, $x, h \in \cialo$.
\end{fakt}

\begin{proof}
	Nieustanne różniczkowanie sprawia, że wielomian $f$ kiedyś stanie się zerem.
	Wystarczy porównać współczynniki przy $x^j$ po obu stronach.
\end{proof}

\begin{historia}[Hensel Kurt]\end{historia}

Założenie z lematu ($f'(x) \not\equiv 0$) można osłabić, choćby do $|f(x)| < |f'(x)|^2$.
Dowód podał już w 1846 Schöneman (?).
Już wkrótce i tak przetłumaczymy wszystko na język waluacji.

\begin{historia}[Schönemann Theodor]\end{historia} %oddkrył lemat Hensela przed Henselem, prawo wzajemności Scholza przed Scholzem oraz kryterium Eisensteina przed Eisensteinem.

Wyznaczymy teraz pierwiastki jedności w $\Q_p$ wielomianem $f(x) = x^m - 1$ z pochodną $f'(x) = mx^{m-1}$.
Aby spełnione było drugie założenie z lematu, musimy mieć $p \nmid m$ (zakładamy to) i pozostaje sprawdzić pierwsze założenie.

\begin{lemat}
	Niech $p \nmid m$.
	Istnieje taka całkowita $n$, że $n^m \equiv 1$ mod $p$ (ale $n \not\equiv 1$ mod $p$), wtedy i tylko wtedy gdy $(m, p - 1) > 1$.
	Dla każdego $n$, najmniejsza $m$ o żądanych własnościach dzieli $p-1$.
\end{lemat}

\begin{proof}
	Załóżmy istnienie $n$.
	Rząd $n$ w $(\Z / p\Z)^\times$ dzieli zarówno $m$, jak i $p-1$, zatem $(m, p-1) > 1$, chyba że $n \equiv 1$ mod $p$.
	Najmniejsze $m$ musi dzielić NWD, a z nim także $p-1$.

	Odwrotnie, w grupie cyklicznej rzędu $p-1$ istnieje element każdego rzędu, który dzieli $p-1$, a taka jest $(\Z/p\Z)^\times$.
\end{proof}

Lemat Hensela daje:

\begin{fakt}
	Jeżeli naturalna $m$ nie dzieli się przez pierwszą $p$, to w $\Q_p$ istnieje $m$-ty pierwiastek pierwotny z jedynki, wtedy i tylko wtedy gdy $m$ dzieli $p-1$.
\end{fakt}

Nie wykluczyliśmy jeszcze istnienia $p^n$-tych pierwiastków jedności w $\Q_p$, uda się to po poznaniu logarytmu.
Pierwiastki jedności w $\Q_p$ dla $p \ge 3$ tworzą grupę $\mu_{p-1}$ o $p-1$ elementach.

,,Jednostka urojona'', czyli kwadratowy pierwiastek z $-1$ w $\Q_p$ istnieje dokładnie wtedy, gdy $\frac 1 2(p-1)$ jest jeszcze parzysta, czyli dla $p$ postaci $4k+1$.

Teraz zajmiemy się kwadratami.

\begin{fakt}
	Jeśli tylko $p > 2$, to każda $p$-adyczna jedność $y$, dla której istnieje $z$, że $z^2 \equiv y$ mod $p \Z_p$, jest kwadratem czegoś z $\Z_p^\times$.
\end{fakt}

\begin{proof}
	Lemat Hensela dla $x^2 - y$, bo $p \neq 2$ i $y \in \Z_p^\times$ pociągają $2 z \not\equiv 0$ mod $p$.
\end{proof}

\begin{wniosek}
	$\{x^2 : x \in \Q_p\} = \{p^{2n} y^2 : n \in \Z, y \in \Z_p^\times\}$, a grupa ilorazowa $\Q_p^\times / (\Q_p^\times)^2$ ma rząd cztery i reprezentantów warstw $\{1, p, c, cp\}$, przy czym $c \in \Z_p^\times$ jest dowolnym elementem, którego redukcja mod $p$ nie jest resztą kwadratową.
\end{wniosek}

\begin{proof}
	Własności reszt kwadratowych.
\end{proof}

Dla $\R$ jest inaczej: dokładnie nieujemne liczby to kwadraty, zaś $\R^\times/(\R^\times)^2$ odpowiada $\{-1, 1\}$. Co może się dziać w $\Q_2$? Potrzebna jest mocniejsza forma lematu, albowiem $f'(x) = 2x$ jest wielokrotnością dwójki.

\begin{fakt}
	Każda liczba $y \in 1 + 8\Z_2 \subseteq \Z_2$, jest kwadratem w $\Z_2$.
	Odwrotnie, $2$-adyczna jedność i kwadrat przystaje do $1$ mod $8$.
	Zatem $\Q^\times_2 / (\Q^\times_2)^2$ ma rząd osiem, odpowiada jej $\{\pm 1, \pm 2, \pm 5, \pm 10\}$.
\end{fakt}

\begin{proof}
	Wystarczy użyć wzmocnionego lematu.
\end{proof}

Lemat Hensela mówi, że jeżeli wielomian dzieli się przez $x - x_0$: $f(x) \equiv (x - x_0) g(x)$ mod $p$, to w podobny sposób daje się rozłożyć także w $\Z_p[x]$.
Warunek nałożony na pochodną dopuszcza jedynie pojedyncze pierwiastki.
Teraz osłabimy to założenie do względnej pierwszości.

Przez $\overline w$, oznaczymy redukcję współczynników wielomianu $w \in \Z_p[x]$ modulo $p$.

\begin{definicja}
	Wielomiany $q, r$ są względnie pierwsze modulo $p$, gdy $(\overline q, \overline r) = 1$ w $\mathbb F_p[x]$.
\end{definicja}

Istnieją wtedy $a, b \in \Z_p[x]$, że $aq + br \equiv 1$ mod $p$.

\begin{twierdzenie}
	Niech dla wielomianu $f(x) \in \Z_p[x]$ istnieją dwa względnie pierwsze mod $p$ wielomiany: $g_1$, $h_1 \in \Z_p[x]$, przy czym $g_1$ jest unormowany i $f(x) \equiv (g_1h_1)(x)$ mod $p$.
	Wtedy istnieją takie $g(x), h(x) \in \Z_p[x]$, że $g$ jest unormowany, $g$ i $g_1$ oraz $h$ i $h_1$ przystają do siebie mod $p$ oraz $f(x) = (gh)(x)$.
\end{twierdzenie}

\begin{proof}
	Postępujemy jak wcześniej: znajdujemy przybliżone rozwiązanie i próbujemy przejść do granicy. 

	Niech $d$ będzie stopniem $f$, zaś $m$: stopniem $g_1$.
	Możemy założyć, że $\deg h_1 \le d - m$.
	Potrzebne są nam dwa ciągi, $g_n$ i $h_n$ (wielomiany), że: każdy $g_n$ jest unormowany, $g_{n+1} \equiv g_n$ mod $p^n$ oraz $h_{n+1} \equiv h_n$ mod $p^n$, a przy tym $f \equiv g_n h_n$ mod $p^n$.
	Wielomiany $h, g$ otrzymamy przez przejście do granicy.

	Mamy już $g_1, h_1$.
	Musi zachodzić $g_2 = g_1 + pr_1$, a przy tym $h_2 = h_1 + ps_1$.
	Po podstawieniu do równania $f \equiv g_2h_2$ mod $p^2$ i uproszczeniu otrzymamy $f - g_1 h_1 = pk_1$ dla $k_1 \in \Z_p[x]$.
	Dalsze uproszczenie do $p k_1 \equiv pr_1 h_1 + p s_1 g_1$ mod ${p^2}$ sprawia, że chcemy podzielić przez $p$.

	Skoro $g_1, h_1$ są względnie pierwsze mod $p$, to istnieją $a, b$ (wielomiany nad $\Z_p$), że $ag_1 + bh_1 \equiv 1$ mod $p$.
	Rozpatrzmy nowe wielomiany, $\overline r_1 = bk_1$ i $\overline s_1 = ak_1$.
	Wiemy już, że
	\[
		\overline r_1 h_1 + \overline s_1 g_1 \equiv k_1 \pmod p.
	\]

	Podzielmy $\overline r_1$ przez $g_1$; niech $r_1$ będzie resztą: $r_1 = g_1 q + r_1$.
	Rzecz jasna $\deg r_1 < \deg g_1$.
	Ale jeśli położymy $s_1 = \overline s_1 + h_1q$, to wszystko będzie grać:
	\begin{align*}
	\ldots & = r_1 h_1 + s_1 g_1
	\equiv (\overline r_1 - g_1 q)h_1 + (\overline s_1 + h_1 q)g_1 \\
	& \equiv \overline r_1 h_1 - g_1 h_1 q + \overline s_1 g_1 + g_1 h_1 q
	\equiv \overline r_1 h_1 + \overline s_1 g_1 \\
	& \equiv k_1 \pmod p.
	\end{align*}

	Tak pokazaliśmy, że $g_2$ oraz $h_2$ istnieją.
	Skoro przystają do $g_1$ i $h_1$ mod $p$, to również są względnie pierwsze mod $p$ i możemy wykonać kolejny krok ,,indukcyjny''.
\end{proof}
