\section{Lemat Hensela o podnoszeniu}
Niewinne \prawo{Gouvea\\3.4} stwierdzenie, jakie za chwilę podamy (w najprostszej postaci, dla ciała $\Q_p$), opisuje jedną z najważniejszych cech zupełnych ciał z niearchimedesową normą: mówi, że istnieje prosty przepis na sprawdzenie, czy wielomian ma pierwiastki w pierścieniu $\Z_p$.
Wystarczy znać bardzo przybliżone oszacowanie pierwiastka i wiedzieć, że pochodnia nie wariuje.

\begin{twierdzenie}[lemat Hensela]
	Każde z zer $x_1 \in \Z_p$ dla wielomianu $f(x) \in \Z_p[x]$ mod $p\Z_p$, że $f'(x_1) \neq 0$ mod $p \Z_p$ można podnieść do prawdziwego zera $x \in \Z_p$, które przystaje do $x_1$ mod $p\Z_p$.
	Co więcej, zero to jest jednoznacznie wyznaczone.
\end{twierdzenie}

\begin{proof}
	Wskażemy ciąg Cauchy'ego zbieżny do $x$ i odwołamy się do zupełności.

	Użyta przez nas idea znana jest jako metoda Newtona, jednak w analizie rzeczywistej wymaga mocniejszych założeń i przypomni nieco to, czego już dokonaliśmy.
	Dokładniej, skonstruujemy ciąg liczb całkowitych $x_n$, taki że $f(x_n) = 0$ mod $p^n$ oraz $x_n = x_{n+1}$ mod $p^n$.
	Łatwo widać, że będzie on ciągiem Cauchy'ego, zaś jego granica stanowić będzie prawdziwe zero wielomianu $f$.
	Odwrotnie, mając zero łatwo odtworzymy ciąg, z którego powstało.

	Liczba $x_1$ istnieje na mocy założeń.
	Niech $x_2 = x_1 + y_1 p$ dla pewnej (nieznanej nam) liczby $y_1 \in \Z_p$.
	Wpychając ją do wielomianu $f$ i rozwijając dostaniemy prostsze równanie, $f(x_2) = f(x_1) + f'(x_1) y_1 p$ mod $p^2$, jako że wyższe wyrazy nas nie obchodzą.
	
	Czujemy potrzebę rozwiązania równania $px + f'(x_1)b_1p = 0$ mod $p^2$, co arcytrudne nie jest.
	Rozwiązanie to $y_1 = - f(x_1) (p f'(x_1))^{-1}$ mod $p$.
	
	Uważny Czytelnik zauważy, że skoro z $x_1$ można dostać $x_2$, to z $x_n$ można dostać $x_{n+1}$ na mocy indukcji.
	Na każdym kroku tylko jeden wyraz pasował do reszty układanki, zatem i cały ciąg $(x_n)$ jest jedyny.
\end{proof}

W dowodzie skorzystaliśmy ze wzoru Taylora. 

\begin{fakt}
	Dla ustalonych $x, h \in \cialo$, ciała $\cialo$ charakterystyki zero oraz wielomianu $f$, 
	\[
		f(x+h) = \sum_{k = 0}^\infty\frac{h^k}{k!} \cdot f^{(k)}(x).
	\]
\end{fakt}

\begin{proof}
	Różniczkujemy obie strony bez końca i porównujemy współczynniki przy $x^k$.
\end{proof}

\begin{historia}[Hensel Kurt]\end{historia}

Założenie z lematu ($f'(x) \not\equiv 0$) można osłabić, choćby do $|f(x)| < |f'(x)|^2$, ale my tymczasowo wstrzymamy się z podaniem dowodu Schönemana z 1846 roku, jako że już niedługo, choć nie do końca jestem świadom, kiedy, przetłumaczymy lemat na język algebry.

\begin{historia}[Schönemann Theodor]\end{historia} %oddkrył lemat Hensela przed Henselem, prawo wzajemności Scholza przed Scholzem oraz kryterium Eisensteina przed Eisensteinem.

Podamy pierwsze zastosowanie lematu do problemu z prawdziwego życia.
Czy możemy sklasyfikować pierwiastki jedności w dowolnym $\Q_p$?
Bezpośrednim rachunkiem można bowiem sprawdzić, że w ciele $\Q_5$ istnieje liczba $j$, której kwadrat to dokładnie $-1$.

Okazuje się, że odpowiedź na nasze pytanie jest pozytywna.
Lemat Hensela żąda od nas wielomianu, odpowiednim wyborem będzie $f(x) = x^m-1$ z pochodną $f'(x) = mx^{m-1}$.
Pochodna w punkcie $x_1$ będzie zerem (opuszczona do $\mathbb F_p$) w dwóch przypadkach: kiedy $p$ dzieli $x_1$ (wtedy i tak nie mamy żadnego pierwiastka) lub gdy $p$ dzieli $m$.
Pozostaje pierwsze założenie.

\begin{lemat}
	Niech $p \nmid m$.
	Istnieje taka całkowita $n$, że $n^m \equiv 1$ mod $p$ (ale $n \not\equiv 1$ mod $p$), wtedy i tylko wtedy gdy $m$ oraz $p-1$ nie są względnie pierwsze.
	Dla każdego $n$, najmniejsza $m$ o żądanych własnościach dzieli $p-1$.
\end{lemat}

\begin{proof}
	Jeśli $n$ istnieje, to jego rząd w $\mathbb F_p^\times$ dzieli zarówno $m$, jak i $p-1$, więc $(m, p-1) > 0$, chyba że $n \equiv 1$ mod $p$.
	Najmniejsze $m$ dzieli najmniejszą wspólną wielokrotność, a z nim też $p-1$.
	W cyklicznej grupie $\mathbb F_p^\times$ istnieje elment każdego rzędu dzielącego rząd grupy.
\end{proof}

Lemat Hensela daje:

\begin{fakt}
	Jeżeli naturalna $m$ nie dzieli się przez pierwszą $p$, to w $\Q_p$ istnieje $m$-ty pierwiastek pierwotny z jedynki, wtedy i tylko wtedy gdy $m$ dzieli $p-1$.
\end{fakt}

Nie wykluczyliśmy jeszcze istnienia $p^n$-tych pierwiastków jedności w $\Q_p$, ale uda się to po poznaniu logarytmu, gdyż sam lemat Hensela jest za słaby.
Wrócimy do nich także przy okazji poznania homomorfizmu Tate'a $\uptau$.
Jeśli zaś $m$ dzieli $p-1$, to $m$-te pierwiastki jedności są także $(p-1)$-szymi.

,,Jednostka urojona'', \prawo{Robert\\1.6.7} czyli kwadratowy pierwiastek z $-1$ w $\Q_p$ istnieje dokładnie wtedy, gdy $\frac 1 2(p-1)$ jest jeszcze parzysta, czyli dla $p$ postaci $4k+1$.

Teraz zajmiemy się kwadratami.

\begin{fakt}
	Jeśli tylko $p > 2$, to każda $p$-adyczna jedność $y \in \Z_p^\times$, dla której istnieje $z$, że $z^2 \equiv y$ mod $p \Z_p$, jest kwadratem czegoś z $\Z_p^\times$.
\end{fakt}

\begin{proof}
	Lemat Hensela dla $x^2 - y$, bo $p \neq 2$ i $y \in \Z_p^\times$ pociągają $2 z \not\equiv 0$ mod $p$.
\end{proof}

Charakteryzację łatwo przenieść na całe ciało $\Q_p$.

\begin{wniosek}
	Niech \prawo{Gouvea\\Cor.3.4.4} $p \neq 2$. 
	Element $x \in \Q_p$ jest kwadratem, wtedy i tylko wtedy gdy jest postaci $p^{2n} y^2$, gdzie $n \in \Z$, $y \in \Z_p$ jest jednością.
	Grupa ilorazowa $\Q_p^\times / (\Q_p^\times)^2$ ma rząd cztery i jest generowana przez $p$ oraz $c \in \Z_p$, element, którego redukcja modulo $p$ nie jest resztą kwadratową.
\end{wniosek}

\begin{proof}
	Własności reszt kwadratowych.
\end{proof}

Dla $\R$ jest inaczej: kwadraty to dokładnie nieujemne liczby, zaś $\R^\times/(\R^\times)^2 \cong \{-1, 1\}$ ma rząd dwa.
Co może się dziać w $\Q_2$?
Potrzebna jest mocniejsza forma lematu, albowiem $f'(x) = 2x$ jest wielokrotnością dwójki.

\begin{fakt}
	Każda \prawo{Gouvea\\Prob. 116} liczba $y \in 1 + 8\Z_2 \subseteq \Z_2$ jest kwadratem w $\Z_2$.
	Odwrotnie, $2$-adyczna jedność i kwadrat przystaje do $1$ mod $8$.
	Zatem $\Q^\times_2 / (\Q^\times_2)^2 \cong \{\pm 1, \pm 2, \pm 5, \pm 10\}$ ma rząd osiem.
\end{fakt}

\begin{proof}
	Wystarczy użyć wzmocnionego lematu.
\end{proof}

Gouvea snuje w tym miejscu opowieść o podnoszeniu rozkładów $f(x) \equiv g(x)h(x)$ mod $p$ do prawdziwych (w $\Z_p$) równości $f = g \cdot h$.
Darujemy sobie.