\subsection{Transformacja Mellina-Mazura}
Jeśli \prawo{Kbltz\\2.5} $X \subseteq \Z_p$ jest zwarty i otwarty, to można do niego obciąć miarę $\mu$ z $\Z_p$: wtedy całka z $f$ nad $X$ to całka nad $\Z_p$ funkcji $f \cdot 1_X$.
Chcieliśmy interpolować $-B_k / k$.
Mamy prosty związek $\kobint 1 \cdot \mu_B^k = \mu_B^k (\Z_p) = B_k$.
Dla różnych $k$ dystrybucje $\mu_B^k$ nie są ze sobą związane, ale ich regularne odpowiedniki tak!

\begin{wniosek}
	Niech funkcja $X \subseteq \Z_p$ będzie zwarto-otwarty.
	Określmy funkcję $f \colon \Z_p \to \Z_p$ wzorem $f(x) = x^{k-1}$.
	Wtedy
\[
	\kobint_X 1 \cdot \mu_{k, \lambda} = k \kobint_X f \mu_{1, \lambda}
\]
\end{wniosek}

\begin{proof}
	Z faktu \ref{heroine} możemy wywnioskować $\mu_{k, \alpha}(\dysk{a}{n}) \equiv ka^{k-1} \mu_{1, \alpha} (\dysk{a}{n})$ mod $p^m$ dla $m = n - v_p (d_k)$.
	Biorąc $n$ tak duże, by $X$ było unią dysków postaci $\dysk{a}{n}$, mamy
	\begin{align*}
		\kobint_X 1 \mu_{k, \lambda} & = 
		\sum_{a=0}^{p^n-1} \mu_{k, \lambda} (\dysk{a}{n}) \equiv 
		\sum_{a=0}^{p^n-1} ka^{k-1} \mu_{1, \lambda} (\dysk{a}{n}) = 
		k \sum_{a=0}^{p^n-1} f(a) \mu_{1, \lambda} (\dysk{a}{n}).
	\end{align*}
	%Sumowaliśmy po $a$ od $0$ do $p^n-1$. 
	Idziemy z $n$ do nieskończoności.
\end{proof}

Prawa strona wygląda dużo lepiej, ponieważ $k$ nie pojawia się magicznie w indeksie $\mu$, ale wykładniku ($f$).
Wiemy już, jak wygląda interpolacja $x^{k-1}$ dla ustalonego $x$: wystarczy nam, by $x \not\equiv 0$ mod $p$.
Aby wszystkie argumenty miały tę własność, weźmy $X = \Z_p^\times$.

Twierdzimy, że całkę z $x^{k-1}$ nad $\Z_p^\times$ względem $\mu_{1, \lambda}$ można interpolować.
Połączymy teraz odkrycia ustępu o przedłużaniu $s \mapsto a^s$ z wnioskami poprzedniej podsekcji.
Wnioski te mówią nam, że jeśli $|f(x) - x^{k-1}|_p \le \varepsilon$ dla $x \in \Z_p^\times$, to
\[
	\left|\kobint_{\Z_p^\times} f \mu_{1, \lambda} - \kobint_{\Z_p^\times} x^{k-1} \mu_{1, \lambda}\right|_p \le \varepsilon.
\]

Wybierzmy za $f$ funkcję $x^{l-1}$, gdzie $l \equiv k$ mod $p-1$ oraz $p^n$.
Wtedy pasujący $\varepsilon$ to na przykład $p^{-n-1}$.
%Spełniona jest nierówność $|x^{l - 1} - x^{k - 1}|_p \le  p^{-n-1}$, a to implikuje $|\kobint_{\Z_p^\times} x^{l- 1} \mu_{1, \lambda} - \kobint_{\Z_p^\times} x^{k - 1} \mu_{1, \lambda}|_p \le p^{-n-1}$.
Tak oto dochodzimy do wniosku, że dla ustalonego $0 \le s_0 \le p -2$ i $k$ biegnącego przez $S(s_0) := [s_0 + (p - 1) \Z] \cap \N_+$ możemy przedłużyć funkcję do $p$-adycznych całkowitych:
\[
	\kobint_{\Z_p^\times} x^{k-1} \mu_{1, \lambda} \rightsquigarrow \kobint_{\Z_p^\times} x^{s_0 + s(p-1) - 1} \mu_{1, \lambda}
\]

Trochę zabłądziliśmy!
Interpolowaliśmy $\kobint_{\Z_p^\times} x^{k-1} \mu_{1, \lambda} = \frac 1 k \kobint_{\Z_p^\times} 1 \mu_{k, \lambda}$.
Zwiążemy teraz te dwie liczby.
\begin{align*}
\frac 1 k \kobint_{\Z_p^\times} 1 \mu_{k, \lambda} & = \frac {\mu_{k, \lambda} (\Z_p^\times)}{k} = \frac{B_k}{k} (1 - \lambda^{-k})(1 - p^{k-1}) \\
& = (\lambda^{-k} - 1) (1 - p^{k - 1}) \cdot \kobint_{\Z_p} \frac{-1}{k} \mu_B^k.
\end{align*}

Wyraz $1 - p^{k-1}$ pojawił się, ponieważ obcięliśmy całkę z $\Z_p$ do $\Z_p^\times$.
Ten fenomen został przewidziany wcześniej: nie można przedłużyć $n^s$, gdy $p \mid n$ -- konieczne jest pozbycie się $p$-czynnika Eulera.
Zajmiemy się więc
\[
	(1 - p^{k-1})(-B_k / k) = \frac{1}{\lambda^{-k} - 1} \kobint_{\Z_p^\times} x^{k-1} \mu_{1, \lambda}
\]

Wystąpienie $k - 1$ zamiast $-k$ zaskakuje tylko początkowo, za sprawą funkcyjnego równania wiążącego coś z czymś innym.

\begin{definicja}
	Jeżeli $k \in \N_+$, to $\zeta_p(1 - k) := -(1 - p^{k - 1}) \cdot B_k/k$.
\end{definicja}

Powinniśmy w tym miejscu wyprowadzić pewne klasyczne fakty o liczbach Bernoulliego.
Uznawano je za eleganckie, ale i tajemnicze niezwykłości, zanim związano je z $\zeta_p$ Kuboty i Leopoldta oraz miarą Mazura $\mu_{1, \alpha}$.
Clausen z van Staudtem też maczali w tym palce.

\begin{fakt}
	$k \zeta(1-k) + B_k = 0$.
\end{fakt}

\begin{proof}
	%Skorzystamy z definicji $\zeta(1 - k) = \sum n^{k-1}$.
	Niech $\mathfrak D$ oznacza (lokalnie) $(\textrm{d}/ \textrm{d}t)^{k-1} \left. \ldots \right|_{t = 0}$.
	\begin{align*}
		\zeta(1-k) 
		& = \sum_{n \ge 1} n^{k-1} 
		= \sum_{n \ge 1} \mathfrak D \exp nt 
		= \mathfrak D \sum_{n \ge 1} \exp nt 
		= \mathfrak D \left[\frac{1}{1 - \exp t} - 1\right] \\
		& = - \mathfrak D \frac 1 t \sum_{n = 1}^\infty B_n \frac{t_n}{n!}
		= \mathfrak D \sum_{n=0}^\infty - \frac{B_n} n \frac{t^{n-1}}{(n-1)!} = - \frac{B_k}k. \qedhere
	\end{align*}
\end{proof}

Głębsze przemyślenia sprawiają, że dla $s \in \Z_p$ i ustalonego $s_0 \in \{0, 1, \dots, p-2\}$ ($s \neq 0$, gdy $s_0 = 0$) określamy:
\[
	\zeta_{p, s_0}(s) := (\lambda^{-(s_0 + (p-1)s)}-1)^{-1}\kobint_{\Z_p^\times} x^{s_0 + (p-1)s-1} \mu_{1,\lambda}.
\]

Jeżeli $k$ jest postaci $s_0 + (p-1)k_0$, to $\zeta_p(1-k)$ i $\zeta_{p,s_0}(k_0)$ są tym samym, więc myślimy o $\zeta_{p, s_0}$ jak o $p$-adycznych ,,gałęziach'' dla $\zeta_p$ (interesują nas parzyste $s_0$!).
Uwaga: prawa strona nie zależy (!) od $\lambda$.

Pominęliśmy przypadek $s = s_0 = 0$, bo wtedy mianownik znika, co odpowiada $\zeta_p(1)$.
Tak więc $p$-adyczna funkcja $\zeta$ też ma ,,biegun'' w $1$.

\begin{fakt}
	Dla ustalonych $p$ i $s_0$ funkcja $\zeta_{p,s_0}(s)$ jest ciągła. % i nie zależy od wyboru $\alpha$.
\end{fakt}

Pozostaje wytłumaczyć, skąd wzięła się nazwa podsekcji.
Otóż transformatą Mellina funkcji $f$ jest
\[
	(\mathcal M f)(s) = \int_0^\infty x^{s-1} f(x)\,\textrm{d}x.
\]
W $\R$-analizie transformatą $(\exp x - 1)^{-1}$ jest $\Gamma(s) \zeta(s)$, zaś $\zeta_p$ możemy traktować jako transformatę Mellina-Mazura dla zregularyzowanej miary $\mu_{1, \lambda}$.