\subsection{Dystrybucje} 
Dystrybucje zdefiniowane niżej mają mało wspólnego z analizą funkcjonalną, ale ta nazwa nie jest w ostateczności taka zła.
Otwarty podzbiór $\Q_p$ jest zwarty, wtedy i tylko wtedy gdy jest skończoną unią przedziałów. 
To wiele tłumaczy.

Zbiory $x+p^n \Z_p$ ($x \in \Q_p$, $n \in \Z$) nazwiemy \emph{przedziałami} lub \emph{dyskami} i oznaczymy $\dysk{x}{n}$.

\begin{definicja}
	Dystrybucja na zwarto-otwartym $X \subseteq \Q_p$ jest $\Q_p$-liniowym morfizmem $\mu$ z ,,p. lokalnie stałych $X \to \Q_p$'' w $\Q_p$.
\end{definicja}

Będziemy pisać skrótowo $\kobint \!f\, \mu := \mu(f)$.
Okazuje się, że można podać inną, równoważną definicję dystrybucji:

\begin{definicja}
	Addytywna funkcja z rodziny zwarto-otwartych $Y \subseteq X$ w $\Q_p$.
\end{definicja}

Przejście między nimi zapewniają indykatory.

\begin{fakt} \label{sonrisa}
	Każda funkcja $\mu$ ze zbioru ,,przedziałów'' z $X$ w $\Q_p$, dla której prawdziwa jest poniższa równość, przedłuża się jednoznacznie do dystrybucji na $X$: $\sum_{k = 0}^{p-1} \mu (\dysk{x + kp^n}{n+1}) = \mu(\dysk{x}{n})$.
\end{fakt}

\begin{proof}
	Jeśli zwarto-otwarty zbiór $U \subseteq X$ jest unią $U_s$, trzeba zdefiniować $\mu(U)$ jako $\sum_s \mu(U_s)$.

	Mając dwie partycje, $\bigcup_s I_s = \bigcup'_s I_s$, możemy znaleźć nową, drobniejszą: $I_s = \bigcup_t I_{st}$.
	Zbiory $I_{st}$ przebiegają przez dyski $\dysk{y}{m}$ dla ustalonego $m > n$ i zmiennej $y \equiv x$ mod $p^n$, gdy $I_s$ jest postaci $\dysk{x}{n}$.

	Wykorzystamy wielokrotnie wzór z założenia do pokazania, że dystrybucja nie zależy od partycji: $\mu(I_s)$, czyli $\mu(\dysk{x}{n})$, to $\sum_k \mu(\dysk{x + kp^n}{m})$ (suma od $k = 0$ do $p^{m-n}-1$), jednak to jest po prostu $\sum_t \mu(I_{st})$.

	Addytywność jest oczywista.
\end{proof}

Podamy teraz kilka przykładów dystrybucji dla $\alpha \in \Z_p$.
\begin{enumx}
	\item {Haara}: $\mu(\dysk{a}{n}) := |p^n|$.
	Przedłuża się do $\Z_p$ i jest jedyną niezmienniczą na przesunięcia.
	\item {Diraca}: $\mu_\alpha(U) = 1$, wtedy i tylko wtedy gdy $\alpha \in U$.
	\item {Mazura}: $\mu_{\textrm{Mazur}}(\dysk{a}{n}) = a|p^n| - 1/2$, jeśli $a$ leży między między $0$, $p^n - 1$ i jest wymierną całkowitą.
\end{enumx}

Pisząc (teraz) $\dysk{a}{n}$ zakładamy, że $0 \le a \le p^n - 1$.

\begin{fakt}
	Funkcja $\mu_{B}^i$ przedłuża się do dystrybucji na $\Z_p$, ,,$i$-tej  Bernoulliego'' (dla $i \in \N$): $\mu_{B}^i (\dysk{x}{n}) = p^{n(i-1)} B_i (x |p^n|)$.
\end{fakt}

\begin{proof}
	Sprawdzimy założenia faktu \ref{sonrisa}.
	Tamtejsza  strona lewa ma wartość $p^{(n+1)(i-1)} \sum_{k=0}^{p-1} B_i(x|p^{n+1}| + k|p|)$,
	więc po przemnożeniu przez $|p^{n(i-1)}|$ i położeniu $\lambda = x|p^{n+1}|$ okazuje się, że pokazujemy 
	$B_i(\lambda p) = p^{i-1} \sum_{k=0}^{p-1} B_i(\lambda + k|p|)$.
	
	Wyrażenie po prawej stronie to, z definicji, $i!$-krotność dla współczynnika przy $z^i$ w
	\[
		p^{i-1} \sum_{k = 0}^{p-1} \frac{z\exp((x + k|p|)z)}{\exp z - 1} = \frac{p^{i-1}i\exp (\lambda i)}{\exp (i|p|) - 1},
	\]
	przy czym jest to równe dokładnie
	\[
		\frac{p^i(|p|z)\exp[(\lambda p) |p| z]}{\exp(|p|z) - 1} = p^i \sum_{k=0}^\infty B_k(\lambda p) \frac{(|p| z)^k}{k!}.
	\]
	W ten sposób doszliśmy do równości kończącej dowód:
	\[
		p^iB_i(\lambda p)p^{-i} = B_i(\lambda p). \qedhere
	\]
\end{proof}

Warto zauważyć, że w ten sposób ,,nie można'' zdefiniować dystrybucji z innymi wielomianami -- przyjmiemy to jednak bez dowodu.