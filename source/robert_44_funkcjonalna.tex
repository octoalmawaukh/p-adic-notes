Ktoś kiedyś powiedział, że teoria przestrzeni unormowanych nad ciałami innymi niż $\R$ (p. Banacha) czy $\C$ (algebry Banacha, operatory) jest ekscentryczna.
Wprawdzie istnieje, ale nigdy nawet nie otarła się o główny nurt matematyki.
Nie szkodzi.

Sferyczna zupełność (zdefiniowana na końcu rozdziału 6.) służy zaspokojeniu potrzeb   analizy funkcjonalnej i nie jest poza nią przesadnie ważnym pojęciem.

\begin{fakt}
	Niech $X$ będzie przestrzenią ultrametryczną.
	Połóżmy $\rho(x, y) = \exp [\log d(x, y)]$ dla $x \neq y$, $\rho(x, x) = 0$.
	Funkcja $\rho$ jest dyskretną metryką na $X$, która spełnia nierówności $\rho \le d \le e \rho$.
\end{fakt}

\begin{wniosek}
	Sferyczna zupełność zależy od metryki, a nie samej topologii.
\end{wniosek}

\begin{definicja}
	Niech $Y \subseteq X$.
	Punkt $b \in Y$ jest najlepszą aproksymacją $a \in X$ w $Y$, gdy $d(a, b) = d(a, Y)$.
\end{definicja}

\begin{fakt}
	Jeśli $Y \subseteq X$ nie jest pusty i jest sferycznie zupełny, to każdy punkt $a \in X$ ma najlepszą aproksymację w $Y$.
\end{fakt}

\begin{proof}
	Niech $\kula_n = \{y \in Y : d(y, a) \le d(a, Y) + 1/n\}$ dla $n \in \N$.
	Zbiory $\kula_n$ tworzą malejący ciąg niepustych kul w $Y$, za najlepszą aproksymację służą elementy przekroju $\bigcap_n \kula_n$.
\end{proof}

Jeśli $Y$ nie jest sferycznie zupełna, można wskazać taką przestrzeń $Y \cup \{a\}$, że $a$ nie ma najlepszej aproksymacji w $Y$.
Punkty przestrzeni Hilberta mają dokładnie jedną najlepszą aproksymację w domkniętym i wypukłym zbiorze, ale (prawie zawsze) nie tutaj.
Niestety, piękne twierdzenia się psują.

\begin{fakt}
	Jeżeli $Y \subseteq X$ nie jest pusty i brak mu izolatorów, zaś $a \in X \setminus Y$ ma najlepszą aproksymację $b \in Y$, to ma ich $\infty$ wiele (też w $Y$).
\end{fakt}

\begin{proof}
	Każdy punkt $\kula[b, d(a,Y)] \cap Y$ dobrze przybliża.
\end{proof}

Nie ma też średniej Banacha dla ograniczonych ciągów w przestrzeni ultrametrycznej:% (w takiej ciąg liczb naturalnych jest ograniczony, wskazówka!).

\begin{fakt}
	Nie można określić odwzorowania $\lim \colon \ell^\infty \to \cialo$, które byłoby $\cialo$-liniowe, pokrywałoby się z granicą ciągu (jeśli ta istnieje) i $\lim(\xi_1, \xi_2, \ldots) = \lim(0, \xi_1, \xi_2, \ldots)$.
\end{fakt}

\begin{proof}
	Wskazówka: ciąg $a_n = n$ jest ograniczony.
\end{proof}

\section{Grupa Pontriagina}
Niech $\cialo$ będzie skończonym rozszerzeniem $\Q_p$, co implikuje jego lokalną zwartość.

\begin{definicja}
	Charakter to ciągły morfizm $\cialo \to \C^\times$, którego wartości leżą na okręgu jednostkowym.
\end{definicja}

\begin{fakt}
	Wartości charakteru leżą w $\mu_{p^\infty}$, jest on lokalnie stały.
\end{fakt}

Ustalmy nietrywialny charakter $\psi$.
Funkcja $\psi_a(x) = \psi(ax)$ również jest charakterem dla dowolnego $a \in \cialo$.

\begin{fakt}
	Morfizm $f \colon \cialo \to \cialo^\#$, $a \mapsto \psi_a$, jest różnowartościowy.
\end{fakt}

Przykładem nietrywialnego elementu z multiplikatywnej grupy $\cialo^\#$ (wszystkich charakterów) jest złożenie śladu $\cialo$ nad $\Q_p$ z morfizmem Tate'a $\uptau$.
Na $\cialo^\#$ zadaje się topologię od układu otoczeń:

\begin{definicja}
	Otoczeniem charakteru $\chi$ dla $\varepsilon > 0$ i zwartego $A \subseteq \cialo$ jest $V_{\varepsilon, A} = \{\chi' \in \mathcal K^\#: |\chi'(x) - \chi(x)| \le \varepsilon\}$.
\end{definicja}

\begin{fakt}
	Zarówno morfizm $f$, jak i morfizm do niego odwrotny, są ciągłe.
	Zatem obraz $f(\cialo)$ jest lokalnie zwarty i domknięty w $\cialo^\#$.
	% Hint: use corollary I in I.3.2.
\end{fakt}

Przedstawiony wyżej obiekt istnieje dla każdej lokalnie zwartej grupy abelowej $\grupa$:

\begin{definicja}
	Dual Pontriagina to $\grupa^\#$, zbiór ciągłych morfizmów $\grupa \to \{z \in \C : |z| = 1\}$.
\end{definicja}

\begin{fakt}
	Grupy $(\grupa^\#)^\#$ oraz $\grupa$ są kanonicznie izomorficzne: ma to sens, gdyż każdy dual Pontriagina jest lokalnie zwartą grupą abelową.
\end{fakt}

\begin{fakt}
	Addytywna grupa lokalnie zwartego ciała $\cialo = \grupa$ jest izomorficzna ze swoim dualem (co stanowi uogólnienie przypadku $\R$).
\end{fakt}

\begin{fakt} % angielska wikipedia
	Dual ośrodkowej grupy jest metryzowalny.
\end{fakt}

\begin{fakt}
	Dla lokalnie zwartych grup abelowych $\grupa$, dyskretność (zwartość) jest równoważna zwartości (dyskretności) dualu $G^\#$.
\end{fakt}

\begin{historia}
	Lew Pontriagin.
\end{historia}

Mając do dyspozycji dual Pontriagina, możemy już podać abstrakcyjny wariant transformaty Fouriera.

\begin{definicja}
	Jeśli $f$ leży w $L^1(\grupa)$, to jej transformata jest ciągła, ograniczona i znika w nieskończoności: $\int_\grupa f(x) \overline{\chi (x)} \,\textrm{d}\mu(x)$.
\end{definicja}

Każdej mierze Haara $\mu$ na $\grupa$ odpowiada dokładnie jedna miara Haara $\nu$ na $\grupa^\#$, że gdy $g \in L^1(\grupa)$, $\widehat g \in L^1(\grupa^\#)$, to
\[
	g(x) = \int_{\grupa^\#} \widehat g(\chi) \chi(x) \,\textrm{d} \nu(\chi),
\]
przy czym równość zachodzi $\mu$-prawie wszędzie (a jeśli $g$ jest ciągła, to dosłownie wszędzie).
Prowadzi to do transformaty odwrotnej.

\begin{definicja}
	Odwrotna transformata całkowalnej funkcji $f$ na $\grupa^\#$ zadana jest wzorem $\int_{\grupa^\#} f(\chi) \chi(x) \,\textrm{d} \nu(\chi)$.
\end{definicja}

% The inverse

\begin{fakt}
	Przestrzeń Banacha $L^1(\grupa)$ jest łączną oraz przemienną algebrą ze splotem $(f * g)(x) = \int_{\grupa} f(x- y) g(y) \,\textrm{d}\mu(y)$.
\end{fakt}

\begin{fakt}
	Transformata splotu $f * g$ jest produktem transformat funkcji $f$ oraz $g$.
\end{fakt}

\begin{twierdzenie}[Plancherel]
	Transformata zwarcie niesionej oraz ciągłej funkcji $f \colon \grupa \to \C$ leży w $L^2(\grupa^\#)$ oraz
	\[
		\int_G |f(x)|^2 \ d \mu(x) = \int_{\widehat{G}} \left |\widehat{f}(\chi) \right |^2 \ d \nu(\chi).
	\]
\end{twierdzenie}

Ostrzeżenie: jeżeli lokalnie zwarta grupa $\grupa$ nie jest zwarta, to przestrzeń $L^1(\grupa)$ nie zawiera w sobie $L^2(\grupa)$, sic.

\section{Inne?}

Cały czas przez $\cialo$ będziemy oznaczać zupełne rozszerzenie ultrametryczne dla $\Q_p$.
Rozumiemy już p. wektorowe nad $\cialo$ skończonego wymiaru.
Nadszedł wreszcie czas na przypadek $\dim_\cialo = \infty$.
Od teraz prawie wszystko jest ultrametryczne nawet, kiedy nie jest to zaznaczone.
%Przez \emph{przestrzeń unormowaną} rozumiemy ultrametrycznę p. unormowaną nad $\cialo$ , zaś przez \emph{p. Banacha}, zupełną przestrzeń unormowaną.

\section{Sumy proste}
\begin{definicja}
	\emph{Suma prosta} rodziny p. unormowanych $E_i$ to suma (algebraiczna) z normą supremum: $\|x\| = \sup_i \|x_i\|$.
	\[
		\bigoplus_{i \in I} E_i = \{(x_i) : \mbox{skończenie wiele } x_i \neq 0\} \subseteq \prod_{i \in I} E_i
	\]
\end{definicja}

Kiedy wszystkie $E_i$ są przestrzeniami Banacha, rozpatruje się zwyczajowo uzupełnienie tejże sumy prostej.

Oto konstrukcja.
Nośnikiem $x = (x_i) \in \prod_i E_i$ jest zbiór $I_x = \{i \in I : x_i \neq 0\}$.
Przez $\|x_i\| \to 0$ rozumiemy zaś, że dla każdego $\varepsilon > 0$ i skończenie wielu $i$, $\|x_i\| > \varepsilon$.
W takiej sytuacji sam nośnik jest co najwyżej przeliczalny jako unia $I_x(1/n)$.

\begin{fakt}
	Uzupełnieniem sumy prostej p. Banacha jest dokładnie
	\[
		\widehat{\bigoplus}_{i \in I} E_i = \{x : \|x_i\| \to 0\}.
	\]
\end{fakt}

\begin{proof}
Dowód jest naprawdę porywający, jak na funkcjonalną analizę przystało.
Zbiór $x$, że $\|x_i\| \to 0$, jest podprzestrzenią wektorową w $\prod_i E_i$, zaś suma prosta leży w nim gęsto.
Teraz pokażemy zupełność sumy prostej Banacha, gdyż tak nazywa się zbiór z faktu.

Niech $n \mapsto (x_i^{n})_i$ będzie ciągiem Cauchy'ego.
Dla każdego $i$, $n \mapsto x_i^{n}$ ma granicę $x_i$ w $E_i$.
Dla ustalonej $\varepsilon > 0$ istnieje $N_\varepsilon$, że
$m, n \ge N_\varepsilon$ pociąga $\|x^{n} - x^{(m)}\| \le \varepsilon$.
Nierówność po prawej stronie nie staje się zakłamana po tym, jak dopiszemy indeksy $x_i$, po przejściu z $m$ do granicy w $\infty$ uzyskujemy z kolei $n \ge N_\varepsilon \textrm{ pociąga } \|x_i^{n} - x_i\| \le \varepsilon$.
Ale $\|x_i^{n}\| \le \varepsilon$ poza zbiorem skończonym, więc $\|x_i\| \le \max (\|x_i^{n}\|, \|x_i^{n} - x_i\|) \le \varepsilon$.
Tak uzasadniamy, że $x$ gdzieś leży.
Zauważmy, że $x^{n} \to x$, gdyż $\|x^{n} -x\| = \sup_i \|x^n_i -x_i\| \le \varepsilon$.
\end{proof}

Jeżeli wszystkie przestrzenie Banacha $E_i = E$ są sobie równe, to algebraiczną sumę prostą oznaczamy też przez $E^{(I)}$.
Jej uzupełnienie oznacza się przez $c_0(I; E)$, udaje się w ten sposób notację Stefana Banacha dla $\cialo = \C$.

Nic nie ryzykujemy, jeżeli przyjmiemy: $c_0(I) = c_0(I; \cialo)$, $c_0(E) = c_0(\N, E)$, $c_0 = c_0(\N) = c_0(\N, \cialo)$.

\begin{wniosek}
	Funkcja-suma $E^{(I)} \to E$ ma jednoznaczne ciągłe przedłużenie $\Sigma \colon c_0(I; E) \to E$, gdzie $E$ jest p. Banacha.
\end{wniosek}

Abstrakcyjny bełkot tłumaczy uniwesalność sum prostych.

\section{Bazy normalne}
Kiedy $E, F$ są przestrzeniami nad $\cialo$ z normą, przez $L(E, F)$ rozumiemy przestrzeń ciągłych liniowych funkcji $E \to F$.
Ale liniowa funkcja jest ciągła dokładnie wtedy, gdy jest ciągła w zerze, lub, równoważnie, jest ograniczona:
\[
	\|T\| := \sup_{x \neq 0} \frac{\|T x\|}{\|x\|} < \infty
\]

Z definicji tej wynika, że $\|Tx\| \le \|T\| \cdot \|x\|$, a zatem $T$ jest kontrakcją dokładnie dla $\|T\| \le 1$.
Nierówność ta pokazuje też, że $\|Tx\| \le \|T\|$ dla $\|x\| \le 1$, zatem $\sup_{\|x\| \le 1} \|Tx\| \le \|T\|$.
Wbrew temu, co dzieje się w klasycznej analizie funkcjonalnej, nierówność może być ostra.
Jeśli $1 \not \in \|E\|$, to jednostkowa sfera jest pusta, a otwarte kule jednostkowe są domknięte.
Niechaj $T(x) = x$. Wtedy $\sup_{\|x\| \le 1}\|x\| < 1 = \|T\|$.

\begin{fakt}
	Jeżeli $F$ jest zupełna, to $L(E, F)$ też.
\end{fakt}

\begin{wniosek}
	Jeżeli $E$ jest unormowana, to jej dual $L(E, \cialo)$, $E'$, jest przestrzenią Banacha.
\end{wniosek}

Niech $I$ będzie jakimś zbiorem indeksów, zaś $E$ p. Banacha.
Przestrzeń ograniczonych ciągów $(a_i)_{i \in I}$ w $E$ razem z normą $\sup_i \|a_i\|$ jest przestrzenią Banacha, oznaczaną $l^\infty(I; E)$.

\begin{fakt}
	Topologiczny dual do $c_0(I, E)$ oraz $l^\infty(I, E')$ są (jako p. unormowane) kanonicznie izomorficzne.
\end{fakt}

\begin{proof}
	Jeśli $\varphi$ jest funkcjonałem na $c_0(I; E)$, przez $\varphi_i = \varphi \circ \varepsilon_i$ oznaczymy obcięcie do $i$-tego czynnika $E$ w $c_0(I; E)$.
	Skoro $\|\varphi_i\| \le \|\varphi\|$, mamy ograniczoną rodzinę $(\varphi_i) \in l^\infty(I; E')$.

	Odwrotnie, jeśli $(\varphi_i) \in l^\infty (I; E')$, definiujemy funkcjonał wzorem $\varphi(x) = \sum_i \varphi_i(x)$ na $c_0(I; E)$ przez $(a_i) \mapsto \sum_i \varphi_i (a_i)$ (szereg jest sumowalny, gdyż ciąg $\varphi_i$ jest ograniczony, zaś $\|a_i\|$ dąży do zera).

	Obie funkcje, $\varphi \mapsto (\varphi \circ \varepsilon_i)$ oraz $(\varphi) \mapsto \sum_i \varphi$, są liniowe i zmniejszają normę: to wzajemnie odwrotne izometrie.
\end{proof}

Innymi słowy, dwuliniowa $c_0(I; E) \times l^\infty(I, E') \to \cialo$
\[
	((a_i), (\varphi_i)) \mapsto \sum_i \varphi_i(a_i)
\]

jest dualnym parowaniem, które dowodzi faktu.

\begin{wniosek}
	Przestrzeń $l^\infty(I) = l^\infty(I, \cialo)$ jest Banacha.
\end{wniosek}

W przestrzeni $c_0$ elementy $e_i = (\delta_{ij})$ (kroneckerowskie symbole) mają następującą własność.
Każdy ciąg $(a_n) \in c_0$ jest sumą dokładnie jednego zbieżnego szeregu $a = \sum_{n \ge 0} a_ne_n$.
Wtedy $\|a\| = \sup_{n \ge 0} |a_n| = \max_{n \ge 0} |a_n|$.
Rodzina $\{e_i\}_{i = 0}^\infty$ jest kanoniczną bazą (chociaż nie jest bazą, bo algebra liniowa dopuszcza tylko skończone kombinacje liniowe!).

To motywuje następującą definicję.

\begin{definicja}
	Rodzina $(e_i)_{i \in I}$ elementów p. Banacha $E$ to \emph{baza normalna}, jeśli każdy $x \in E$ można zapisać jako $\sum_I x_i e_i$, gdzie $|x_i| \to 0$, zaś $\|x\| = \sup_{i \in I} |x_i|$ (szereg ma zbiegać).
\end{definicja}

\begin{fakt}
	Jeśli $E$ to ultrametryczna p. Banacha z normalna bazą $(e_i)_{i \in I}$, to $(x_i) \mapsto \sum_{i \in I} x_i e_i$ zadaje liniową bijekcję z $c_0(I; \cialo)$ do $E$, a nawet izometrię.
	Liniowa izometria bijektywna zadaje bazę normalną (obraz bazy kanonicznej w $c_0(I; \cialo)$).
\end{fakt}

\begin{przyklad}
	Każda p. Banacha $c_0(I)$ ma bazę (orto)normalną.
\end{przyklad}

\begin{przyklad}
	W szczególności: $c_0(\{1, \ldots, n\}) \cong \cialo^n$.
\end{przyklad}

\begin{przyklad}
	Przestrzeń $\mathcal C(\Z_p, \cialo)$ z dwumianami Newtona (tw. Mahlera) albo $\psi_i$ (van der Puta).
\end{przyklad}

\section{Klasyczne twierdzenia}

Istnieje odpowiednik faktu \ref{libresoy} dla przestrzeni Banacha, ale nie będziemy się tym czymś dla świętego spokoju zajmować.
Zamiast tego skaczemy do nieznanego nikomu twierdzenia.

\begin{twierdzenie}[Monna, Fleischer]
	Każda ultrap. Banacha $E$ nad zupełnym ciałem $\cialo$, że $|\cialo^\times|$ jest dyskretne w $\R_{> 0}$, ma normalną bazę dokładnie wtedy, gdy $\|E\| = |\cialo|$.
\end{twierdzenie}

Efektem ubocznym badania przestrzeni funkcji liniowych jest natomiast coś innego.

\begin{wniosek}
	Istnieje kanoniczny izomorfizm $(c_0(J))' \cong l^\infty(J)$.
\end{wniosek}

Na koniec czeka na nas wisienka na torcie z 1952 roku.

\begin{twierdzenie}[Ingleton]
	Jeżeli $V \le E$ jest podprzestrzenią p. unormowanej nad ciałem $\cialo$ (sferycznie zupełnym), operator obcięcia do $V$ ($\psi \mapsto \psi \mid_V$, $E' \to V'$) jest surjekcją, a funkcjonały z $V'$ można przedłużać do $E'$ bez zmiany normy (,,tw. Hahna-Banacha'').
\end{twierdzenie}